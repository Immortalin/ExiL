%%%%%%%%%%%%%%%%%%%%%%%%%%%%%%%%%%%%%%%%%%%%%%%%%%%%%%%%%%%%%%%%%%%%%%%%%%%%%%%%
\section{Experní systém}
Expertní systém je počítačový program, který se pokouší nalézt řešení problému
v~situaci, kde by jinak bylo zapotřebí jednoho či několika odborníků.
Idea expertního systému byla představena vývojáři ze Standfordského projektu
\emph{Heuristic programming} v~čele s~Edwardem A. Feigenbaumem na systémech
\textsf{Dendral} a \textsf{MYCIN}. Hlavními přispěvateli tohoto projektu byli
Bruce Buchanan, Edward Shortliffe, Randall Davis, William vanMelle a Carli
Scott (\cite{expert-system}).

Expertní systém patří do skupin KBS (Knowledge-Based System, \uv{systém
založený na znalostech}) a RBS (Rule-Based System, \uv{systém založený na
pravidlech}), spadá tedy do odvětví informatiky označované jako umělá
inteligence.

Vývojář expertního systému nejprve definuje vhodnou reprezentaci znalostí nějaké
domény a zákonitostí mezi těmito znalostmi. Poté probíhá proces učení systému --
odborníci na danou problematiku buď sami, nebo (časteji) skrze znalostního
inženýra (\emph{knowledge engineer}) zadají systému základní informace
o~problematice, čímž sestaví tzv. vědomostní bázi (\emph{knowledge base})
a bázi pravidel (\emph{rule base}). V~provozu pak systém ze zadaných znalostí
pomocí sestavených pravidel odvozuje nová fakta. Díky tomu je jej možno využít
v~praxi jako asistenta odborníka v~daném oboru, jehož práce je tímto značně
zefektivněna, nebo odborníka, v~ideálním případě, zcela nahradit.

Pro lepší představu uvedu příklad -- program na bázi expertního systému
v~ordinaci praktického lékaře. Na začátku zadá lékař expertnímu systému
informace o~příznacích známých chorob, možnostech jejich léčby, medikaci,
konfliktních lécích, atd. Expertní systém potom při zadání příznaků pacienta
s~jistou pravděpodobností (odvislou od pravděpodobností zadaných v~jednotlivých
pravidlech příznak $\rightarrow$ choroba) určí možné příčiny, alternativy léčby,
apod. Takto nějak právě funguje zmíněný systém \textsf{MYCIN}.
%%%%%%%%%%%%%%%%%%%%%%%%%%%%%%%%%%%%%%%%%%%%%%%%%%%%%%%%%%%%%%%%%%%%%%%%%%%%%%%%
\subsection{Charakteristické vlastnosti expertních systémů}
Expertní systémy typicky
\begin{itemize}
\item simulují lidské usuzování na základě dedukce
  -- systém neprovádí nad množinou znalostí algebraické výpočty
\item usuzují nad libovolnou množinou správně reprezentovaných znalostí
  -- tato znalostní báze je zcela oddělena od odvozovacího (\emph
  {inferenčního}) mechanismu a není na něm nijak závislá
\item problémy řeší heuristickým přístupem, často také jen na
  základě pravděpodobností -- úspěch tedy není vždy zaručen. Díky tomu tyto
  systémy na druhou stranu nevyžadují perfektní data, pokud se uživatel
  spokojí s~jistou pravděpodobností správného výsledku
\end{itemize}
--- Citováno z~\cite{introduction}.
%%%%%%%%%%%%%%%%%%%%%%%%%%%%%%%%%%%%%%%%%%%%%%%%%%%%%%%%%%%%%%%%%%%%%%%%%%%%%%%%
\subsection{Dopředné a zpětné řetězení}
\begin{itemize}
\item systém s~\textbf{dopředným řetězením} vyvozuje z~poskytnutých dat možné
závěry -- jako v~systém asistující lékaři ve výše uvedeném příkladu --
lékař při zadávání dat systému ještě pochopitelně neví, co bude výsledkem
-- jaké příčiny problému systém vyhodnotí a jaké možnosti léčby pacientu doporučí
\item u~systému se \textbf{zpětným řetězením} je tomu naopak -- systému
zadáme cíle, jichž bychom rádi dosáhli, např. \uv{rád bych založil softwarovou
firmu}. Vědomostní báze systému musí disponovat informacemi o~tom, co je
pro založení takové firmy potřeba, tyto informace budou pravděpodobně větvené
-- pro firmu potřebuji prostory, zaměstnance, potřebné úřední formality.
Vyřízení formalit obnáší podnikatelský záměr, návrh rozvahy rozpočtu, \ldots.
Systém z~těchto dat vyhodnotí, zda je zadaný cíl možno rovnou provést, či
co je k~jeho splnění ještě potřeba. V~příkladu u~lékaře by byl takovýto
systém nepoužitelný.
\end{itemize}
