\subsubsection{Modifikace pracovní paměti}
\label{modifikace}

Pracovní paměť je množina faktů, které systém v danou chvíli považuje za platné.
Její obsah můžeme vypsat voláním funkce \verb|facts|. Funkcí \verb|reset|
inicializujeme pracovní paměť ze znalostní báze. Jejím voláním jsou do pracovní
paměti zavedeny fakty všech definovaných skupin faktů (viz kapitola
\ref{knowledge base definition}).

Obsah pracovní paměťi může být dále modifikován třemi makry:
\begin{itemize}
  \item \verb|assert| přidává fakt(y) do pracovní paměti,
  \item \verb|retract| fakt(y) z pracovní paměti odebírá a
  \item \verb|modify| přímo modifikuje existující fakt.
\end{itemize}
Ta lze volat buď před započetím inference (ale po volání \verb|reset|, neboť to
dodatečné úpravy vymaže), nebo v jejím průběhu, pokud inferenci krokujeme (viz
kapitola \ref{inference}). Makra také typicky voláme v důsledcích pravidel.

Makra \verb|assert| a \verb|retract| berou jako parametry libovolný počet
specifikací faktů ve stejném formátu, jako u makra \verb|deffacts| (ale bez
názvu skupiny). Makro \verb|modify| lze použít jen u strukturovaných faktů. Toto
makro bere jako první parametr specifikaci faktu, zbytek parametrů tvoří plist
určující hodnoty slotů ke změně. Např.
\begin{minted}{cl}
(modify (in :object box :location A) :location B)
\end{minted}
nahradí v pracovní paměti fakt \verb|(in :object box :location A)| faktem
\verb|(in :object box :location B)|.
Toto makro je obzvláště užitečné, navážeme-li v podmínkách pravidla celý fakt na
proměnnou (viz kapitola \ref{inference}).

Pracovní paměť se skutečně chová jako množina, každý fakt tedy může být v
pracovní paměti jen jednou. Opětovné volání \verb|assert| nemá žádný efekt.
Volání modify, jehož výsledkem by bylo nahrazení nějakého faktu faktem, který
již v~pracovní paměti existuje, sice odebere původní fakt, nový však znovu
nepřidá.

Všechny fakty můžeme z pracovní paměti odebrat voláním \verb|(retract-all)|. To
je ale zřídka užitečné, typicky použijeme spíše funkci \verb|reset| pro
navrácení pracovní paměti do výchozího stavu.
