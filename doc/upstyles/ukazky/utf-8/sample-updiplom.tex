%%%  Vzor pro použití makra pro diplomovou práci
%%%  (c) Miloš Kudělka, David Skoupil, březen 1998
%%%  Vzorový soubor revidován a doplněn v září 2001
%%%  (c) 2001 Vilém Vychodil, <vilem.vychodil@upol.cz>
%%%  Vzorový soubor upraven v květnu 2009
%%%  (c) 2009 Jan Outrata, <jan.outrata@upol.cz>
%%%
%%%  Po přeložení programem CSLaTeX (třikrát) je potřeba použít
%%%  program DVIPS a takto získaný PostScriptový soubor vytisknout
%%%  na PostScriptové tiskárně nebo pomocí programu GhostScript.
%%%
%%%  Rovněž je možné použít program DVIPDFM a vytvořit z dokumentu
%%%  soubor ve formátu PDF včetně hypertextových odkazů.


%%% Deklarace hlavičky dokumentu, použijte písmo velikosti 12 bodů.
\documentclass[12pt]{article}

%%% Připojení dodatečného stylu pro diplomové práce. V případě
%%% (magisterské) diplomové práce použijte nepovinný argument
%%% `master', který zajistí vysázení správného označení práce
%%% ``DIPLOMOVÁ PRÁCE'' na titulní straně (výchozí je ``BAKALÁŘSKÁ
%%% PRÁCE'').
%%%
%%% Nepovinné argumenty `tables' a `figures' použijte pouze v případě,
%%% že váš dokument obsahuje tabulky a obrázky a chcete vytvořit
%%% jejich seznamy za obsahem.
%%%
%%% Argument `joinlists' způsobí zřetězení obsahu a seznamů tabulek a obrázků.
%%% Není-li použít, všechny seznamy jsou uvedeny na samostatných stránkách.
%%%
%%% Pokud chcete vytvářet pouze dokument ve formátu PostScript, můžete uvést
%%% dodatečný argument `nopdf'. Tím se potlačí chybová hlášení při použití
%%% programu `dvips'.
\usepackage[tables,figures]{updiplom}

%%% Dodatečné standardní styly.
\usepackage[utf8]{inputenc}

%%% Parametry pro vytvoření úvodních stránek. Makrem \subtitle je možné
%%% vytvořit druhý řádek v názvu diplomové práce.
\title{Název diplomové práce}
%\subtitle{Druhý řádek názvu diplomové práce}
\author{Jan Novák}
\year{2009}
\date{21. květen 2009}

%%% Pomocí \docinfo je možné vytvořit název pro PDF dokument, zpravidla je
%%% dobré použít předcházející název, ale bez diakritiky. Možné je však zvolit
%%% úpolně jiný výstižný název. Při tvorbě PostScriptu bude příkaz ignorován.
\docinfo{Jan Novak}{Nazev diplomove prace}

%%% Vytvoření anotace. Pouze jeden odstavec!
\annotation{%
Anotace stručně popisuje zpracovanou práci a neměla by
 přesáhnout zhruba 10~řádků. V~žádném případě by neměla být rozdělena
do více odstavců.}

%%% Nepovinný text poděkování. Pouze jeden odstavec!
\thanks{%
Poděkování vedoucímu práce, rodině apod. (nepovinné)}

\begin{document}

%%% Vytvoření úvodních stránek, obsahu a seznamu tabulek a obrázků.
\maketitle
\newpage


%%% Text diplomové práce.
\section{První kapitola}
Toto je první kapitola.

\medskip
\suboutline{Vložená záložka}
Na toto místo směřuje záložka vytvořená pomocí makra \verb|\suboutline|.

\subsection{První podkapitola}
Toto je první podkapitola.
Zde jsem vycházel z~prací docenta Novotného. Lze jej kontaktovat
na adrese \mail{martin.novotny@diff.rov.cz}.


\subsection{Druhá podkapitola}
\hyplabel{dotextu}
Toto je druhá podkapitola. Sem povede odkaz z~textu.


\newpage
\section{Druhá kapitola}
Toto je druhá kapitola a budu se zde zabývat tím, jak vypadá
program podle~\cite{smith}.

\subsection{První podkapitola}\label{kapX}
Toto je text podkapitoly číslo~\ref{kapX}
\begin{table}[ht]
  \begin{center}
    \renewcommand{\arraystretch}{1.2}
    \begin{tabular}{||l|rr||}
      \hline
      & \multicolumn{2}{|c||}{\bf \hbox{Informace}} \\
      \cline{2-3}
      \bf Sloupec 1 & \bf Sloupec 2 & \bf Sloupec 3 \\
      \hline
      Buňka 1 & Buňka 2 & Buňka 3 \\
      Buňka 4 & Buňka 5 & Buňka 6 \\
      Buňka 7 & Buňka 8 & Buňka 9 \\
      \hline
    \end{tabular}
    \caption{Toto je tabulka.} \label{tab}
  \end{center}
\end{table}

\nextoutline{Jiné jméno}
\subsection{Další podkapitola}
Toto je další podkapitola. Tato podkapitola bude dále členěna.
Tato podkapitola se bude v~záložkách jmenovat jinak.

\subsubsection{Podkapitola}\label{podkapX}
Tato podkapitola má číslo~\ref{podkapX}
Zde budu řešit programování podle knihy~\cite{kovar}.
V~této části je i~obrázek, viz~\ref{obr}

\begin{figure}[ht]
  \centerline{\epsfbox{uplogo.eps}}
  \caption{Toto je obrázek.} \label{obr}
\end{figure}

\subsubsection{Podkapitola}\label{podkapY}
Tato část má číslo~\ref{podkapY} a~je umístěna na stránce \pageref{podkapY}.
Na stránce uvedu i~odkaz na URL, třeba \url{http://www.inf.upol.cz}.
Dále lze vytvářet i~odkazy na kapitoly \emphref{v~rámci dokumentu}{kapX}.
Stejně tak lze vytvářet i~odkazy \emphref{přímo do textu}{dotextu},
pro vytvoření návěstí je potřeba použít makro \verb|\hyplabel|.

\medskip
Zde budu řešit programování podle knihy~\cite{kovar}.


%%% Závěr práce v~češtině
\begin{conclusions-cz}
  Závěr práce v češtině.
\end{conclusions-cz}


%%% Závěr práce v~angličtině
\begin{conclusions-en}
  Conclusion in english.
\end{conclusions-en}


%%% Vytvoření seznamu literatury.
\newpage
\begin{thebibliography}{99}

\bibitem{smith} Smith, John. \emph{User and program.}
                Publisher, City, 1990.
\bibitem{kovar} Kovář, Jan. \emph{Jak programovat.}
                Nakladatelství, Město, 1990.
\bibitem{slozi} Novotný, Martin.
                \link{\emph{Překladač s~nakladačem.}}{http://www.inf.upol.cz}
                Elektronická publikace, 2001.

\end{thebibliography}


%%% Přílohy.
\newpage
\appendix

\section{První příloha} \label{PrvniPriloha}
Závěrečné poznámky, k~programování.

\newpage
\section{Obsah přiloženého CD} \label{ObsahCD}
V~samotném závěru práce je uveden stručný popis obsahu přiloženého
CD/DVD, tj. závazné adresářové struktury, důležitých souborů apod.

\begin{description}

\item[\texttt{bin/}] \hfill \\
Instalátor \textsc{Instalator} programu a další program
\textsc{Program} spustitelné přímo z CD/DVD. / Kompletní adresářová
struktura webové aplikace \textsc{Webovka} (v ZIP archivu) pro
zkopírování na webový server. Adresář obsahuje i všechny potřebné
knihovny a další soubory pro bezproblémové spuštění programu / pro
bezproblémový provoz na webovém serveru.

\item[\texttt{doc/}] \hfill \\
Dokumentace práce ve formátu PDF, vytvořená dle závazného stylu KI PřF
pro diplomové práce, včetně všech příloh, a všechny soubory nutné pro
bezproblémové vygenerování PDF souboru dokumentace (v ZIP archivu),
tj. zdrojový text dokumentace, vložené obrázky, apod.

\item[\texttt{src/}] \hfill \\
Kompletní zdrojové texty programu \textsc{Program} / webové aplikace
\textsc{Webovka} se všemi potřebnými (převzatými) zdrojovými texty,
knihovnami a dalšími soubory pro bezproblémové vytvoření spustitelných
verzí programu / adresářové struktury pro zkopírování na webový server
(v ZIP archivu).

\item[\texttt{readme.txt}] \hfill \\
Instrukce pro instalaci a spuštění programu \textsc{Program}, včetně
požadavků pro jeho provoz. / Instrukce pro nasazení webové aplikace
\textsc{Webovka} na webový server, včetně požadavků pro její provoz, a
webová adresa, na které je aplikace nasazena pro testovací účely a pro
účel obhajoby práce.

\end{description}

Navíc CD/DVD obsahuje:

\begin{description}

\item[\texttt{data/}] \hfill \\
Ukázková a testovací data použitá v práci a pro potřeby obhajoby
práce.

\item[\texttt{install/}] \hfill \\
Instalátory aplikací, knihoven a jiných souborů nutných pro provoz
programu / webové aplikace, které nejsou standardní součástí operačního
systému.

\item[\texttt{literature/}] \hfill \\
Některé položky literatury odkazované z dokumentace práce.

\end{description}

U veškerých odjinud převzatých materiálů obsažených na CD/DVD jejich
zahrnutí dovolují podmínky pro jejich šíření nebo přiložený souhlas
držitele copyrightu. Pro materiály, u kterých toto není splněno, je
uveden jejich zdroj (webová adresa) v textu dokumentace práce nebo v
souboru \texttt{readme.txt}.

\end{document}
