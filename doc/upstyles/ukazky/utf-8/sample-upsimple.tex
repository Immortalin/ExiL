%%%  Vzor pro použití základní sady maker `upsimple'
%%%  (c) 2001 Vilém Vychodil, <vilem.vychodil@upol.cz>
%%%
%%%  Po přeložení programem CSLaTeX (třikrát) je potřeba použít
%%%  program DVIPS a takto získaný PostScriptový soubor vytisknout
%%%  na PostScriptové tiskárně nebo pomocí programu GhostScript.
%%%
%%%  Rovněž je možné použít program DVIPDFM a vytvořit z dokumentu
%%%  soubor ve formátu PDF včetně hypertextových odkazů.


%%% Deklarace hlavičky dokumentu, je třeba použít styl `article'. Ostatní
%%% argumenty jsou zcela libovolné. V tomto případě je použito písmo velikosti
%%% jedenácti bodů.
\documentclass[11pt]{article}

%%% Připojení dodatečného stylu pro diplomové práce. Nepovinné argumenty
%%% `index', `outlines' a `czech' lze použít pro ovlivnění chování stylu.
%%%
%%% Přepínač `index' umožňuje vytvářet hypertextový rejstřík. Není-li argument
%%% uveden, bude dokument v případě použití indexu obsahovat klasicky 
%%% vypadající neinteraktivní rejstřík.
%%%
%%% Přepínačem `outlines' se zapíná automatické vytváření záložek s každou 
%%% novou kapitolou. Není-li přepínač uveden, všechny záložky musejí být 
%%% vytvářeny explicitně.
%%%
%%% Přepínač `czech' inicialisuje české typografické konvence.
%%%
%%% Pokud chcete vytvářet pouze dokument ve formátu PostScript, můžete uvést
%%% dodatečný argument `nopdf'. Tím se potlačí chybová hlášení při použití
%%% programu `dvips'.
\usepackage[outlines,czech]{upsimple}

%%% Dodatečné standardní styly.
\usepackage{czech}
\usepackage{epsf}
\usepackage{a4wide}
\usepackage[utf8]{inputenc}

%%% Parametry pro vytvoření úvodních stránek. Jejich použítí či nepoužití je
%%% na libovůli uživatele stylu.
\title{Název dokumentu}
\author{Jan Novák}

%%% Pomocí \docinfo je možné vytvořit název pro PDF dokument, zpravidla je
%%% dobré použít předcházející název, ale bez diakritiky. Možné je však zvolit
%%% úpolně jiný výstižný název. Při tvorbě PostScriptu bude příkaz ignorován.
\docinfo{Jan Novak}{Nazev dokumentu}

%%% Makrem \displayoutlines se provede zobrazení záložek.
%%% Není-li makro uvedeno, záložky jsou v dokumentu skryty.
\displayoutlines

\begin{document}

\maketitle
\tableofcontents{}

%%% Text diplomové práce.
\section{První kapitola}
Toto je první kapitola.

\medskip
\suboutline{Vložená záložka}
Na toto místo směřuje záložka vytvořená pomocí makra \verb|\suboutline|.

\subsection{První podkapitola}
Toto je moje první podkapitola.
Zde jsem vycházel z~prací kolegy Složitého. Lze jej kontaktovat 
na adrese \mail{tomas.slozity@diff.rov.cz}.


\subsection{Druhá podkapitola}
\hyplabel{dotextu}
Toto je moje druhá podkapitola. Sem povede odkaz z~textu.


\section{Druhá kapitola}
Toto je druhá kapitola a budu se zde zabývat tím, jak vypadá
program podle~\cite{smith}.

\subsection{První podkapitola}\label{kapX}
Toto je text mé podkapitoly číslo~\ref{kapX}
\begin{table}[ht]
  \begin{center}
    \renewcommand{\arraystretch}{1.2}
    \begin{tabular}{||l|rr||}
      \hline
      & \multicolumn{2}{|c||}{\bf \hbox{Informace}} \\
      \cline{2-3}
      \bf Čaj & \bf Cena & \bf Množství \\
      \hline
      Chun Mee & 30\,Kč & 100\,g \\
      Lung Ching & 86\,Kč & 50\,g \\
      Show Mee & 147\,Kč & 50\,g \\
      \hline
    \end{tabular}
    \caption{Toto je tabulka.} \label{tab}
  \end{center}
\end{table}

\nextoutline{Jiné jméno}
\subsection{Další podkapitola}
Toto je moje další podkapitola. Tato podkapitola bude dále členěna.
Tato podkapitola se bude v~záložkách jmenovat jinak.

\subsubsection{Podkapitola}\label{podkapX}
Tato podkapitola má číslo~\ref{podkapX}
Zde budu řešit programování podle knihy~\cite{kovar}.
V~této části je i~obrázek, viz~\ref{obr}

\begin{figure}[ht]
  \centerline{\epsfbox{uplogo.eps}}
  \caption{Toto je obrázek.} \label{obr}
\end{figure}

\subsubsection{Podkapitola}\label{podkapY}
Tato část má číslo~\ref{podkapY} a~je umístěna na stránce \pageref{podkapY}.
Na stránce uvedu i~odkaz na URL, třeba \url{http://www.inf.upol.cz}.
Dále lze vytvářet i~odkazy na kapitoly \emphref{v~rámci dokumentu}{kapX}.
Stejně tak lze vytvářet i~odkazy \emphref{přímo do textu}{dotextu},
pro vytvoření návěstí je potřeba použít makro \verb|\hyplabel|.

\medskip
Zde budu řešit programování podle knihy~\cite{kovar}.


%%% Vytvoření seznamu literatury.
\newpage
\begin{thebibliography}{99}

\bibitem{smith} Smith, John. \emph{User and program.}
                Publisher, City, 1990.
\bibitem{kovar} Kovář, Jan. \emph{Jak programovat.}
                Nakladatelství, Město, 1990.
\bibitem{slozi} Složitý, Tomáš. 
                \link{\emph{Překladač s~nakladačem.}}{http://www.inf.upol.cz}
                Elektronická publikace, 2001.

\end{thebibliography}


%%% Přílohy.
\newpage
\appendix

\section{První příloha} \label{PrvniPriloha}
A tady už budou jen závěrečné poznámky k~mému programování.

\end{document}
