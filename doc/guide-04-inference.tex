\subsubsection{Inference}
\label{inference}

Inference (odvozování nových faktů z aktuálních) probíhá v krocích. V každém
kroku jsou vyhodnoceny podmínky všech pravidel, načež je ze splněných pravidel
vybráno jedno, které je posléze aktivováno. Inferenční kroky můžeme buď spouštět
jednotlivě voláním \verb|(step)|, nebo voláním \verb|(run)| spustit
cyklus, který provádí inferenční kroky, dokud je to možné. Cyklus je buď
přerušen ve chvíli, kdy není splněno žádné další pravidlo, nebo voláním
\verb|(halt)| v důsledcích právě aktivovaného pravidla.

Podmínky pravidel jsou ve tvaru vzorů a jsou spojeny logickou konjunkcí,
pravidlo je tedy splněno, jsou-li splněny všechny jeho podmínky. Kromě toho
mohou být některé podmínky negovány. Taková podmínka je splněna tehdy,
neexistuje-li v pracovní paměti žádný fakt kongruentní s jejím vzorem (při
zachování konzistence vazeb proměnných). Negovanou podmínku značí znak \verb|-|
(minus) na prvním místě specifikace vzoru.

Vyhodnocování podmínek pravidel probíhá ve dvou fázích. V první fází srovnáváme
vzory jednotlivých podmínek se všemi fakty v pracovní paměti a~to pouze
strukturálně, tedy bez ohledu na vazby proměnných. Prvním požadavkem shody je u
jednoduchých faktů stejná délka (počet atomů), u strukturovaných faktů stejná
šablona. Jednoduchý fakt se nikdy nemůže shodovat se strukturovaným vzorem a
naopak.

Dále jsou pak porovnávány jednotlivé atomy (u jednoduchých) či
sloty (u~složených) faktu vůči odpovídajícímu atomu (slotu) vzoru. Není-li atom
(slot) vzoru proměnná, je jednoduše porovnán s atomem faktu. Je-li atomem
proměnná, považujeme jej v této fázi automaticky za shodu. Například vzor
\cl|(in :object robot :location ?loc)|
se shoduje s faktem
\cl|(in :object robot :location A)|
nikoli však s fakty
\begin{minted}{cl}
(in :object box :location A)
(is-in :object robot :location A)
(in robot A).
\end{minted}
Tímto předvýběrem
tedy získáme ke každé podmínce pravidla množinu faktů, které mají stejnou
strukturu a stejné hodnoty neproměnných atomů.

Ve druhé fázi vyhodnocování hledáme z předvybraných faktů takovou posloupnost
(délka odpovídá počtu podmínek pravidla), kde po spárování s odpovídajícími
vzory podmínek obdržíme konzistentní vazby proměnných. To znamená, že
vyskytuje-li se v podmínkách pravila některá proměnná vícekrát, musí mít
odpovídající fakty na daných pozicích stejný atom. Mějme například pravidlo
s~podmínkami
\begin{minted}{cl}
(goal move ?obj ?from ?to)
(in :object ?obj  :location ?from)
(in :object robot :location ?to).
\end{minted}
Vzor první podmínky je jednoduchý, zatímco další dva jsou strukturované. To ale
ničemu nevadí, je třeba pouze najít fakty odpovídající struktury. Posloupnost
faktů
\begin{minted}{cl}
(goal move box A B)
(in :object box   :location B)
(in :object robot :location A)
\end{minted}
neprojde druhou fází výběru, neboť vazby proměnných nejsou konzistentní.
Proměnná \verb|?from| je například v první podmínce navázána na symbol \verb|A|, v
druhé ale na \verb|B|. Kdyby si ovšem krabice s robotem vyměnili pozice, budou
vazby proměnných konzistentní a podmínky pravidla budou splněny. Proměnná
\verb|?from| by pak měla vazbu \verb|A|, proměnná \verb|?to| vazbu
\verb|B| a proměnná \verb|?obj| vazbu \verb|box|.

Vyhodnocení negovaných podmínek si můžeme představit tak, že nejprve vyhodnotíme
a navážeme proměnné všech ostatních podmínek. Pokud poté neexistuje fakt
kongruentní se vzorem negované podmínky s konzistentními vazbami se zbytkem
navázaných proměnných, je tato podmínka splněna. Mějme například pravidlo
s~podmínkami
\begin{minted}{cl}
(goal move box ?from ?to)
(in box ?from)
(- in robot ?from).
\end{minted}
Máme-li v pracovní paměti pouze fakty
\begin{minted}{cl}
(goal move box A B)
(in box A)
(in robot B),
\end{minted}
budou podmínky pravidla splněny, neboť po spárování vzorů prvních dvou podmínek
s prvními dvěma fakty bude proměnná \verb|?from| navázána na hodnotu
\verb|A|~a~neexistuje fakt kogruentní se vzorem \verb|(in robot A)|.
Přesuneme-li ale robota na pozici \verb|A|, podmínka již splněna nebude a
pravidlo nelze aktivovat. U negovaných podmínek je třeba dát pozor na to, aby se
všechny jejich proměnné vyskytovaly už dříve v nějaké pozitivní podmínce. V
opačném případě nebude vyhodnocení vazeb proměnných fungovat správně (viz
kapitola \ref{rete}).

Ve vzorech podmínek pravidla můžeme využít speciální proměnné \verb|?|.
Konzistence vazby této proměnné není při vyhodnocování testována, takže
vyskytuje-li se tato proměnná na více místech, chová se tak, jako kdyby byl
každý výskyt označen unikátním názvem (podobně jako proměnná \verb|_| v
Prologu). Použitím této proměnné dáváme najevo, že nás konkrétní hodnota daného
atomu nazajímá. Ve strukturovaných vzorech podmínek není třeba tyto sloty uvádět,
neboť \verb|?| je výchozí hodnotou slotu vzoru.

Posledním speciálním konstruktem je navázání celého faktu na proměnnou.
Například pravidlo
\begin{minted}{cl}
(defrule move
  ?fact <- (in :object ? :location A)
  =>
  (modify ?fact :location B))
\end{minted}
přesune každý objekt z pozice \verb|A| na pozici \verb|B|. Na proměnnou
můžeme navázat i~jednoduchý fakt, pak ale nemůžeme použít makra \verb|modify|.
Můžeme ovšem volat \verb|(retract ?fact)|, neboť proměnná \verb|?fact| je při
aktivaci pravidla nahrazena specifikací faktu, který byl se vzorem podmínky
spárován.

Podmínky některých pravidel mohou být při vyhodnocování splněny několika různými
posloupnostmi faktů. Výsledkem vyhodnocování pravidel tedy není pouze množina
splněných pravidel, nýbrž množina shod (\emph{match}). Každá shoda je tvořena
pravidlem a substitucí proměnných navázaných při vyhodnocování jeho podmínek.
Substituci chápejme jako zobrazení z množiny všech proměnných, vyskytujících se
v podmínkách pravidla, na konkrétní hodnoty atomů (slotů). Aplikací této
substituce na vzory podmínek pravidla získáme opět posloupnost faktů, kterými
byly podmínky v dané shodě splněny. Tato substituce je následně použita při
aktivaci pravidla k náhradě proměnných v jeho důsledcích.

ExiL uchovává aktuální množinu shod. Ta ve skutečnosti není přepočítána v~první
fázi inferenčního kroku, jak jsem dosud pro jednoduchost tvrdil, nýbrž
automaticky po každé změně pracovní paměti či množiny pravidel (detaily viz
kapitola \ref{rete}). Aktuální množinu shod nazývám, po vzoru CLIPSu,
\emph{agenda} a~lze ji vypsat voláním stejnojmenné funkce. Každá shoda v agendě
je opatřena časovým razítkem, shody je tedy možné uspořádat podle toho, kdy do
agendy přibyly.

Je-li na začátku inferenčního kroku v agendě více shod, je třeba z nich jednu
vybrat k aktivaci. Výběr shody záleží na zvolené strategii. ExiL poskytuje
následující strategie výběru shody:
\begin{description}[leftmargin=5cm,style=sameline,align=right,labelsep=0.5cm]
  \item[depth-strategy] vybírá shodu, která do agendy přibyla nejpozději
  \item[breadth-strategy] vybírá shodu, která do agendy přibyla jako první
  \item[simplicity-strategy] vybere shodu, jejíž pravidlo má nejméně podmínek
  \item[complexity-strategy] volí shodu, jejíž pravidlo má nejvíce podmínek.
\end{description}
Názvy prvních dvou strategií vychází z toho, že jde o prohledávání stromu
stavů systému do hloubky, či do šířky (viz kapitola \ref{theory inference}).

Výchozí strategií je \verb|depth-strategy|. Strategii, která bude v inferenci
použita, můžeme zvolit voláním makra \verb|setstrategy| s názvem strategie,
např. \verb|(setstrategy breadth-strategy)|. Seznam názvů strategií můžeme
vypsat voláním \verb|(strategies)|, název aktuálně zvolené strategie pak voláním
\verb|(current-strategy)|.

Po výběru shody k aktivaci je její pravidlo aktivováno. Nejprve jsou v
důsledcích pravidla nahrazeny všechny proměnné pomocí substituce shody. Výrazy
v~důsledcích jsou poté vyhodnoceny lispovým makrem \verb|eval|. Tento způsob
vyhodnocení vede ke značným omezením. Výrazy totiž nejsou vyhodnoceny v
\emph{lexikálním prostředí} definice pravidla, nýbrž ve výchozím
(\emph{top-level}) prostředí Lispu. Díky tomu nemůžeme v důsledcích pravidla
volat \emph{lokální funkce} či přistupovat k~\emph{lokálním proměnným}.

V důsledcích každého pravidla chceme typicky alespoň jednou volat některé
z~maker modifikujících pracovní paměť, abychom zneplatnili podmínky pravidla,
jinak se inference zacyklí. Druhou možností je přerušit inferenci voláním
\verb|(halt)|. Volání \verb|(step)| nebo \verb|(run)| ve chvíli, kdy již nelze
dále odvozovat (žádné z pravidel nemá splněné podmínky), nemá žádný efekt.
