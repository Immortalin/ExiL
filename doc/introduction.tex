%%%%%%%%%%%%%%%%%%%%%%%%%%%%%%%%%%%%%%%%%%%%%%%%%%%%%%%%%%%%%%%%%%%%%%%%%%%%%%%%
\section{Úvod}

Pojem expertního systému spadá do oblasti umělé inteligence. Jde o počítačový
systém, který simuluje rozhodování experta nad zvolenou problémovou doménou.
Expertní systém může experta zcela nahradit, nebo mu při rozhodování asistovat.

Ve své bakalářské práci \cite{bakalarka} jsem implementoval základní knihovnu
pro tvorbu expertních systémů (tzv. prázdný expertní systém) s~dopředným
řetězením v~jazyce Common Lisp - ExiL\footnote{EXpert system In Lisp, původcem
  názvu knihovny je Zdenek Eichler}. Cílem této práce je knihovnu rozšířit o
\begin{itemize}
  \item syntaktický režim pro zajištění přiměřené kompatibility se systémem
    CLIPS,
  \item možnost vrácení provedených změn včetně odvozovacích kroků,
  \item základní zpětné řetězení,
  \item podporu pro ladění s jednoduchým grafickým uživatelským rozhraním pro
    prostředí LispWorks\texttrademark,
  \item podporu kompozitních podmínek (vnořené AND, OR a NOT) a jejich všeobecné
    kvantifikace.
\end{itemize}

Jazyk Common Lisp\footnote{\url{http://en.wikipedia.org/wiki/Common\_Lisp}}
(případně jiné dialekty Lispu) je častou volbou pro implementaci umělé
inteligence díky svým schopnostem v oblasti symbolických výpočtů (manipulace
symbolických výrazů), na nichž řešení těchto problémů často staví. Navíc jde
o~vysokoúrovňový, dynamicky typovaný jazyk, díky čemuž je programový kód
stručný, snadno pochopitelný a tudíž jednoduše rozšiřitelný.

Syntax systému CLIPS\footnote{\url{http://clipsrules.sourceforge.net}} byla
zvolena proto, že jde o reálně používaný
systém\footnote{\url{http://clipsrules.sourceforge.net/FAQ.html\#Q6}}, jehož
syntax je Lispu velmi blízká, takže není těžké ji napodobit.

Expertní systém se syntaxí podobnou systému CLIPS implementuje v Common Lispu
také projekt Lisa\footnote{\url{http://lisa.sourceforge.net/}}. Ten však
poskytuje pouze makra pro přímou práci se systémem nikoli funkce, které by bylo
možné volat z jiného kódu. Tento nedostatek řeším pomocí funkčních alternativ
maker popsaných v kapitole \ref{external calls} Projekt také neposkytuje možnost
zpětného řetězení, vracení provedených změn ani grafické uživatelské rozhraní.

V teoretické části textu popíši obecné principy expertních systémů, typické
příklady jejich použití a způsoby, jakými expertní systém reprezentuje znalosti
a~vyvozuje z nich závěry. Stručně také představím syntaxi systému CLIPS a jeho
možnosti.

V praktické části poté popíši všechny možnosti knihovny ExiL, její instalaci a
použití, architekturu programu, algoritmus RETE, na kterém implementace staví, a
rozeberu některé aspekty implementace rozšíření.
