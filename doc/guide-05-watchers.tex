%%%%%%%%%%%%%%%%%%%%%%%%%%%%%%%%%%%%%%%%%%%%%%%%%%%%%%%%%%%%%%%%%%%%%%%%%%%%%%%%
\subsubsection{Sledování průběhu inference}
\label{inference tracing}
ExiL umožnuje sledovat několik typů událostí, ke kterým dochází během inference.
K nastavení sledovaných událostí slouží makra \verb|watch| a \verb|unwatch|. K
zjištění stavu sledování pak makro \verb|watchedp|.

Základním výstupem programu \ref{typical structure} na straně \pageref{typical
structure} je
\begin{minted}{cl}
Firing MOVE
Firing PUSH
Firing STOP
Halting.
\end{minted}
Zapneme-li sledování faktů voláním \verb|(watch facts)|, obdržíme výstup
\begin{minted}{cl}
==> (IN BOX A)
==> (IN ROBOT B)
==> (GOAL MOVE BOX A B)
Firing MOVE-ROBOT
<== (IN ROBOT B)
==> (IN ROBOT A)
Firing MOVE-OBJECT
<== (IN ROBOT A)
<== (IN BOX A)
==> (IN ROBOT B)
==> (IN BOX B)
Firing STOP
Halting.
\end{minted}

Sledování pravidel (\verb|(watch rules)|), přidává informace o pravidlech
přidaných do (odebraných ze) znalostní báze, např.
\begin{minted}{cl}
==> (RULE STOP
  (GOAL MOVE ?OBJECT ?FROM ?TO)
  (IN ?OBJECT ?TO)
  =>
  (HALT)).
\end{minted}

Po zapnutí sledování agendy (voláním \verb|(watch activations)|, název je kvůli
kompatibilitě se systémem CLIPS) budeme navíc informováni o shodách, které do
agendy přibyly, nebo z ní byly odstraněny. Výstup programu pak bude následující:
\begin{minted}[samepage]{cl}
==> (MATCH MOVE-ROBOT
      ((GOAL MOVE BOX A B) (IN BOX A) (IN ROBOT B)))
Firing MOVE-ROBOT
==> (MATCH MOVE-OBJECT
      ((GOAL MOVE BOX A B) (IN BOX A) (IN ROBOT A)))
==> (MATCH MOVE-ROBOT
      ((GOAL MOVE BOX A B) (IN BOX A) (IN ROBOT A)))
<== (MATCH MOVE-ROBOT
      ((GOAL MOVE BOX A B) (IN BOX A) (IN ROBOT A)))
Firing MOVE-OBJECT
==> (MATCH STOP ((GOAL MOVE BOX A B) (IN BOX B)))
Firing STOP
Halting.
\end{minted}
Každá shoda je zde reprezentována názvem pravidla a posloupností faktů, které
byly spárovány s jeho podmínkami. Odtud můžeme snadno odvodit substituci, jež
byla při vyhodnocení použita. Negované podmínky nejsou spárovány s žádným
konkréním faktem, proto jsou zde jen tři fakty, přestože pravidlo
\verb|move-robot| má podmínky čtyři.

Je zde také vidět, že po aktivaci pravidla
\verb|move-robot| se v agendě na chvíli objeví opětovná shoda tohoto pravidla.
To je způsobeno tím, že obsah agendy se přepočítává po každé změně pracovní
paměti, takže se zde mohou objevit dočasné výsledky.
