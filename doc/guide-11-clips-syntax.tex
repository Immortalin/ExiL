%%%%%%%%%%%%%%%%%%%%%%%%%%%%%%%%%%%%%%%%%%%%%%%%%%%%%%%%%%%%%%%%%%%%%%%%%%%%%%%%
\subsubsection{Kompatibilita se systémem CLIPS}
\label{guide clips}

\begin{listing}[t]
\caption{Definice znalostní báze s použitím CLIPSové syntaxe}
\label{clips syntax}
\begin{clcode}
(deftemplate goal
  (slot action (default move))
  (slot object)
  (slot from)
  (slot to))

(deftemplate in
  (slot object)
  (slot location))

(deffacts world
  (in (object robot) (location A))
  (in (object box) (location B))
  (goal (object box) (from B) (to A))).

(defrule move-robot
  (goal (action move) (object ?obj) (from ?from))
  (in (object ?obj) (location ?from))
  (- in (object robot) (location ?from))
  ?robot <- (in (object robot) (location ?z))
  =>
  (modify ?robot (location ?from)))

(defrule move-object
  (goal (action move) (object ?obj) (from ?from) (to ?to))
  ?object <- (in (object ?obj) (location ?from))
  ?robot <- (in (object robot) (location ?from))
  =>
  (modify ?robot (location ?to))
  (modify ?object (location ?to)))

(defrule stop
  ?goal <- (goal (action move) (object ?obj) (to ?to))
  (in (object ?obj) (location ?to))
  =>
  (halt))
\end{clcode}
\end{listing}

Dalším z požadavků zadání práce bylo přiblížit syntaxi exilových volání systému
CLIPS, aby bylo možné programy v něm napsané snáze převést na programy exilové.
Toho se mi podařilo dosáhnout jen částečně.

\FloatBarrier

Systém CLIPS použivá jiný formát specifikací slotů šablony, strukturovaných
faktů a požadovaných změn při volání \verb|modify|. Tuto syntaxi nyní ExiL podporuje
také. Příklad \ref{clips syntax} na straně \pageref{clips syntax} ukazuje
definici znalostní báze ekvivalentní příkladu \ref{structured facts} s použitím
CLIPSové syntaxe. Syntax je dostatečně odlišná na to, aby ji ExiL rozpoznal,
není tedy třeba syntaktický mód nijak přepínat. Díky tomu dokonce můžeme oba
typy syntaxe kombinovat.

ExiL také po vzoru CLIPSu umožňuje omezit seznam vrácený voláním \verb|(facts)|
volitelnými číselnými parametry. První volitelný parametr udává index prvního
faktu v seznamu (číslováno od 1). Druhý parametr udává index posledního faktu.
Třetí parametr pak maximální počet vrácených faktů.

Makra \verb|assert| a \verb|retract| také nyní umožňují přidání či odebrání
více faktů najednou. Makru \verb|retract| lze navíc místo specifikací faktů k
odstranění předat jejich číselné indexy v seznamu faktů. Obě možnosti lze
dokonce kombinovat.

V podmínkách pravidel lze nyní také použít speciální proměnnou \verb|?| a
navázání celého faktu na proměnnou pomocí operátoru \verb|<-|, jak jsem již
popsal v kapitole \ref{inference}

CLIPS dále umožňuje specifikovat u slotu šablony datový typ. To považuji v
dynamickém jazyce, jakým je Common Lisp, za zbytečné. CLIPS také nabízí několik
vlastních funkcí a možnost definice nových, které pak lze používat v~důsledcích
pravidel. To nemá v ExiLu smysl, neboť můžeme použít vestavěné nebo uživatelsky
definované funkce Lispu.

CLIPS navíc poskytuje možnost definovat některé sloty šablony jako
\emph{multislot}. Ve vzorech podmínek pravidel pak můžeme používat proměnnou
\verb|$?|, která se naváže na jeden nebo více atomů multislotu, případně
jednoduchého faktu. Dále je v podmínkách CLIPSu možné používat libovolně vnořené
agregační funkce \verb|and|, \verb|or|, \verb|not| a podobně. Ve vzorech
podmínek CLIPSových pravidel lze také používat speciální proměnné, které se
shodují jen s některými atomy, např. proměnná \verb|?color&~red&~yellow| se
shoduje se všemi atomy, kromě \verb|red| a \verb|yellow|.

Tyto možnosti ExiL neposkytuje, neboť jejich implementace by vyžadovala přepsat
velkou část algoritmu RETE, který podmínky pravidel vyhodnocuje, jak popíšu v
kapitole \ref{composite conditions}.
