\subsection{Potřebná znalost Common Lispu}

Knihovna ExiL je implementována v jazyce Common Lisp. Pro práci s ní a~pochopení
její implementace je tedy potřebná znalost tohoto jazyka. Za perfektní příručku
k jazku Common Lisp považuji knihu Practical Common Lisp\footnote{Siebel, P.:
\textit{Practical Common Lisp}. Apress, 2005, ISBN 1-59059-239-5.}, která je k
dispozici
online\footnote{\url{http://www.gigamonkeys.com/book/}}.
Následující seznam uvádí relevantní pojmy z Common Lispu a kapitoly z knihy, kde
se o nich lze dočíst. V dalším textu budu vždy první výskyt nového pojmu značit
kurzívou.

\begin{description}[style=nextline]
  \item[2. Lather, Rinse, Repeat: A Tour of the REPL] implementace lispu,
    Lispbox, práce s REPLem, obsluha výjimek
  \item[4. Syntax and Semantics] prefixová syntax, S-výrazy, funkce,
    speciální operátory, makra
  \item[5. Functions] funkce - definice, typy parametrů, anonymní funkce
  \item[6. Variables] proměnné - vytváření vazeb, lexikální proměnné a
    uzávěry, dynamické proměnné, přiřazení, makra upravující hodnoty
    proměnných
  \item[7. Macros: Standard Control Constructs] řídicí struktury, cykly
  \item[8. Macros: Defining Your Own] makra - jejich funkce, vyhodnocování,
    definice, parametry
  \item[10. Numbers, Characters, and Strings] základní vestavěné typy a
    operace s nimi
  \item[11. Collections] kolekce (pole, hashovací tabulky) a práce s nimi
  \item[12. They Called It LISP for a Reason: List Processing] seznamy -
    reprezentace, operace
  \item[13. Beyond Lists: Other Uses for Cons Cells] stromy, množiny,
    asociativní a property listy, destrukturující makra
  \item[16. Object Reorientation: Generic Functions] generické funkce,
    metody, multimetody, řetězec volání
  \item[17. Object Reorientation: Classes] třídy a objekty
    CLOSu\footnote{\url{http://en.wikipedia.org/wiki/Common\_Lisp\_Object\_System}}
    - definice, sloty, dědičnost
  \item[19. Beyond Exception Handling: Conditions and Restarts] práce s
    výjimkami
  \item[20. The Special Operators] kontrola vyhodnocování S-výrazů, kvotování;
    načítání zdrojových souborů pomocí \verb|load|
  \item[21. Programming in the Large: Packages and Symbols] package a práce se
    symboly - klíčky, import, export, zastiňování
\end{description}

Jako referenční příručku vestavěných funkcí a maker Common Lispu pak lze použít
Common Lisp
HyperSpec\texttrademark\footnote{\url{http://www.lispworks.com/documentation/HyperSpec/Front/}}.
