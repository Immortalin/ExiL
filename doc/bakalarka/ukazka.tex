Na závěr této sekce uvedu krátký příklad práce s~knihovnou. Řádky začínající
``\verb|EXIL>|'' označují vstup uživatele, zbytek je výstup programu.
\begin{verbatim}
EXIL> (deftemplate goal (action object from to))
#<TEMPLATE GOAL ((ACTION :DEFAULT NIL) (OBJECT :DEFAULT NIL)
                 (FROM :DEFAULT NIL) (TO :DEFAULT NIL))>

EXIL> (deftemplate in (object location))
#<TEMPLATE IN ((OBJECT :DEFAULT NIL) (LOCATION :DEFAULT NIL))>
\end{verbatim}
\begin{verbatim}
EXIL> (deffacts world
        (in :object robot :location A)
        (in :object box :location B)
        (goal :action push :object box :from B :to A))
T
\end{verbatim}
\begin{verbatim}
EXIL> (defrule stop
        (goal :object ?x :to ?y)
        (in :object ?x :location ?y)
      =>
        (halt))
#<RULE STOP>
\end{verbatim}
\begin{verbatim}
EXIL> (defrule move
        (goal :object ?x :from ?y)
        (in :object ?x :location ?y)
        (- in :object robot :location ?y)
        (in :object robot :location ?z)
      =>
        (modify (in :object robot :location ?z)
                (in :object robot :location ?y)))
#<RULE MOVE>
\end{verbatim}
\begin{verbatim}
EXIL> (defrule push
        (goal :object ?x :from ?y :to ?z)
        (in :object ?x :location ?y)
        (in :object robot :location ?y)
      =>
        (modify (in :object robot :location ?y)
                (in :object robot :location ?z))
        (modify (in :object ?x :location ?y)
                (in :object ?x :location ?z)))
#<RULE PUSH>
\end{verbatim}
\begin{verbatim}
EXIL> (watch facts)
T

EXIL> (watch activations)
T
\end{verbatim}
\begin{verbatim}
EXIL> (reset)
==> (IN (OBJECT . ROBOT) (LOCATION . A))
==> (IN (OBJECT . BOX) (LOCATION . B))
==> (GOAL (ACTION . PUSH) (OBJECT . BOX) (FROM . B) (TO . A))
==> Activation MOVE:
((GOAL (ACTION . PUSH) (OBJECT . BOX) (FROM . B) (TO . A))
 (IN (OBJECT . BOX) (LOCATION . B))
 (IN (OBJECT . ROBOT) (LOCATION . A)))
NIL
\end{verbatim}
\begin{verbatim}
EXIL> (run)
Firing Activation MOVE:
((GOAL (ACTION . PUSH) (OBJECT . BOX) (FROM . B) (TO . A))
 (IN (OBJECT . BOX) (LOCATION . B))
 (IN (OBJECT . ROBOT) (LOCATION . A)))
<== (IN (OBJECT . ROBOT) (LOCATION . A))
==> (IN (OBJECT . ROBOT) (LOCATION . B))
==> Activation PUSH:
((GOAL (ACTION . PUSH) (OBJECT . BOX) (FROM . B) (TO . A))
 (IN (OBJECT . BOX) (LOCATION . B))
 (IN (OBJECT . ROBOT) (LOCATION . B)))
\end{verbatim}
\begin{verbatim}
Firing Activation PUSH:
((GOAL (ACTION . PUSH) (OBJECT . BOX) (FROM . B) (TO . A))
 (IN (OBJECT . BOX) (LOCATION . B))
 (IN (OBJECT . ROBOT) (LOCATION . B)))
<== (IN (OBJECT . ROBOT) (LOCATION . B))
==> (IN (OBJECT . ROBOT) (LOCATION . A))
<== (IN (OBJECT . BOX) (LOCATION . B))
==> (IN (OBJECT . BOX) (LOCATION . A))
==> Activation STOP:
((GOAL (ACTION . PUSH) (OBJECT . BOX) (FROM . B) (TO . A))
 (IN (OBJECT . BOX) (LOCATION . A)))
\end{verbatim}
\begin{verbatim}
Firing Activation STOP:
((GOAL (ACTION . PUSH) (OBJECT . BOX) (FROM . B) (TO . A))
 (IN (OBJECT . BOX) (LOCATION . A)))
Halting
NIL
\end{verbatim}
Příklad je převzat z~\cite{introduction} a je původně určen pro systém \textsf{CLIPS}.
