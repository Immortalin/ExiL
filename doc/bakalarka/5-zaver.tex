%%%%%%%%%%%%%%%%%%%%%%%%%%%%%%%%%%%%%%%%%%%%%%%%%%%%%%%%%%%%%%%%%%%%%%%%%%%%%%%%
\section{Závěr}
Má implementace prázdného expertního systému podává (při použití stejné
inferenční strategie) stejné výsledky jako systém \textsf{CLIPS}, jímž byla
inspirována. Neposkytuje ale všechny jeho možonsti a nedosahuje, pochopitelně,
zdaleka takovéhu výpočetního výkonu. V~dalším studiu bych se rád zabýval právě
zvyšováním efektivity výpočtů a tedy snižováním reakční doby systému. Věnovat
bych se chtěl také rozšíření systému o~další inferenční strategie (např.
\textsf{LEX} a \textsf{MEA}), které podobné systémy často poskytují,
a rozšiřováním syntaxe pravidel. Systém \textsf{CLIPS} např. poskytuje
možnost \uv{uložení} fakta, jež vyhovuje dané podmínce pravidla, do proměnné
a následné práce s~ním. Při rozhodování o~použité syntaxi, se kterou se
knihovní funkce volají, jsem se snažil co nejvíce přiblížit lispovým idiomům
(volání makra \verb|deftemplate| je např. velice podobné volání \verb|defclass|
standardního objektového systému \textsf{CLOS}),
rád bych v~dalším studiu poskytl rozšíření syntaxe tak, aby bylo možno
programy vytvrořené pro systém \textsf{CLIPS} přímo použít v~mé implementaci.
