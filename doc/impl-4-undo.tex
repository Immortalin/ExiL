\subsubsection{Undo/redo}

Možnost vrácení provedených akcí a jejich opětovného provedení (undo/redo) je
implementována na úrovni prostředí, které udržuje aktuální stav systému.
Hodnoty, které stav tvoří, jsem popsal v sekci \ref{env cleanup} a možnosti,
které undo/redo poskytuje, v sekci \ref{undo}

Prostředí je v programu reprezentováno objektem \verb|environment|. Stav
systému je uchováván ve slotech tohoto objektu.

Prostředí dále udržuje dva zásobníky - \verb|undo| a \verb|redo|.
Každý z~těchto zásobníků ukládá seznamy tvořené popiskem akce k vrácení,
uzávěrem, který akci vrátí, a~uzávěrem, který ji opět provede. První uzávěr ve
svém lexikálním prostředí uchovává hodnoty potřebné k vrácení akce. Zásobníky
jsou manipulovány pouze několika makry a metodami k tomu určenými, zbytek kódu k
zásobníkům přímo nepřistupuje.

Tělo každé akce, která mění hodnoty prostředí, je obaleno buď voláním makra
\verb|with-undo|, nebo \verb|with-saved-slots|. Volání \verb|with-undo| je kromě
těla akce předán uzávěr, který zajistí její vrácení. Makro \verb|with-undo|
po vyhodnocení těla akce zajistí, že se na zásobník \verb|undo| uloží předaný
uzávěr spolu s uzávěrem, který akci opět provede při volání \verb|redo|.
Tento druhý uzávěr pouze vyhodnotí původní tělo akce.

Makro \verb|with-saved-slot| zjednodušuje volání \verb|with-undo|. Toto makro
bere kromě těla akce seznam slotů prostředí, které je třeba před akcí uložit.
Makro pak automaticky vytvoří uzávěr, který předá makru \verb|with-undo|. Tento
uzávěr si ve svém lexikálním prostředí pamatuje kopie původních hodnot požadovaných
slotů prostředí a při vyhodnocení je nastaví zpět na tyto hodnoty.

Prostředí ke každému svému slotu definuje metodu \verb|copy-<slot>|, např.
\verb|copy-facts|. Makro \verb|with-saved-slots| tedy vytvoří potřebný uzávěr
tak, že pomocí volání makra \verb|let| naváže výsledky volání těchto metod pro
každý požadovaný slot. V těle tohoto volání pak vytvoří anonymní (lambda)
funkci, která hodnoty prostředí nastaví zpět na ty, které \verb|let| navázal.

Funkce \verb|undo| pak jednoduše odstraní seznam z vrcholu zásobníku
\verb|undo|, zavolá uzávěr, který vrátí poslední akci, a druhý uzávěr, který akci
opět provede, uloží na vrchol zásobníku \verb|redo|. Funkce \verb|redo| funguje
symetricky.

Všechny objekty definované modulem \verb|core| - šablony, fakty, vzory a
pravidla jsou implementovány jako neměnné (\emph{immutable}). Každá jejich
případná změna tedy znamená vytvoření nového objektu. Díky tomu není na úrovni
prostředí nutné tyto objekty kopírovat. Kopírovat je třeba pouze datové
struktury, které udržují kolekce těchto objektů.

Nakonec je třeba zajistit, aby každé volání \verb|with-undo| uložilo na zásobník
právě jeden uzávěr k vrácení akce. Například pokud samostatně voláme makro
\verb|assert| pro přidání nějakého faktu do pracovní paměti, chceme, aby toto volání
bylo možné vrátit. Pokud ale voláme \verb|step|, jehož výsledkem je aktivace
nějakého pravidla, které ve svých důsledcích volá makro \verb|assert|, chceme,
aby se volání \verb|step| uložilo na zásobník jako jedna akce. Volání
\verb|assert|, ke kterému v průběhu této akce dojde, už samostatně ukládat
nechceme.

K tomuto účelu definuje prostředí \emph{dynamickou proměnnou}
\verb|*undo-enabled*|.  Makro \verb|with-undo| ukládá uzávěr na zásobník jen v
případě, že je hodnota této proměnná \verb|true|. Tělo akce pak makro
vyhodnocuje s hodnotou této proměnné navázanou na \verb|false|. Tím je
zajištěno, že se uzávěr na zásobník \verb|undo| uloží vždy jen v
\uv{nejvnějšnějším} volání makra \verb|with-undo|.

Díky makrům \verb|with-undo| a \verb|with-saved-slots| je možné velmi snadno
přidat do prostředí další akce s možností jejich vrácení. Stačí jen tělo akce
obalit jedním z těchto maker bez nutnosti vědět, jak je vrácení akce
implementováno. Pokud potřebujeme do prostředí přidat další slot, jehož hodnotu
je třeba při volání \verb|undo| obnovovat, stačí k tomuto slotu implementovat
metodu \verb|copy|.

Nejsložitějším problémem při implementaci undo/redo bylo implementovat metodu
\verb|copy-rete|. Ta kopíruje dataflow síť RETE uloženou ve slotu \verb|rete|
prostředí. Kopírovnání sítě RETE je složité, neboť síť obsahuje cykly. Síť se
sice z~pohledu aktivace uzlů chová jako acyklický graf, některé uzly však
uchovávájí reference na své sousedy proti směru hran tohoto grafu.

Pro účely kopírování sítě RETE jsem implementoval obecnou funkci vyššího
řádu\footnote{\url{http://en.wikipedia.org/wiki/Higher-order\_function}}
pro průchod cyklickým grafem. Tato funkce využívá techniky
memoizace\footnote{\url{http://en.wikipedia.org/wiki/Memoization}}. Funkce bere
kromě výchozího uzlu grafu tři funkce, které aplikuje na navštěvované uzly před
memoizací a jimiž agreguje hodnoty vrácené sousedy uzlu.

Fungování funkce je pro textový popis příliš složité, považuji ji však za jednu
z nejzajímavějších částí nového kódu, hlavně díky její obecnosti. Zdrojový kód
funkce je v souboru \verb|rete/graph-traversal.lisp|, kód kopírující síť rete,
který funkci využívá pak v souboru \verb|rete/rete-copy.lisp|. Celkově je kód
zajišťující kopírování sítě RETE asi třikrát delší, než zbytek kódu
implementující funkcionalitu undo/redo. Ten se nachází v souboru
\verb|environment/env-undo.lisp|.
