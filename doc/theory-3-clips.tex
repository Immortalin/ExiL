\subsection{Systém CLIPS}

Prázdný expertní systém CLIPS poskytuje obecnější a tudíž flexibilnější
reprezentaci faktů a odvozovacích pravidel než systém STRIPS. Atomy faktů CLIPSu
mohou být nejen symboly, ale také řetězce znaků a čísla (celá i desetinná).
Systém rozlišuje dva typy faktů - jednoduché a strukturované. Jednoduchý fakt je
reprezentován seznamem (uspořádanou \emph{n}-ticí) atomů, např.
\verb|(in box A)|. Strukturovaný fakt je reprezentován objektem s pojmenovanými
atributy (sloty), např. \verb|(in (object box) (location A))|. Hodnotami slotů jsou
opět atomy.

Vzory mají v CLIPSu stejný tvar jako fakty. Symboly proměnných zde začínají
znakem \verb|?|, např. \verb|?location|. Kongruenci faktů a vzorů lze definovat
podobně jako ve STRIPSu. U jednoduchých faktů srovnáváme atomy na odpovídajících
pozicích, u strukturovaných pak hodnoty odpovídajících slotů objektu.

Systém CLIPS umožňuje spojovat podmínky odvozovacích pravidel logickými
konjunkcemi i disjunkcemi a tyto libovolně vnořovat, navíc můžeme některé dílčí
podmínky negovat. Dále je možné podmínky pravidel kvantifikovat jak existenčně,
tak všeobecně. Podmínky CLIPSového odvozovacího pravidla tedy mohou vypadat
například takto \cite{clips}:
\begin{minted}{cl}
(or (and (temp high)
         (valve closed))
    (and (temp low)
         (valve open))).
\end{minted}

Definice pravidla CLIPSu vypadá například takto \cite{clips}:
\begin{minted}{cl}
(defrule system-flow
  (error-status ?status)
  (or (and (temp high)
           (valve closed))
      (and (temp low)
           (valve open)))
  =>
  (retract (error-status ?status))
  (assert (error-status confirmed))
  (printout t "The system is having a flow problem." crlf)).
\end{minted}
Definice obsahuje název pravidla, jeho podmínky a důsledky. Podmínky jsou od
důsledků odděleny symbolem \verb|=>|. Důsledky pravidla nejsou pouze vzory faktů
k~přidání do či odebrání ze znalostní báze, nýbrž libovolné výrazy jazyka, který
CLIPS definuje. Ty jsou při aktivaci pravidla, po nahrazení proměnných jejich
vazbami, vyhodnoceny CLIPSovým interpreterem.

Systém CLIPS poskytuje velmi široké možnosti, další už nebudu uvádět. Lze je
však najít v dokumentaci
systému\footnote{\url{http://clipsrules.sourceforge.net/OnlineDocs.html}}.
Praktická část této práce popisuje implementaci knihovny ExiL, která je
výsledkem práce a implementuje prázdný expertní systém inspirovaný právě
systémem CLIPS.  Ve srovnání s ním však poskytuje jen velmi omezené možnosti.
Knihovna je vytvořena v programovacím jazyce Common Lisp a v důsledcích pravidla
lze použít libovolné lispové výrazy, včetně volání maker pro modifikaci stavu
systému.
