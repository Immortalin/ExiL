\subsubsection{Kompozitní podmínky pravidel}
\label{composite conditions}

Podporu kompozitních podmínek odvozovacích pravidel (vnořené aplikace logických
funkcí v podmínkách) a všeobecné kvantifikace podmínek jsem v ExiLu
neimplementoval, neboť ta by vyžadovala komplexní změny v implementaci algoritmu
RETE, která je už tak poměrně složitá. Síť navíc při definici nových pravidel
velice rychle roste a obsahuje cykly (nikoli z pohledu aktivace uzlů, ale z
pohledu referencí, které na sebe uzly udržují), díky čemuž se velice těžko ladí.
Alespoň tedy nastíním, jak lze podporu těchto podmínek implementovat.

\cite{doorenbos} uvádí, jak v síti RETE implementovat tzv. konjunktivní negace,
například podmínku ve tvaru
\begin{minted}{cl}
(and (?x on ?y)
     (?y left-of ?z)
     (not (and (?z color red)
               (?z on ?w)))).
\end{minted}
Příklad pracuje s barevnými bloky, které lze stavět na sebe. Tato podmínka je
splněna ve chvíli, kdy existuje blok \verb|?x| na bloku \verb|?y|, nalevo od
bloku \verb|?z|, který není zároveň červený a umístěn na nějakém dalším bloku.
Negace a konjunkce mohou být v~podmínkách libovolně vnořovány.

Pro skupinu podmínek v negaci vytvoříme podsíť uzlů, stejně jako v původním
algoritmu. Místo použití jednoho \verb|beta-negative-node| je pak pro
vyhodnocení složené negace použita dvojice speciálních uzlů. První z nich se
stane potomkem uzlu \verb|beta-memory-node|, který ukládá výsledky předchozích
podmínek. Druhý je pak potomkem prvního, zároveň však i potomkem posledního uzlu
v podsíti.

Druhý uzel funguje podobně, jako běžný \verb|beta-negative-node|
s tím, že ale testuje konzistenci vazeb mezi tokeny z předchozích podmínek a
tokeny přicházející z~podsítě. Pokud k danému tokenu shora neexistuje
konzistentní token z~podsítě, je konjunktivní negace na této úrovni splněna a
druhý uzel aktivuje potomky.

Takto můžeme postupovat rekurzivně až na úroveň, kdy jsou negovány pouze
jednotlivé podmínky. Pro ty je pak použit běžný \verb|beta-negative-node|. Díky
konjunkci a negaci, které lze libovolně vnořovat, můžeme implementovat také
disjunkci (neboť
$\varphi \lor \psi \Leftrightarrow \lnot(\lnot \varphi \land \lnot \psi)$)
a dokonce všeobecný kvantifikátor
(neboť $(\forall x)(P(x)) \Leftrightarrow (\nexists x)(\lnot P(x))$).

Síť už ale nelze budovat lineárně, nýbrž je nutné stavět ji stromovitě s
rekurzivním vyhodnocováním podmínek. \cite{doorenbos} už navíc neuvádí, jak při
aktivaci pravidel navazovat hodnoty proměnných. Například u disjunktivních
podmínek obdrží sice \verb|production-node| token s posloupností faktů, pokud
mají ale podmínky v disjunkci stejnou strukturu, nevíme, na kterou z podmínek
byl daný fakt návázán.
