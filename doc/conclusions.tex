\section{Závěr}

Ke knihovně ExiL jsem implementoval možnost vrácení a opětovného provedení všech
akcí s vedlejším efektem, včetně inferenčních kroků. Tato funkcionalita je
kompletní, implementace je však neefektivní. V každém kroku se například
kopíruje celá síť RETE. Ukládat pouze inverzní volání však není možné, neboť to
vede ke změně pořadí ukládaných hodnot stavu systému, tudíž ke zvolení jiné
výpočetní větve po navrácení.

Systém podporuje základní syntaxi systému CLIPS. Některé konstrukty však nejsou
podporovány, protože buď v Common Lispu nemají smysl, nebo by vyžadovaly rozsáhlé
úpravy vnitřní implementace systému.

Zpětné řetězení je implementováno v omezené podobě, založené na prohledávání
SLD-stromů v jazyce Prolog. Stejně jako Prolog nepodporuje zpětné řetězení práci
s negací a odstraňování faktů z pracovní paměti systému. Podpora této
funkcionality není problematická implementačně, nýbrž principiálně. Problémy, ke
kterým dochází, jsem popsal v kapitole \ref{impl backward}

Grafické uživatelské rozhraní je funkční, umožňuje však pouze prohlížení stavu
systému a odebírání jednotlivých definic, nikoli jejich přidávání. To by
znamenalo přidat textová definiční pole do jednotlivých rozhraní. K tomu ale
nevidím důvod, neboť definice lze zadávat pomocí REPLu (Listeneru) interpreteru
Common Lispu, přičemž se grafické rozhraní automaticky obnoví. Rozhraní je
svázáno s~prostředím systému, lze tedy používat více prostředí zároveň a ke
každému zobrazit grafické rozhraní.

Podpora kompozitních podmínek není implementována vůbec. Podpora by
vyžadovala rozšíření implementace algoritmu RETE, která je už tak složitá. Síť
RETE navíc velmi rychle roste a obsahuje cykly, v důsledku čehož se velmi těžko
ladí. V kapitole \ref{composite conditions} ale popisuji, jak by tuto
funkcionalitu bylo možné implementovat.

Kromě podpory kompozitních podmínek a některých složitějších konstruktů systému
CLIPS, jako například multisloty, by bylo zajímavé, implementovat podporu
použití objektů CLOSu jako faktů expertního systému. Knihovna by také mohla
podporovat například uživatelsky definované testy v podmínkách pravidel, jako
porovnávání číselných hodnot, nebo rozhodování na základě návratových hodnot
volání lispových funkcí. Implementace této funkcionality by však byla velmi
složitá.

Zajímavá by byla také podpora nejistoty u faktů a odvozovacích pravidel
systému, jakou poskytuje například systém
MYCIN\footnote{\url{http://en.wikipedia.org/wiki/Mycin}}. Tu je už teď možné
nasimulovat připojením hodnot k faktům a jejich manipulací použitím funkčních
alternativ maker v důsledcích pravidel. Kromě rovnosti však není možné tyto
hodnoty testovat v podmínkách pravidel, navíc je tento nepřímý způsob
nepohodlný.

Funkčnost systému jsem testoval na několika malých příkladech, které jsem buď
vytvořil od základu, nebo převzal z \cite{paradigms}. Vytvoření rozsáhlejších
příkladů nebo jejich přepsání z programů vytvořených pro systém CLIPS by bylo
časově velmi náročné. Kód programu je navíc opatřen velkou sadou integračních a
unit testů.
