%%%%%%%%%%%%%%%%%%%%%%%%%%%%%%%%%%%%%%%%%%%%%%%%%%%%%%%%%%%%%%%%%%%%%%%%%%%%%%%%
\subsubsection{Práce s více prostředími}
\label{multiple environments}
Při práci s ExiLem se nemusíme omezovat pouze na jedno prostředí (byť si s ním
často vystačíme). Nové prostředí lze definovat voláním \verb|defenv| s~názvem
prostředí, např. \verb|(defenv test)|. Prostředí pak lze přepnout voláním
\verb|(setenv test)|, případně smazat voláním \verb|(undefenv test)|.

Každé prostředí má oddělený stav, tedy trvalé i dočasné hodnoty (viz předchozí
kapitola). Na název aktuálního prostředí se lze dotázat voláním
\verb|(current-environment)|. Název výchozího prostředí je \verb|default|.
Seznam všech prostředí získáme voláním \verb|(environments)|.

Máme-li už definované prostředí, např. \verb|test|, opětovné volání
\verb|(defenv test)| skončí výjimkou. Tím je zajištěno, že si omylem nevymažeme
celé prostředí.  Chceme-li jej opravdu vymazat, musíme volat
\verb|(defenv test :redefine t)|.
