%%%  Vzor pro použití makra pro reporty
%%%  (c) 2001 Vilém Vychodil, <vilem.vychodil@upol.cz>
%%%
%%%  Po přeložení programem CSLaTeX (třikrát) je potřeba použít
%%%  program DVIPS a takto získaný PostScriptový soubor vytisknout
%%%  na PostScriptové tiskárně nebo pomocí programu GhostScript.
%%%
%%%  Rovněž je možné použít program DVIPDFM a vytvořit z dokumentu
%%%  soubor ve formátu PDF včetně hypertextových odkazů.


%%% Deklarace hlavičky dokumentu. Pokud je potřeba oboustranný dokument,
%%% je možné zapnout jej argument `twoside'.
\documentclass{article}

%%% Připojení dodatečného stylu pro ročníkovou práci. Nepovinné argumenty
%%% `tables' a `figures' použijte pouze v případě, že váš dokument obsahuje 
%%% tabulky a obrázky a chcete vytvořit jejich seznamy za obsahem.
%%% 
%%% Argument `joinlists' způsobí zřetězení obsahu a seznamů tabulek a obrázků.
%%% Není-li použít, všechny seznamy jsou uvedeny na samostatných stránkách.
%%%
%%% Je-li použit argument `seceqn', rovnice budou číslovány v rámci každé
%%% kapitoly. Implicitně jsou rovnice číslovány v rámci celého dokumentu.
%%%
%%% Pokud chcete vytvářet pouze dokument ve formátu PostScript, můžete uvést
%%% dodatečný argument `nopdf'. Tím se potlačí chybová hlášení při použití
%%% programu `dvips'.
\usepackage[tables,figures]{upreport}

%%% Dodatečné standardní styly.
\usepackage[utf8]{inputenc}

%%% Parametry pro vytvoření úvodních stránek. Není-li datum definováno,
%%% je bráno aktuální datum překladu. Makrem \subtitle je možné vytvořit druhý
%%% řádek v názvu reportu.
\title{Název reportu}
%\subtitle{Druhý řádek názvu}
\author{Jan Novák}
\report{TR--CS--99--01}
\date{Říjen 1999}

%%% Pomocí \docinfo je možné vytvořit název pro PDF dokument, zpravidla je
%%% dobré použít předcházející název, ale bez diakritiky. Možné je však zvolit
%%% úpolně jiný výstižný název. Při tvorbě PostScriptu bude příkaz ignorován.
\docinfo{Jan Novak}{Nazev reportu}

%%% Zadání abstraktu díla. Abstrakt může být vynechán.
\abstract{%
Tato anotace by měla stručně popisovat obsah textu a neměla by
přitom přesáhnout zhruba 10 řádků. V~žádném případě by neměla být rozdělena
do více odstavců.}

%%% Poznámky o autorovi a kontakt na něj.
\about{%
  Text vznikl na katedře informatiky, PřF, UP Olomouc.
  Při výrobě textu byl použit textový editor Emacs, typografický systém
  \TeX\ a~operační systém
  \link{Debian GNU/Linux~2.2}{http://www.debian.org}.
  Autor může být kontaktován elektronickou poštou na adrese\/
  \mail{jan.novak@upol.cz}. \\
  \lower\smallskipamount\hbox{{\rm Copyright \copyright\hskip.5em%
      Jan Novák, 2001}}}

%%% Generování indexových frází. Zahájení dokumentu.
\makeindex
\begin{document}

%%% Vytvoření úvodních stránek, obsahu a seznamu tabulek a obrázků.
\maketitle


%%% Obsah textu. Lze jej tradičně dělit do kapitol. Pro zvýraznění větších
%%% bloků je možné využít i části deklarované pomocí \part{}. Obecně se ale
%%% Jejich používání příliš nedoporučuje. Mnohem lepší je využít dodatky.
\section{První kapitola}
Obsah kapitoly.
Zde jsem vycházel z~prací kolegy Složitého, viz \cite{slozi}. 
Lze jej kontaktovat na adrese \mail{tomas.slozity@tezke.problemy.cz}.
\INEM{problém}
\IN{problém!zapeklitý}
\IN{problém!zapeklitý!úplný}
\INEM{problém!a tady je ukázka dlouhého popisu}

\subsection{První podkapitola}\label{kapX}
Toto je text mé podkapitoly číslo \ref{kapX}
Zde jsou uvedeny číslované vztahy,
\begin{align}
  f(1) & = 1, \\
  f(n) & = n \cdot f(n-1).
\end{align}

\subsection{Druhá podkapitola}
Obsah druhé podkapitoly. 
\IN{tělo!druhé kapitoly}

\begin{table}[ht]
  \begin{center}
    \renewcommand{\arraystretch}{1.2}
    \begin{tabular}{||l|rr||}
      \hline
      & \multicolumn{2}{|c||}{\bf \hbox{Informace}} \\
      \cline{2-3}
      \bf Čaj & \bf Cena & \bf Množství \\
      \hline
      Chun Mee & 30\,Kč & 100\,g \\
      Lung Ching & 86\,Kč & 50\,g \\
      Show Mee & 147\,Kč & 50\,g \\
      \hline
    \end{tabular}
    \caption{Toto je tabulka.} \label{tab}
  \end{center}
\end{table}

\subsection{Další podkapitola}
Toto je moje další podkapitola. Tato podkapitola bude dále členěna.

\subsubsection{Podkapitola}\label{podkapX}
Tato podkapitola má číslo \ref{podkapX} Zde budu řešit programování podle
knihy \cite{kovar}. V této části je i obrázek, viz \ref{obr}

\begin{figure}[ht]
  \centerline{\epsfbox{uplogo.eps}}
  \caption{Toto je obrázek.} \label{obr}
\end{figure}

\newpage
\section{Druhá kapitola}\label{podkapY}
Tato část má číslo \ref{podkapY} a~je umístěna na stránce \pageref{podkapY}.
Na stránce uvedu i~odkaz na URL, třeba \url{http://www.inf.upol.cz}.
\IN{problém!silný}
Dále lze vytvářet i~odkazy na kapitoly \emphref{v~rámci dokumentu}{kapX}.
Stejně tak lze vytvářet i~odkazy \emphref{přímo do textu}{dotextu},
pro vytvoření návěstí je potřeba použít makro \verb|\hyplabel|.

\newpage
\section{Třetí kapitola}
Obsah třetí kapitoly.
\IN{problém}\IN{problém!snesitelný}

\medskip
\suboutline{Vložená záložka}
Na toto místo směřuje záložka vytvořená pomocí makra \verb|\suboutline|.

\subsection{Další podkapitola}
\hyplabel{dotextu}
A opět podkapitola. Sem povede odkaz z~textu.

\nextoutline{Jiné jméno}
\subsection{A ještě další podkapitola}
Zase jedna podkapitola. Tato kapitola se bude v~záložkách jmenovat jinak.


%%% Vytvoření seznamu literatury.
\newpage
\begin{thebibliography}{MMM99}

\bibitem[Smi90]{smith}
  Smith, John. \emph{User and program.}
  Publisher, City, 1990.

\bibitem[Kov91]{kovar}
  Kovář, Jan. \emph{Jak programovat.}
  Nakladatelství, Město, 1991.

\bibitem[Slo01]{slozi} 
  Složitý, Tomáš.
  \link{\emph{Překladač s~nakladačem.}}{http://www.inf.upol.cz}
  Elektronická publikace, 2001.

\end{thebibliography}


%%% Dodatky. V některých případech se dodatky sázejí písmem petit (stupeň 8),
%%% Pokud tak chcete učinit, uzavřete každou sekci do bloku a na jejím začátku
%%% použijte makro \small.
\appendix
\newpage
\section{První dodatek}
Text prvního dodatku.

\subsection{Sekce v prvním dodatku}
Dodatky lze rovněž členit, i když se to příliš nedoporučuje.

\newpage
\section{Druhý dodatek}
Text druhého dodatku.


%%% Rejstřík. Redefinicí makra \indexcolumns lze měnit počet sloupců rejstříku,
%%% implicitně je rejstřík vytvářen do dvou sloupců. Makrem \printindex je
%%% rejstřík vložen do dokumentu.
%%%
%%% Pokud je potřeba změnit styl zvýraznění podstatných stránek, je možné 
%%% redefinovat makro \indexemph[1]. Například redefinicí
%%% \renewcommand{\indexemph}[1]{{\bfseries #1}}
%%% způsobí ztučnění stránek vkládaných pomocí \INEM{}.
%%%
%%% Styl sazby rejstříku lze dále ovlivnit i redefinicí makra \indexfashion,
%%% například při definici \renewcommand{\indexfashion}{\footnotesize} bude
%%% rejstřík sázen písmem velikosti poznámky pod čarou.
\renewcommand{\indexcolumns}{3}
\printindex

\end{document}
