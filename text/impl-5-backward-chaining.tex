\subsubsection{Zpětná inference}

Zpětná inference hledá odpovědi na základě zadaných cílů. Odpovědi jsou zde
reprezentovány použitou substitucí proměnných v cílech a posloupností faktů
a~pravidel, použitých k jejich splnění. Možnosti zpětného řetězení a způsob jeho
použití jsem popsal v sekci \ref{backward inference}

Kód implementující zpětnou inferenci pracuje se dvěma datovými strukturami -
seznamem cílů a zásobníkem pro backtracking (návrat ve výpočtu, pokud daná cesta
nevede ke splnění všech cílů, nebo pro hledání alternativních odpovědí). Program
nemodifikuje pracovní paměť, ani nevyhodnocuje odvozovací pravidla, pouze
analyzuje volání \verb|assert| v jejich důsledcích.

Cíle jsou v programu reprezentovány vzory, podobně jako podmínky pravidel.
Inferenci si tedy můžeme představit tak, jako bychom definovali jedno pravidlo,
jehož podmínky reprezentují seznam cílů a hledali všechny možné cesty výpočtu,
vedoucí k jeho splnění.

Zásobník pro backtracking (dále jen zásobník) ukládá struktury, z nichž každá
reprezentuje právě provedený krok výpočtu. Tato struktura je tvořen aktuálním
seznamem cílů, faktem či pravidlem, použitým pro splnění aktuálního cíle a
seznamem faktů a pravidel, která již byla pro splnění tohoto cíle použita (v
případě návratu výpočtu).

Zpětná inference je řízena třemi metodami prostředí - \verb|back-step|,
\verb|backtrack| a \verb|back-run|. Metoda \verb|back-step| vybere první cíl ze
seznamu a hledá nejprve fakta, poté pravidla, vedoucí k jeho splnění, přičemž
ignoruje ta, která už byla pro splnění cíle dříve použita.

Při hledání faktů, která splňují cíl, srovnáváme fakt se vzorem cíle podobně,
jako při vyhodnocování podmínek pravidel při dopředné inferenci (viz sekce
\ref{inference}). Výsledkem tohoto srovnání je v případě shody substituce
proměnných, vyskytujících se ve vzoru cíle. Je-li nalezen fakt, který aktuální
cíl splňuje, program uloží nalezenou shodu na zásobník, načež cíl odstraní ze
seznamu cílů a~aplikuje použitou substituci proměnných na zbytek cílů. Není-li
nalezen takový fakt, uvažuje program dále odvozovací pravidla.

Při hledání pravidel, která vedou ke splnění aktuálního cíle, program zohledňuje
všechna volání \verb|assert| v jejich důsledcích. Reprezentace faktů, použité ve
voláních \verb|assert|, nechá program zpracovat \verb|parser|em a výsledné fakty
pak srovnává se vzorem cíle podobně, jako fakty v pracovní paměti. Protože jde
však o~důsledkovou část pravidla, volání \verb|assert| mohou obsahovat proměnné.
Program tedy hledá unifikaci aktuálního cíle s tímto \uv{proměnným faktem}.

V případě shody je výsledkem unifikace opět substituce proměnných. Program v
tomto případě opět uloží shodu na zásobník, odstraní aktuální cíl ze seznamu a
aplikuje použitou substituci proměnných na zbytek cílů. Nejpreve ale program
přidá podmínky použitého pravidla do seznamu cílů. Podmínky použitého pravidla
se tedy stávají dalšími cíli, které je třeba splnit, abychom nalezli hledanou
odpověď.

Hledání unifikace vzoru cíle s \uv{proměnným faktem} je podobné jako v jazyce
Prolog. Je však jednodušší o to, že symbolická reprezentace faktů a vzorů je
lineární (seznamy nejsou vnořené) a není třeba rozlišovat mezi relačními
a~funkčními symboly. V případě strukturovanch faktů a vzorů jsou jednotlivé
sloty srovnávány, jako by šlo o atomy seznamu, jen zde nezáleží na pořadí, nýbrž
se srovnávají odpovídající sloty faktu a vzoru.

Pokud metoda \verb|back-step| nenalezne fakt ani pravidlo vedoucí ke splnění
aktuálního cíle, volá metodu \verb|backtrack|. Ta odstraňuje postupně struktury z
vrcholu zásobníku, obnoví cíle do stavu při uložení struktury a hledá shody s
dosud nepoužitými fakty a pravidly. Pokud takovou shodu nalezne, ubírá se
výpočet dále touto cestou. Je-li celý zásobník vyčerpán před nalezením shody,
nahlásí metoda neúspěch.

Metoda \verb|back-run| volá opakovaně metodu \verb|back-step|, dokud nejsou
splněny všechny cíle, nebo není nahlášen neúspěch. V případě, že už jsou všechny
cíle splněny, volá \verb|back-step| metodu \verb|backtrack| pro nalezení
alternativních odpovědí. To může vést k nalezení nové shody, nebo vyčerpání
zásobníku, což znamená, že už žádné další odpovědi neexistují.

Dospěje-li volání \verb|back-run| ke splnění všech cílů, vytiskne se seznam
faktů a~pravidel, použitých ke splnění jednotlivých cílů. Dále se vytiskne
substituce proměnných, která byla pro splnění cílů použita. Tato substituce je
tvořena složením všech substitucí, použitých v jednotlivých krocích výpočtu. Ve
výsledné substituci jsou ponechány pouze proměnné, které uživatel použil v
definicích původních cílů. Průběžné proměnné, které se sem dostaly při aplikaci
odvozovacích pravidel nejsou pro uživatele zajímavé.

Implementace zpětné inference je inspirována prohledáváním SLD-stromů používaným
v jazyce
Prolog\footnote{\url{http://en.wikipedia.org/wiki/SLD_resolution\#SLD_resolution_strategies}}.
Z toho také vyplývají její omezení. Implementace pracuje pouze s \uv{pozitivní
znalostí} - neumožňuje zadání negativních cílů, ani negované podmínky pravidel.
Implementace také uvažuje pouze přidání nových faktů v důsledcích pravidel,
nikoli jejich odebrání nebo změnu. Díky tomu, že se důsledky pravidel
nevyhodnocují, nedochází navíc k vedlejším efektům jako při dopředné inferenci.

\begin{framed}
  \begin{itemize}
    \item není implementačně složité, ale problém je v tom, jak vyhodnocovat
      retract v důsledcích (podobně s modify)
    \item pokud retract vyhodnocován jen pro splnění negativních podmínek (jinak
      ignorován), problém se zneplatněním následujícího pravidla
    \item pokud retract faktu znamená, že nemůžeme pravidlo použít, pokud máme
      fakt v cílech, nenajde řešení, pokud je cesta okolo - bankomat
    \item pokud by měl nějak odstranit fakta ze seznamu - jak? kdy?
      \begin{itemize}
        \item hned - zneplatní vlastní podmínky
        \item po splnění podmínek? může být pozdě - obě pravidla jsou už
          splněna, i když jedno zneplatňuje druhé
        \item fakta k odstranění by se mohla přidávat nějaký seznam, odstraňovat
          na konci a hlásit problém, pokud je nějaký fakt odstraňován vícekrát,
          nemůže dojít k problémům s pořadím odstraňování - co když nějaké další
          pravidlo má negativní podmínku, která vyžaduje, aby byl fakt odstraněn
          dříve
        \item podrobnější rozbor je nad rámec práce
      \end{itemize}
    \item podobně může dojít ke zneplatnění negativní podmínky assertem (ten je
      zatím zohledňován jen pro splnění pozitivních podmínek $\rightarrow$
      stejné problémy
    \item modify dokonce obojí zároveň
    \item rozdělit na dvě kapitoly?
    \item v teoretické části rozebrat, jak tyto problémy řeší GPS v PoAIP
  \end{itemize}
\end{framed}

Uvažovat negativní cíle, negované podmínky pravidel a odebírání faktů v
důsledcích pravidla by nebylo implementačně náročné. Hrozí zde ovšem problém, že
důsledky některého z pravidel zneplatní podmínky dalšího. Uvažujme následující
zadání:
\begin{minted}{cl}
(defgoal (at home))

(deffacts initial
  (out of city)
  (have money))

(defrule take-bus-to-city
  (have money)
  (out of city)
  =>
  (assert (at city))
  (retract (have money)))

(defrule take-taxi-home
  (have money)
  (at city)
  =>
  (at home)).
\end{minted}
Dopředná inference v tomto případě korektně selže v odvození faktu (at home),
neboť aplikace pravidla \verb|take-bus-to-city| odstraní fakt
\verb|(have money)|, tudíž druhé pravidlo už nemůže být splněno. Dopředná
inference

\begin{framed}
  posunout následující odstavec výš?
\end{framed}

Implementace nevyužívá sítě RETE, neboť ta vyhodnocuje splnění podmínek
odvozovacích pravidel fakty v pracovní paměti. Zpětná inference ale pracuje
opačně - analyzuje důsledky pravidel a pracovní paměť nemodifikuje. Síť RETE
tedy nelze k implementaci využít.
