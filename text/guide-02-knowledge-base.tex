\subsubsection{Definice znalostní báze}
\label{knowledge base definition}

ExiL, stejně jako CLIPS, rozlišuje dva typy faktů - jednoduché (\emph{simple,
ordered}) a strukturované (\emph{templated}). Stuktura jednoduchého faktu je udána
pouze pořadím \emph{atomů}, typickou volbou je např. \verb|objekt-attribut-hodnota|:
\cl|(box color red),| či \verb|relace-objekty|: \cl|(in box hall).|

Strukturované fakty mají naproti tomu explicitně pojmenované složky (sloty).
Typicky popisují objekt s množinou pojmenovaných atributů: \cl|(box :color red :size small),|
či relaci s pojmenovanými aktory: \cl|(in :object box :location hall),| kde
\verb|box| a \verb|in| jsou šablony (\emph{template}), které je třeba definovat
předem. Na pořadí specifikace slotů u strukturovaných faktů
nezáleží.

Vyjadřovací síla obou typů faktů je stejná, použitím explicitnějších
strukturovaných faktů ale docílíme lepší čitelnosti a jednoznačnější sémantiky
exilového programu, zvláště třeba v případě relací na jedné množině objektů:
\cl|(father john george).|

Šablonu definujeme voláním \emph{makra} \verb|deftemplate|, např:
\cl|(deftemplate in object (location :default here)).| Prvním parametrem je
název šablony, za ním následuje libovolný počet specifikací slotů. Specifikací
slotu je buď symbol - jméno slotu, nebo \emph{seznam}, jehož hlavou
(\emph{car}) je jméno slotu a~tělem (\emph{cdr}) je \emph{property list (plist)}
s dalšími parametry. Aktuálně systém umožňuje pouze specifikaci výchozí hodnoty
slotu \emph{klíčem} \verb|:default|. Ta je použita, není-li při specifikaci
faktu, používajícího tuto šablonu, uvedena hodnota pro daný slot. Příklad
\ref{structured facts} na straně \pageref{structured facts} ukazuje definici
znalostní báze ekvivalentní příkladu \ref{typical structure} s použitím
strukturovaných faktů.

\begin{listing}[h]
\caption{Definice znalostní báze s použitím strukturovaných faktů}
\label{structured facts}
\begin{clcode}
(deftemplate goal (action :default move) object from to)
(deftemplate in object location)

(deffacts world
  (in :object robot :location A)
  (in :object box :location B)
  (goal :object box :from B :to A))

(defrule move-robot
  (goal :object ?obj :from ?from)
  (in :object ?obj :location ?from)
  (- in :object robot :location ?from)
  ?robot <- (in :object robot :location ?)
  =>
  (modify ?robot :location ?from))

(defrule move-object
  (goal :object ?obj :from ?from :to ?to)
  ?object <- (in :object ?obj :location ?from)
  ?robot <- (in :object robot :location ?from)
  =>
  (modify ?robot :location ?to)
  (modify ?object :location ?to))

(defrule stop
  ?goal <- (goal :object ?obj :to ?to)
  (in :object ?obj :location ?to)
  =>
  (retract ?goal))
\end{clcode}
\end{listing}

Je-li už šablona požadovaného názvu definována, ale neexistují v pracovní
paměti fakty, které ji používají, je její stávající definice nahrazena. Pokud
ale v~pracovní paměti existují takové fakty, skončí volání \verb|deftemplate|
výjimkou.

Seznam názvů všech definovaných šablon můžeme získat voláním \verb|(templates)|.
Specifikaci šablony pak získáme voláním makra
\verb|find-template|, např. \verb|(find-template in).| Definici šablony zrušíme
voláním makra \verb|undeftemplate|, např. \verb|(undeftemplate goal)|. To opět
skončí výjimkou, existují-li v pracovní paměti fakty, které šablonu využívají.

Fakty, ze kterých expertní systém vychází, zavádíme pomocí skupin faktů. Ty
definujeme makrem \verb|deffacts|, např.:
\begin{minted}{cl}
(deffacts initial
  (goal move box A B)
  (in :object box :location A))
\end{minted}
Prvním parametrem je název skupiny, pak následuje libovolný počet specifikací
faktů. Opakovaným voláním makra \verb|deffacts| je skupina faktů
redefinována.

Specifikace faktu je vždy tvořena seznamem. Pokud jde o jednoduchý fakt,
specifikací je prostě seznam atomů. Jde-li o fakt strukturovaný, je prvním
prvkem specifikace název šablony, za ním následuje plist určující hodnoty slotů
faktu. Pokud není hodnota některého slotu uvedena, je buď použita
výchozí hodnota, pokud byla v šabloně specifikována, nebo hodnota \verb|nil| v
opačném případě.

Seznam názvů všech definovaných skupin faktů získáme voláním
\verb|(fact-groups)|. Specifikaci skupiny pak voláním makra
\verb|find-fact-group|, např. \verb|(find-fact-group initial)|. Ke zrušení
definice skupiny slouží makro \verb|undeffacts| (voláme s názvem skupiny).

Pravidla, pomocí nichž expertní systém během inference odvozuje nové fakty,
definujeme makrem
\verb|defrule|, např.:
\begin{minted}{cl}
(defrule move-robot
  "move robot to object's location"
  (goal :action move :object ?obj :from ?from)
  (in :object ?obj :location ?from)
  (- in :object robot :location ?from)
  ?robot <- (in :object robot :location ?)
  =>
  (modify ?robot :location ?from)).
\end{minted}
Podmínková část pravidla (před symbolem \verb|=>|) je tvořena vzory. Ty mohou
být, stejně jako fakty, jednoduché, nebo strukturované. Kromě toho umožňuje
definice pravidla několik speciálních konstruktů (negace podmínky, navázání
proměnné na celou podmínku). Ty popíšu podrobně v sekci \ref{inference} spolu~s
tím, jak jsou podmínky pravidla při inferenci vyhodnocovány.

Důsledkovou část pravidla (za symbolem \verb|=>|) tvoří libovolný počet
lispových výrazů. Jak se tyto vyhodnocují popíšu opět v sekci \ref{inference}

\FloatBarrier

Opakovaným voláním makra \verb|defrule| odvozovací pravidlo redefinujeme.
K~získání seznamu názvů definovaných pravidel a jejich specifikací slouží
\emph{funkce} \verb|rules| a makro \verb|find-rule|, podobně jako u šablon a
skupin faktů.  Ke zrušení definice pravidla slouží makro \verb|undefrule|.
