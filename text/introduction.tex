%%%%%%%%%%%%%%%%%%%%%%%%%%%%%%%%%%%%%%%%%%%%%%%%%%%%%%%%%%%%%%%%%%%%%%%%%%%%%%%%
\section{Úvod}

Pojem expertního systému spadá do oblasti umělé inteligence. Jde o počítačový
systém, který simuluje rozhodování experta nad zvolenou problémovou doménou.
Expertní systém může experta zcela nahradit, nebo mu při rozhodování asistovat.

Ve své bakalářské práci \cite{bakalarka} jsem implementoval základní knihovnu
pro tvorbu expertních systémů (tzv. prázdný expertní systém) s~dopředným
řetězením v~jazyce Common Lisp. Cílem této práce je knihovnu rozšířit o
\begin{itemize}
  \item syntaktický režim pro zajištění přiměřené kompatibility se systémem
    CLIPS,
  \item možnost vrácení provedených změn včetně odvozovacích kroků,
  \item základní zpětné řetězení,
  \item podporu pro ladění s jednoduchým grafickým uživatelským rozhraním pro
    prostředí LispWorks\texttrademark,
  \item podporu kompozitních podmínek (vnořené AND, OR a NOT) a všeobecné
    kvantifikace.
\end{itemize}

Jazyk Common Lisp\footnote{\url{http://en.wikipedia.org/wiki/Common\_Lisp}}
(případně jiné dialekty Lispu) je častou volbou pro implementaci umělé
inteligence díky svým schopnostem v oblasti symbolických výpočtů (manipulace
symbolických výrazů), na nichž řešení těchto problémů často staví. Navíc jde
o~vysokoúrovňový, dynamicky typovaný jazyk, díky čemuž je programový kód
stručný, snadno pochopitelný a tudíž jednoduše rozšiřitelný.

Syntax systému CLIPS\footnote{\url{http://clipsrules.sourceforge.net}} byla
zvolena proto, že jde o reálně používaný
systém\footnote{\url{http://clipsrules.sourceforge.net/FAQ.html\#Q6}}, jehož
syntax je Lispu velmi blízká, takže není těžké ji v~Lispu napodobit.

\begin{framed}
  Kdo už to dělal, čím se tohle liší
\end{framed}
