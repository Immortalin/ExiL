%%%%%%%%%%%%%%%%%%%%%%%%%%%%%%%%%%%%%%%%%%%%%%%%%%%%%%%%%%%%%%%%%%%%%%%%%%%%%%%%
\subsubsection{Undo/redo}

Jedním z implementovaných rozšíření původního programu je schopnost vrácení
provedených změn. K tomu slouží makra \verb|undo| a \verb|redo|. Ta lze použít k
vrácení jakékoli akce s vedlejším efektem, včetně kroků inference. K vypsání
zásobníků s~akcemi, které je možné vrátit, jsou k dispozici makra
\verb|undo-stack| a \verb|redo-stack|.

Pokud například vyhodnotíme prvních 32 řádků programu \ref{typical structure} na
straně \pageref{typical structure} a~zavoláme dvakrát \verb|undo|, bude výpis
zásobníků následující (přeformátováno):
\begin{minted}[samepage]{cl}
EXIL-USER> (undo-stack)
  1: (defrule MOVE-ROBOT
       ((GOAL MOVE ?OBJECT ?FROM ?TO)
        (IN ?OBJECT ?FROM)
        (- IN ROBOT ?FROM) (IN ROBOT ?Z)
        =>
        (RETRACT (IN ROBOT ?Z))
        (ASSERT (IN ROBOT ?FROM))))
  2: (deffacts WORLD
       ((IN BOX A) (IN ROBOT B) (GOAL MOVE BOX A B)))
\end{minted}
\begin{minted}[samepage]{cl}
EXIL-USER> (redo-stack)
  1: (defrule MOVE-OBJECT
       ((GOAL MOVE ?OBJECT ?FROM ?TO)
        ?OBJ-POS <- (IN ?OBJECT ?FROM)
        ?ROB-POS <- (IN ROBOT ?FROM)
        =>
        (RETRACT ?ROB-POS)
        (RETRACT ?OBJ-POS)
        (ASSERT (IN ROBOT ?TO))
        (ASSERT (IN ?OBJECT ?TO))))
  2: (defrule STOP
       ((GOAL MOVE ?OBJECT ?FROM ?TO)
        (IN ?OBJECT ?TO)
        =>
        (HALT)))
\end{minted}
Vidíme tedy, že jsme vrátili zpět definice pravidel \verb|move-object| a
\verb|stop| (ty bychom mohli opět provést voláním \verb|(redo)|). Dalším voláním
\verb|(undo)| by pak byla vrácena definice pravidla \verb|move-robot| a poté
definice skupiny faktů \verb|world|.

Nemá-li akce žádný vedlejší efekt - např. volání \verb|assert| s faktem, který
už v~pracovní paměti je, či volání \verb|run| ve chvíli, kdy už není co
odvozovat - prázdná akce se na zásobník neuloží.

\begin{framed}
popsat chování undo v kombinaci s reset
\end{framed}
