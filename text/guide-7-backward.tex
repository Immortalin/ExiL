%%%%%%%%%%%%%%%%%%%%%%%%%%%%%%%%%%%%%%%%%%%%%%%%%%%%%%%%%%%%%%%%%%%%%%%%%%%%%%%%
\subsubsection{Zpětná inference}
\begin{framed}
  \begin{itemize}
    \item cíle jako patterny, vypsání
    \item základní inference - nejdřív fakty, pak pravidla, v jakém pořadí
      vybírá
    \item alternativní odpovědi - backtracking
  \end{itemize}
\end{framed}

Dalším z implementovaných rozšíření je možnost zpětné inference. Inference
popsaná v sekci \ref{inference} je dopředná. V každém kroku jsou nalezeny
všechny možnosti dalšího postupu odvozování, načež je zvolena jedna, kterou se
program dále ubírá. To činí průběh inference značně nedeterministickým. Možnosti
postupu, které nebyly vybrány, mohou být navíc dalším postupem ztraceny, pokud
aktivace některého pravidla zneplatní podmínky jiného.
Míru nedeterminismu můžeme snížit tím, že budeme navrhovat odvozovací pravidla
tak, aby se výpočet neubíral nechtěnými cestami. To ale není vždy jednoduché,
nebo dokonce možné.

Zpětná inference naproti tomu umožňuje definovat cíle, kterých chceme dosáhnout.
K tomu slouží makro \verb|defgoal|, kterému předáme vzor ve stejném formátu,
jako u podmínek pravidel. Definice cíle ovšem nepodporuje negaci ani navázání
faktu na proměnnou (k tomu ani není důvod).

Ke spuštění zpětné inference pak slouží funkce \verb|back-step| a
\verb|back-run|, podobně jako u inference dopředné.

Mějme následující znalostní bázi:
\begin{minted}{cl}
(deffacts world
  (have-money))

(defrule buy-car
  (have-money)
  =>
  (retract (have-money))
  (assert (have-car)))

(defrule pay-rent
  (have-money)
  =>
  (retract (have-money))
  (assert (rent-payed))).
\end{minted}
Spustíme-li dopřednou inferenci, systém nám vesele doporučí nákup auta.
Mít nové auto je sice pěkné, hrozí-li nám ale vyhození z pronajatého bytu,
není nákup auta pravděpodobně cestou, kterou bychom se chtěli ubírat.
Systém by nám v tuto chvíli mohl stejně dobře doporučit správnou cestu. Že bylo
vybráno zrovna první pravidlo je výsledkem toho, jak funguje síť RETE, která
pravidla vyhodnocuje. Za daných okolností ale nechceme špatnou variantu ani
připouštět.

V tomto případě bychom mohli upravit definici programu tak, že do znalostní báze
přidáme informaci o cíli, kterou budou pravidla zohledňovat, podobně jako
v~příkladu \ref{typical structure} na straně \pageref{typical structure}. Muset
ale programovat zohlednění cíle v každém pravidle je přinejmenším otravné. U
větších programů to navíc může být velmi náročné, neboť cíl bude třeba
programově modifikovat v průběhu výpočtu.

S použitím zpětné inference je problém podstatně jednodušší. Zavoláme-li
\begin{minted}{cl}
(reset)
(defgoal (rent-payed))
(back-run),
\end{minted}
bude výsledkem výstup
\begin{minted}[samepage]{cl}
All goals have been satisfied
(RENT-PAYED) satisfied by (RULE PAY-RENT
  (HAVE-MONEY)
  =>
  (RETRACT (HAVE-MONEY))
  (ASSERT (RENT-PAYED)))
(HAVE-MONEY) satisfied by (HAVE-MONEY).
\end{minted}
Zde vidíme, že po spuštění zpětné inference nezačal systém bezhlavě provádět
akce, ke kterým měl dostatečné prostředky. Místo toho systém uvážil zadaný cíl a
jal se hledat akce, které k jeho splnění směřují.

Uvažme nyní složitější příklad (definice šablon vynechána):
\begin{minted}[samepage]{cl}
(deffacts world
  (female jane)
  (male john)
  (parent :parent jane :child george)
  (parent :parent john :child george))

(defrule father-is-male-parent
  (male ?father)
  (parent :parent ?father :child ?child)
  =>
  (assert (father :father ?father :child ?child)))

(defrule mother-is-female-parent
  (female ?mother)
  (parent :parent ?mother :child ?child)
  =>
  (assert (mother :mother ?mother :child ?child)))
\end{minted}
Zajímá-li nás, kdo je matkou George, můžeme zkusit spustit dopřednou inferenci.
Po jejím skončení bude v pracovní paměti jak informace o Georgově matce, tak o
jeho otci. Systém se tedy v tomto případě dobral správného výsledku, vypočítal
ale i další fakty, které nás nezajímaly. Dokážeme si snadno představit, že ve
větším programu může být výpočet všech odvoditelných závěrů velmi výpočetně
a tudíž i časově náročný.

Spustíme-li naopak zpětnou inferenci voláním
\begin{minted}[samepage]{cl}
(reset)
(defgoal (mother :mother ?mother-of-george :child george))
(back-run),
\end{minted}
je výsledkem
\begin{minted}[samepage]{cl}
All goals have been satisfied
(MOTHER (MOTHER . ?MOTHER-OF-GEORGE) (CHILD . GEORGE)) satisfied by
  (RULE MOTHER-IS-FEMALE-PARENT
    (FEMALE ?MOTHER)
    (PARENT (PARENT . ?MOTHER) (CHILD . ?CHILD))
    =>
    (ASSERT (MOTHER MOTHER ?MOTHER CHILD ?CHILD)))
(FEMALE ?MOTHER) satisfied by (FEMALE JANE)
(PARENT (PARENT . JANE) (CHILD . GEORGE)) satisfied by
  (PARENT (PARENT . JANE) (CHILD . GEORGE))
These variable bindings have been used:
((?MOTHER-OF-GEORGE . JANE))
\end{minted}
Systém tedy vyvodil pouze závěr, který nás zajímal.

Zpětná inference umožnuje také výpočet alternativních odpovědí, tedy dalších
cest výpočtu (a vazeb proměnných), které vedou ke splnění všech cílů. Na další
alternativní odpověď se dotážeme jednoduše opětovným voláním \verb|(back-run)|.

Zajímají-li nás například oba rodiče George, můžeme zadat cíl
\cl|(defgoal (parent :parent ?parent :child george)).|
Pokud poté třikrát zavoláme \verb|(back-run)|, obdržíme výstup
\begin{minted}[samepage]{cl}
All goals have been satisfied
(PARENT (PARENT . ?PARENT) (CHILD . GEORGE)) satisfied by
  (PARENT (PARENT . JANE) (CHILD . GEORGE))
These variable bindings have been used:
((?PARENT . JANE))

All goals have been satisfied
(PARENT (PARENT . ?PARENT) (CHILD . GEORGE)) satisfied by
  (PARENT (PARENT . JOHN) (CHILD . GEORGE))
These variable bindings have been used:
((?PARENT . JOHN))

No feasible answer found.
\end{minted}
Systém tedy najde obě možné odpovědi (vazby proměnných), vedoucí ke splnění
cíle, načež oznámí, že další odpověď už neexistuje.

Zpětná inference nemodifikuje obsah pracovní paměti. To ani není možné vzhledem
k tomu, že ve chvíli, kdy inference zvolí pravidlo, jehož důsledky vedou ke
splnění aktuálního cíle, nejsou jeho podmínky často ještě splněny. Místo toho
pracuje zpětná inference pouze s množinou cílů. Jednotlivé cíle jsou postupně
vybírány a hledají se cesty k jejich splnění. Inference nejprve zkoumá fakty
v~pracovní paměti. Není-li aktuální cíl splněn žádným z platných faktů, uvažuje
dále jednotlivá pravidla. Najde-li pravidlo, které by po aktivaci vedlo ke
splnění aktuálního cíle, naváže proměnné použité v jeho aktivacích na jeho
podmínky (tedy opačně než u inference dopředné) a ty pak přidá do množiny cílů.
Použité vazby proměnných se navíc aplikují i na zbytek cílů.

Zkusme nyní krokovat (použitím \verb|(back-step)|) předchozí příklad s dotazem na
matku George s průběžným výpisem cílů pomocí \verb|(goals)|.
\begin{minted}[samepage]{cl}
GOALS: ((MOTHER :MOTHER ?MOTHER-OF-GEORGE :CHILD GEORGE))
(MOTHER (MOTHER . ?MOTHER-OF-GEORGE) (CHILD . GEORGE)) satisfied by
  (RULE MOTHER-IS-FEMALE-PARENT
    (FEMALE ?MOTHER)
    (PARENT (PARENT . ?MOTHER) (CHILD . ?CHILD))
    =>
    (ASSERT (MOTHER MOTHER ?MOTHER CHILD ?CHILD)))

GOALS: ((FEMALE ?MOTHER) (PARENT :PARENT ?MOTHER :CHILD GEORGE))
(FEMALE ?MOTHER) satisfied by (FEMALE JANE)

GOALS: ((PARENT :PARENT JANE :CHILD GEORGE))
(PARENT (PARENT . JANE) (CHILD . GEORGE)) satisfied by
  (PARENT (PARENT . JANE) (CHILD . GEORGE))
\end{minted}
V prvním kroku je proměnná \verb|?mother-of-george|, použitá v definici cíle,
nahrazena proměnnou \verb|?mother|, použitou v důslecích pravidla
\verb|mother-if-female-parent|. Proměnná \verb|?child| v důsledcích je navázána
na \verb|george| a touto vazbou jsou nahrazeny výskyty proměnné v podmínkách
pravidla. Podmínky jsou poté přidány do množiny cílů.

V druhém kroku je nový cíl \verb|(female ?mother)| porovnán s faktem
\verb|(female jane)| v pracovní paměti, je jím splněn a v posledním cíli je
proměnná \verb|?mother| navázána na \verb|jane|. Tento cíl je pak
v posledním kroku triviálně splněn identickým faktem.


\begin{framed}
  \begin{itemize}
    \item pouze assert v důsledcích
    \item zpětná inference nemusí nutně snížit míru nedeterminismu, pokud
    existuje hodně pravidel, která splní daný subcíl
  \end{itemize}
\end{framed}
