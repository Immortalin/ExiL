\subsection{Expertní systém}

Expertní systém je počítačový program, který usuzuje na základě znalostí
z~nějaké \emph{problémové domény} za účelem vyřešení zadaného problému nebo
poskytnutí odpovědi na otázku \cite{introduction}. Problémovou doménou zde
rozumíme množinu znalostí, omezenou na určitý okruh, který nás zajímá, s
vyloučením všech ostatních, irelevantních znalostí \cite{problem-domain}.
Znalosti z problémové domény jsou reprezentovány jejím
\emph{modelem}\footnote{\url{http://en.wikipedia.org/wiki/Domain\_model}}.
Expertní systém tedy pracuje s modelem zvolené problémové domény.

\begin{framed}
  \url{http://en.wikipedia.org/wiki/Expert\_system#Applications} poskytuje dobré
  příklady aplikací
\end{framed}

K typickým problémům, řešeným pomocí expertních systému patří
\cite{introduction}
\begin{itemize}
  \item plánování posloupnosti akcí vedoucích k zadanému cíli (např.
    plánování trasy ze znalosti dopravních spojů a dopravní situace),
    \emph{nevhodný příklad - potřeba numerických výpočtů při výběru z více tras
    - ne že by nešlo, ale není typickým příkladem}
  \item interpretace dat (např. detekce nebezpečí z dat získaných z
    termografické kamery), analýza složení (vyhodnocení dat získaných metodami
    analytické chemie),
  \item diagnóza a plánování oprav poruch systému,
  \item konfugurace komplexních objektů (např. navržení serverového cloudu pro
    poskytnutí zadaných služeb).
\end{itemize}

V první kapitole své bakalářské práce \cite{bakalarka} uvádím typické vlastnosti
expertních systémů, jejich zařazení v rámci informatiky a umělé inteligence a
jejich rozdíly oproti jiným typům programů, které daný problém mohou řešit.
Stěžejní charakteristikou expertního systému je úplné oddělení reprezentace
znalostí, nad kterými systém usuzuje, od samotného \emph{odvozovacího aparátu}.
Odvozovací aparát je tedy obecný a je schopen usuzovat nad jakoukoli problémovou
doménou, jejíž model lze v expertním systému reprezentovat.

Jak vyplývá z uvedené definice, expertní systém je hotový program schopný
odpovídat na dotazy, či řešit problémy v rámci zvolené problémové domény. Tento
program můžeme vytvořit od základu v nějakém obecném programovacím jazyce. To je
ale poměrně složité a pokud nemáme specifické nároky, nevyplatí se to. Častěji
použijeme existující \emph{nástroj} či \emph{knihovnu pro tvorbu expertních
systémů}. Tento nástroj už obsahuje odvozovací aparát a definuje reprezentaci
znalostí a~dotazů a~její syntax (viz sekce \ref{knowledge representation}).
Poskytuje tedy tzv. \emph{prázdný expertní systém}. Takovým nástrojem je i
knihovna ExiL, která je přílohou této práce.

Zařazení expertního systému do procesu sestává z následujících fází:
\begin{enumerate}
  \item sestavení modelu zvolené problémové domény,
  \item výběr nástroje pro tvorbu expertních systémů podle typu domény a dotazů,
    které chceme být schopni zadávat,
  \item reprezentace modelu problémové domény použitím syntaxe zvoleného
    nástroje.
\end{enumerate}
Vytvořená reprezentace modelu problémové domény tvoří po načtení nástrojem pro
tvorbu expertních systémů tzv. \emph{znalostní bázi} výsledného expertního systému.
Tomu pak můžeme zadávat dotazy buď přímo, použitím jeho syntaxe, nebo skrze
nějaké uživatelské rozhraní. Je-li zvolený nástroj navržen jako knihovna, můžeme
také vytvořit další program, který zadává znalosti do expertního systému a/nebo
mu zadává dotazy a odpovědi na ně pak dále zpracovává.

Jednou z běžných vlastností expertních systémů je také schopnost vysvětlit, jak
systém k řešení problému dospěl. To umožňuje evaluaci správnosti řešení
a~poskytuje tak zpětnou vazbu pro případnou opravu znalostní báze, zadaného cíle
či dotazu, nebo odvozovacího aparátu (v případě vlastního návrhu expertního
systému).

Expretní systém nelze použít k řešení jakéhokoli typu problému. To je dáno
reprezentací znalostí v expertním systému a způsobem, jakým systém nad těmito
znalostmi usuzuje. Vlastnosti znalostí, které lze v expertním systému
reprezentovat, jejich reprezentace, způsoby zadávání cílů a dotazů i to, jak
systém zadaných cílů dosahuje, je náplní následujících sekcí.
