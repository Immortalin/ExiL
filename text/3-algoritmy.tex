%%%%%%%%%%%%%%%%%%%%%%%%%%%%%%%%%%%%%%%%%%%%%%%%%%%%%%%%%%%%%%%%%%%%%%%%%%%%%%%%
\section{Použité algoritmy}
Přiložená implementace expertního systému zahrnuje několik z~literatury
čerpaných algoritmů. V~prvé řadě je to algoritmus Rete, který se stará o~pattern
matching podmínek odvozovacích pravidel proti faktům znalostní baze, tedy
o~zjišťování, která odvozovací pravidla je možno aplikovat v~dalším kroku výpočtu.
Poté je použito několik velice jednoduchých algoritmů pro výběr pravidla, jenž
bude jako další aktivováno.

\subsection{Algoritmus Rete}

Algoritmus Rete (materiály nabízí několik alternativ čtení názvu tohoto algoritmu,
mezi nimi \uv{reet} [rýt] či \uv{ree-tee} [rýtý] \cite{doorenbos}) kontroluje
splnění podmínek inferenčních pravidel fakty znalostní baze. Použití naivního
přístupu, tedy zkoušení platnosti všech podmínek proti všem faktům při každé
změně znalostní baze není možné, neboť tato kontrola se již při několika stovkách
faktů či desítkách pravidel (závisí samozřejmě na výkonu výpočetního hardware, jeho
zatížení, atd.) stává příliš časově náročnou.

Algoritmus pracuje na základě dataflow sítě (či chcete-li sítě toku dat),
kterou průběžne konstruuje a upravuje při každém přidání či odebrání
inferenčního pravidla. Přepočet splněných pravidel algoritmem Rete je o~tolik
rychlejší díky tomu, že tato síť uchovává v~každém uzlu výsledky z~minulých
výpočtů a nově přidaný (či odebraný) fakt touto sítí jen velice rychle
\uv{proteče} a aktualizuje výsledky jen ve velmi malé části uzlů.

Dataflow síť algoritmu je z~pohledu teorie grafů spojitý orientovaný graf.
Máme-li dva uzly a orientovanou hranu, která je spojuje, budu uzel, z~nějž
hrana vede, označovat jako rodiče a uzel, do nějž vede, jako potomka.
Síť je tvořena dvěma částmi označovanými jako alfa a beta. Navštívení uzlu
sítě Rete algoritmu nazýváme aktivací. Některé uzly mohou být aktivovány
i více způsoby.

\begin{figure}[h]
\centerline{\epsfbox{network.eps}}
\caption{Dataflow síť Rete algoritmu\cite{rete}}
\label{network}
\end{figure}

Na obrázku \ref{network} je znázorněna dataflow síť Rete algoritmu.
Pokud by ovšem mělo znázornění odpovídat používané terminologii, měla
by být alfa část sítě na pravé straně. To si ale čtenář jistě snadno
představí.

\subsubsection{Alfa síť}
Alfa část sítě řeší konstantní porovnávání přidávaných faktů s~patterny
v~podmínkách pravidel, ignoruje tedy vazby proměnných a jejich konzistenci.
Jde v~podstatě o~strom, jehož každý uzel testuje právě jeden atom faktu.
Kořen alfa sítě je aktivován faktem, faktem aktivuje všechny své potomky.
Tyto už provádějí testy jednotlivých atomů, má-li atom požadovanou hodnotu,
předá uzel fakt svým potomkům a ty pokračují v~testování. Projde-li fakt
všemi konstatními testy, aktivuje poslední testovací uzel fakterm k~němu
přidružený paměťový alfa-uzel. Ten fakt uloží do své paměti a aktivuje
jím své následníky v~beta částí sítě.

Kořen alfa části sítě může ještě rozlišovat typ faktu (jde-li o~fakt
jednoduchý či složený, ve druhém případě pak typ šablony). A~podle toho
aktivovat faktem jen část potomků.

Pro lepší představu uvedu příklad. Mějme následující pravidlo:
\begin{verbatim}
(defrule move
  (goal :object ?x :from ?y)
  (in :object ?x :location ?y)
  (- in :object robot :location ?y)
  (in :object robot :location ?z)
  =>
  (modify (in :object robot :location ?z)
  (in :object robot :location ?y)))
\end{verbatim}
Kořen alfa sítě tedy bude mít dva potomky - pro šablony \verb|goal|
a \verb|in| (ignorujme prozatím, že třetí podmínka pravidla je negovaná,
to řeší až beta část dataflow sítě). Z~uzlu pro šablonu \verb|goal|
povede hrana rovnou do alfa-paměti, všechny zbylé atomy podmínky jsou
totiž proměnné. Z~uzlu pro šablonu povede také hrana do další alfa
paměti, neboť totéž platí pro druhou podmínku. Bude mít ale i dalšího
následníka a to testovací uzel, zda je slot \verb|object| roven hodnotě
\verb|robot|. Ten pak už aktivuje poslední paměťový alfa-uzel.

Máme tedy dva typové uzly (pro šablony \verb|goal| a \verb|in|, jeden
testovací uzel a tři uzly paměťové. To zatím není tak působivé,
uvědomme si ale, že definujeme-li další pravidlo, které bude mít
některé podmínky stejné jako výše uvedené, žádné uzly nám pro tyto
podmínky nepřibudou. A~budou-li podmínky alespoň podobné, přibudou
jen testovací uzly pro atomy patternu, které se budou lišit.

Zde je možno pozorovat první z~výhod oproti naivnímu přístupu. Totiž
máme-li pravidla se stejnými či podobnými podmínkami, velká část testů
se provádí jen jednou, což nám markantně snižuje časovou náročnost výpočtu.

\subsubsection{Beta síť}
Beta část sítě je opět tvořena několika typy uzlů. Kromě uzlů paměťových
jsou zde uzly, které testují konzistenci vazeb proměnných mezi jednotlivými
podmínkami inferenčního pravidla, uzly negativních testů a produkční uzly.

Paměťové beta-uzly už neuchovávají jednotlivá fakta, jako je tomu v~alfa
části dataflow sítě, nýbrž tzv. \emph{tokeny}, tedy skupiny faktů. Každý
paměťový uzel udržuje seznam tokenů, jimiž byly splněny dosavadní podmínky
inferenčního pravidla (tzv. \emph{partial match}). Máme-li tedy pravidlo
o~čtyřech podmínkách, budeme pro něj potřebovat čtyři paměťové beta-uzly.
První uzel uchovává jednomístné tokeny s~fakty, které prošly první podmínkou.
Neliší se tedy příliš od paměťového alfa-uzlu. Druhý už ale uchovává tokeny
s~dvojicí faktů - faktem, jenž prošel první podmínkou a faktem, který prošel
druhou. Takto se fakta v~tokenech nabalují, až máme konečně v~posledním uzlu
token stejné délky, jako je počet podmínek inferenčního pravidla. Tento uzel
označíme jako produkční - je-li aktivován tokenem, našli jsme kompletní
shodu faktů s~podmínkami pravidal (\emph{complete match}), v~tuto chvíli
algoritmus Rete ohlásí nalezení shody (dvojice pravidlo-token) prostředí
a ta je přidána do seznamu aktivací.

Hybnou silou beta síťě jsou spojovací uzly (tzv. \emph{join node}). Ty
testují konzistenci vazeb proměnných mezi fakty splňujícími jednotlivé
podmínky inferenčního pravidla. Jsou-li vazby konzistentní, uzel vytvoří
nový token a aktivuje jím přidružený paměťový beta-uzel. Spojovací uzel
může být aktivován dvěma způsoby - buď nově přidaným či ubraným
(a konstantními testy prošlým) faktem z~paměťového alfa-uzlu (tzv.
\emph{aktivace zprava}; toto je místo, kde beta část sítě navazuje na alfa
část), nebo tokenem z~nadřazeného paměťového beta-uzlu (\emph{aktivace
zleva}). V~obou případech ale spojovací uzel prohledá druhou paměť toho
druhého rodičovského uzlu pro získání konzistentních tokenů (resp. faktů)
a při nalezení shody aktivuje potomky.

Speciálním případem spojovacích uzlů jsou uzly s~negativními testy. Tyto
mohou být, stejně jako běžné spojovací uzly, aktivovány z~obou stran.
Podstatný rozdíl je ale v~tom, že negativní spojovací uzel aktivuje své
potomky jedině tehdy, nenalezl-li pro daný token žádný konzistentní fakt
ve paměti svého rodičovského alfa-uzlu.

Opět uvedu příklad. Zůstaňme u~pravidla z~příkladu předchozího. Alfa část
dataflow sítě pro uvedené pravidlo má tři paměťové uzly. V~beta části
budeme potřebovat tři spojovací uzly (z~nichž jeden bude negativní).
První bude testovat konzistenci vazeb mezi prvními dvěma podmínkami
pravidla, druhý mezi třetí podmínkou a prvními dvěma, atd. Druhý i třetí
spojovací uzel bude mít zprava připojen týž paměťový alfa-uzel (opět
recyklace uzlu, tentokrát dokonce v~rámci jednoho pravidla). Ke každému
spojovacímu uzlu je jako následník připojen paměťový beta-uzel, který
uchovává tokeny s~částečnými (v~prvních třech případech) či úplnými
(ve čtvrtém případě) shodami.

\begin{figure}[h]
\centerline{\epsfbox{condition-joining.eps}}
\caption{Spojovací uzly mezi jednotlivými podmínkami inferenčního pravidla\cite{doorenbos}}
\label{condition-joining}
\end{figure}

\subsubsection{Přidání/odebrání inferenčních pravidel}
Definuje-li uživatel nové inferenční pravidlo je třeba pro něj vytvořit
nové či pospojovat již existující uzly dataflow sítě. Algoritmus postupně
prochází podmínky pravidla, vytváří (či recykluje) pro každou paměťový
beta-uzel a spojovací beta-uzel, který napojí na následující paměťový beta-uzel.
Ke každému spojovacímu uzlu také vytvoří (či plně nebo částečně recykluje)
cestu v alfa části sítě. Poslední paměťový uzel označí jako produkční
a uloží v něm pravidlo aby mohlo být při splnění všech podmínek předáno
spolu s tokenem zpět prostředí a zařazeno do agendy.

Při odstranění pravidla ze systému je pravidlo odstraněno také z odpovídajícího
produkčního uzlu a všechny nyní zbytečné uzly jsou ze sítě odstraněny.

\subsection{Strategie pro výběr pravidla k aktivaci}
Expertní systém je velice obecný výpočetní mechanismus a proto není možné
určit jednu odvozovací strategii, která by pokrývala všechny domény problémů,
v nichž ho lze využít. Implementace proto poskytuje několik vestavěných
strategií pro výběr pravidla z agendy k následné aktivaci a umožňuje i
definici uživatelských strategií.
\begin{itemize}
\item {\bf{}Strategie nejnovější aktivace (\emph{depth strategy})} vybírá z agendy
pravidlo, které sem bylo přidáno jako poslední. Inferenční mechanismus tedy
provádí jakýsi výpočet do hloubky, spouští nově aktivovaná pravidla dokud to
lze a teprve při jejich vyčerpání se ve výpočtu vrací k pravidlům aktivovaným
dříve.
\item {\bf{}Strategie nejstarší aktivace (\emph{breadth strategy})} vybírá z agendy
pravidlo, které sem bylo přidáno jako první. Je tedy přesným opakem strategie
předchozí, provádí výpočet \uv{do šířky}.
\item {\bf{}Strategie nejobecnějšího pravidla (\emph{simplicity strategy})} vybírá
pravidlo s nejnižším počtem podmínek, tedy obecnější (statisticky čím méně je podmínek,
tím více najdeme shod, které pravidlo splní).
\item {\bf{}Strategie nejkonkrétnějšího pravidla (\emph{complexity strategy})} vybírá
pravidlo s nejvyšším počtem podmínek, tedy specifičtější.
\end{itemize}
-- Volně převzato z \cite{introduction}.
