%%%%%%%%%%%%%%%%%%%%%%%%%%%%%%%%%%%%%%%%%%%%%%%%%%%%%%%%%%%%%%%%%%%%%%%%%%%%%%%%
\subsection{Uživatelská příručka}
\subsubsection{Instalace}
- instalace - slime, sbcl, quicklisp, asdf, lispworks, získání kódu, git

Zdrojový kód knihovny je přiložen k~této diplomové práci a~lze jej také získat
zklonováním gitového\footnotemark{} repozitáře na adrese
\verb|git@github.com:Incanus3/ExiL.git| (v *nixových systémech např. zadáním
příkazu \verb|git clone git@github.com:Incanus3/ExiL.git|).
\footnotetext{http://git-scm.com/}

\subsubsection{Common Lisp}
- úvod do lispu - odkaz na practical common lisp, clhs

%%%%%%%%%%%%%%%%%%%%%%%%%%%%%%%%%%%%%%%%%%%%%%%%%%%%%%%%%%%%%%%%%%%%%%%%%%%%%%%%
\subsubsection{Struktura programu}
popsat jednotlivé sekce kódu, jejich význam (korespondence s fázemi návrhu ES,
formulace problému, formát dat, vstupní znalosti, odvozovací krok, řízení
odvozování, ladění)
\begin{itemize}
  \item definice prostředí
  \item definice šablon - formát dat
  \item definice znalostní báze - vstupní znalost - deffacts, defrules
  \item (nastavení sledování průběhu inference - watchers)
  \item (úprava průběhu inference - strategie)
  \item spuštění / krokování inference - reset, run, step
  \item dotazy nad working memory - facts, agenda
  \item úprava working memory - assert, retract, modify
  \item dotazy nad znalostní bází - fact-groups, rules
  \item cleanup - volatile vs durable sloty prostředí
  \item undo/redo
  \item zpětné řetězení
  \item GUI
\end{itemize}

\begin{listing}[H] % position specifier as for figures
\caption{ExiL code example}
\label{example}
\begin{clcode}
(deftemplate goal action object from to)
(deftemplate in object location)

(deffacts world
  (in :object robot :location A)
  (in :object box :location B)
  (goal :action push :object box :from B :to A))

(defrule move
  (goal :action push :object ?obj :from ?from)
  (in :object ?obj :location ?from)
  (- in :object robot :location ?from)
  ?robot <- (in :object robot :location ?)
  =>
  (modify ?robot :location ?from))

(defrule push
  (goal :action push :object ?obj :from ?from :to ?to)
  ?object <- (in :object ?obj :location ?from)
  ?robot <- (in :object robot :location ?from)
  =>
  (modify ?robot :location ?to)
  (modify ?object :location ?to))

(defrule stop
  ?goal <- (goal :action push :object ?obj :to ?to)
  (in :object ?obj :location ?to)
  =>
  (retract ?goal)
  (halt))

(reset)

; (step)
(run)
\end{clcode}
\end{listing}
