\subsection{Referenční příručka}
\begin{framed}
  \begin{itemize}
    \item reprezentace objektů vracené findery + názvy vracené makry jako rules či current
strategy - proč klíčky - různé package
    \item kombinace s funkčními alternativami.
    \item clipsová syntax - deftemplate, specifikace faktů, volání facts s
      čísly, singleton proměnná
  \end{itemize}
\end{framed}

ExiL umožňuje definici vlastní strategie makrem \verb|defstrategy|. Makro bere
jako parametry název strategie a výběrovou funkci. Výběrové funkci je předána
agenda (množina shod), z nichž funkce musí jednu vrátít. Definice vlastní
strategie ovšem není v praxi příliš použitelná, neboť pro její implementaci je
třeba znát vnitřní implementaci shody. Definice mohou vypadat například takto:
\begin{minted}[samepage]{cl}
(defmethod newer-than-p ((match1 match) (match2 match))
  (> (timestamp match1) (timestamp match2)))

(defmethod simpler-than-p ((rule1 rule) (rule2 rule))
  (< (length (conditions rule1))
     (length (conditions rule2))))
(defmethod simpler-than-p ((match1 match) (match2 match))
  (simpler-than-p (match-rule match1) (match-rule match2)))

(defstrategy breadth-stragegy #'newer-than-p)
(defstrategy simplicity-strategy #'simpler-than-p)
\end{minted}
Symboly názvů metod \verb|timestamp|, \verb|conditions| a \verb|match-rule| jsou
navíc interní v jiném package, než běžně uživatel pracuje, bylo by tedy třeba je
plně kvalifikovat.
