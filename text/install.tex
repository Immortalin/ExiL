%%%%%%%%%%%%%%%%%%%%%%%%%%%%%%%%%%%%%%%%%%%%%%%%%%%%%%%%%%%%%%%%%%%%%%%%%%%%%%%%
\subsection{Instalace}
\subsubsection{Získání zdrojového kódu}
Zdrojový kód knihovny je přiložen k~této diplomové práci a~lze jej také získat
zklonováním\footnote{\url{http://git-scm.com/docs/git-clone}}
gitového\footnote{\url{http://git-scm.com/}} repozitáře na adrese
\verb|git@github.com:Incanus3/ExiL.git|.
Kód knihovny se nachází v podadresáři \verb|src|, ten budu dále nazývat kořenovým
adresářem knihovny či projektu.
%%%%%%%%%%%%%%%%%%%%%%%%%%%%%%%%%%%%%%%%%%%%%%%%%%%%%%%%%%%%%%%%%%%%%%%%%%%%%%%%
\subsubsection{Prerekvizity}
Pro práci s knihovnou ExiL potřebujeme lispový
interpreter\footnote{\url{http://en.wikipedia.org/wiki/Interpreter_(computing)}}
\footnote{lispové interpretery jsou většinou zároveň
kompilátory\footnotemark[6], označením
interpreter tedy budu nazývat obojí},
vývojové prostředí (s interpreterem bychom si ve skutečnosti vystačili, ale
přímá práce s ním není většinou příliš pohodlná) a knihovny umožňující
dávkové načtení celého projektu včetně závislostí.
\footnotetext[6]{\url{http://en.wikipedia.org/wiki/Compiler}}

Knihovnu jsem vyvíjel v prostředí
SLIME\footnote{\url{http://www.common-lisp.net/project/slime/}}, což je plugin
pro textový editor GNU Emacs\footnote{\url{http://www.gnu.org/software/emacs/}}
(poskytující mimo jiné pomůcky pro editaci lispového zdrojového kódu,
REPL\footnote{\url{http://en.wikipedia.org/wiki/Read-eval-print_loop}} a
debugger\footnote{\url{http://en.wikipedia.org/wiki/Debugger}})
s interpreterem SBCL\footnote{\url{http://www.sbcl.org/}} a tuto kombinaci
mohu vřele doporučit. V operačním systému Debian GNU Linux, který jsem pro vývoj
použil, lze Emacs, SLIME i SBCL nainstalovat z výchozího repozitáře a aktivovat
úpravou inicializačního souboru Emacsu, viz
\url{http://www.common-lisp.net/projects/slime/doc/html/Installation.html}.
Prostředí poté můžeme v~Emacsu spustit voláním příkazu \verb|slime|
(\verb|M-x slime<enter>|). Při prvním spuštění se kód prostředí kompiluje, což
může chvíli trvat, pak už se v editoru otevře buffer\footnote{emacs buffer}
s~lispovým REPLem.

Knihovnu jsem testoval také ve vývojovém prostředí
LispWorks\registered\footnote{\url{http://www.lispworks.com/}} Personal Edition 6.1,
pro které jsem také vytvořil minimalistické grafické uživatelské rozhraní.
Součástí prostředí LispWorks je i lispový interpret. Prostředí můžeme nainstalovat
podle návodu zde \url{lispworks installation}.

Pro efektivní načtení knihovny včetně závislostí potřebujeme ještě dvě knihovny:
\begin{itemize}
  \item ASDF\footnote{\url{http://common-lisp.net/project/asdf}} je knihovna
    umožňující snadnou definici struktury projektu a jeho dávkové načtení,
  \item quicklisp\footnote{\url{http://www.quicklisp.org/beta/}} staví na knihovně
    ASDF a umožňuje pohodlně stáhnout a načíst knihovny třetích stran z internetové
    databáze.
\end{itemize}
Knihovna ASDF je součástí instalace interpreteru SBCL i prostředí LispWorks.
Knihovnu quicklisp jsem k projektu přiložil a pokud není součástí prostředí, je
automaticky načtena před načtením ExiLu.
%%%%%%%%%%%%%%%%%%%%%%%%%%%%%%%%%%%%%%%%%%%%%%%%%%%%%%%%%%%%%%%%%%%%%%%%%%%%%%%%
\subsubsection{Načtení knihovny}
V prostředí SLIME načteme knihovnu načtením souboru \verb|load.lisp| z kořenového
adresáře knihovny, tedy zadáním \cl|(load "cesta/k/projektu/src/load.lisp")|
v~REPLu). Tento soubor nejprve načte knihovnu \verb|quicklisp|, je-li potřeba,
a s její pomocí poté načte celý projekt ExiL včetně závislostí. Nakonec soubor
definuje výchozí prostředí, viz sekce \ref{multiple environments}

V prostředí LispWorks načítání pomocí knihovny \verb|quicklisp| nefunguje správně,
knihovnu je proto třeba načítat načtením souboru \verb|load-manual.lisp| (opět
z kořenového adresáře projektu). Načíst můžeme opět voláním \verb|load| v~REPLu,
nebo vybráním položky \verb|Load...| v nabídce \verb|File| menu libovolného okna
prostředí.

Všechna makra a funkce, které knihovna definuje pro přímé volání uživatelem jsou
\emph{exportována} z \emph{package} \verb|exil|. Před interakcí s knihovnou je
tedy třeba vstoupit do package \verb|exil-user|, který symboly z~package
\verb|exil| \emph{importuje}. Symboly z~package je také možno importovat do
existujícího package takto:
\begin{minted}{cl}
  (defpackage :my-package
    (:documentation "user-defined package")
    (:use :common-lisp :exil)
    (:shadowing-import-from :exil :assert :step))
\end{minted}
Package \verb|exil| exportuje několik symbolů, které již v package
\verb|common-lisp| existují. Ty je třeba \emph{zastínit}, jak je vidět z ukázky.
