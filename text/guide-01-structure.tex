\subsubsection{Struktura programu}

\begin{listing}[h]
\caption{Základní struktura exilového programu}
\label{typical structure}
\begin{clcode}
;;; definition of knowledge base
;; facts
(deffacts world
  (in box A)
  (in robot B)
  (goal move box A B))

;; inference rules
(defrule move-robot
  "move robot to object's location"
  (goal move ?object ?from ?to)
  (in ?object ?from)
  (- in robot ?from)
  (in robot ?z)
  =>
  (retract (in robot ?z))
  (assert (in robot ?from)))

(defrule move-object
  "move robot and object to desired location"
  (goal move ?object ?from ?to)
  ?rob-pos <- (in robot ?from)
  ?obj-pos <- (in ?object ?from)
  =>
  (retract ?rob-pos)
  (retract ?obj-pos)
  (assert (in robot ?to))
  (assert (in ?object ?to)))

(defrule stop
  "stop if object is in desired location"
  (goal move ?object ?from ?to)
  (in ?object ?to)
  =>
  (halt))

;;; initialization of working memory
(reset)

;;; inference execution
(run)
\end{clcode}
\end{listing}

Příklad \ref{typical structure} na straně \pageref{typical structure} ukazuje
minimální strukturu programu nad knihovnou ExiL (dále exilový program). První
část programu tvoří definice znalostní báze. Ta sestává z definic faktů, ze
kterých expertní systém vychází, a definic odvozovacích pravidel, jež jsou
následně aplikována při inferenci.

Definice faktů jsou uspořádány do skupin označených názvem (v tomto případě
\verb|world|). V ukázkovém programu si snadno vystačíme s jednou skupinou faktů,
v reálných programech bude ale těchto skupin většinou více. Tato organizace
umožňuje snadnou redefinici, případně odebrání, jen některých skupin faktů v
případě potřeby.  Definice skupiny faktů \verb|world| v příkladu přidává do
znalostní báze informaci o~počáteční pozici robota, krabice a~o~našem záměru
přesunout krabici z~pozice \verb|A| na pozici \verb|B|.

Následuje definice odvozovacích pravidel. Definice každého pravidla sestává z
jeho názvu, volitelného řetězce sloužícího k dokumentaci pravidla,
množiny podmínek, tedy předpokladů pro jeho splnění (a následnou aktivaci),
a~množiny důsledků, tedy libovolných lispových výrazů, které jsou při
aktivaci pravidla vyhodnoceny.  Tyto dvě množiny jsou od sebe odděleny
\emph{symbolem}~\verb|=>|.

Podmínky odvozovacích pravidel jsou ve formě vzorů (\emph{pattern}). Struktura
vzorů je stejná jako struktura faktů (viz sekce \ref{knowledge base
definition}), ale na rozdíl od nich mohou obsahovat proměnné (symboly začínající
otazníkem).  Při vyhodnocování podmínek pravidla je zajišťěna
konzistence vazeb těchto proměnných a výskyty všech proměnných v~důsledcích
pravidla jsou při jeho aktivaci nahrazeny jejich vazbami. Detaily viz sekce
\ref{inference}

Důsledky pravidel typicky obsahují příkazy pro modifikaci pracovní paměti (viz
sekce \ref{modifikace}), tedy přidání (\verb|assert|), odebrání
(\verb|retract|), či úpravu (\verb|modify|) faktů v ní. Nemusí tomu tak ale být
vždycky - důsledkem aktivace pravidla může být např. vypsání výstupu, logování,
zápis souboru, ale také např. ovládání externího systému.

Ukázkový příklad definuje tři odvozovací pravidla. Pravidlo
\verb|move-robot| je aktivováno, pokud chceme přesunout nějaký objekt z pozice
\verb|?from| na pozici \verb|?to|, objekt se nachází v pozici \verb|?from|
a~robot nikoli (třetí podmínka je negovaná, viz sekce \ref{inference}).
Poslední podmínka slouží pouze k navázání původní pozice robota.  Při aktivaci
pravidla je v pracovní paměti nahrazena informace o~původní pozici robota
pozicí \verb|?from|. Robot se tedy nyní nachází na stejné pozici, jako kýžený
objekt.

Podmínky pravidla \verb|move-object| vyžadují, aby byl jak robot, tak objekt
určený k přesunu, na pozici \verb|?from|. Při jeho aktivaci je robot i s objektem
přesunut na pozici \verb|?to| nahrazením faktů o původních pozicích novými,
podobně jako v~prvním pravidle. Definice pravidla obsahuje speciální notaci (s
použitím operátoru \verb|<-|), jejímž účelem je navázání celého faktu na
proměnnou. Ten pak můžeme v~důsledcích snadno ostranit z pracovní paměti.
Detaily opět viz sekce \ref{inference}

Poslední pravidlo slouží k zastavení inference, pokud se již objekt nachází na
cílové pozici. Inference je zde zastavena explicitním voláním \verb|(halt)|.
Druhou možností by bylo odstranit z pracovní paměti fakt definující cíl (jak
vidíme v~příkladu \ref{structured facts}), neboť v takovou chvíli nemůže být
žádné další pravidlo splňeno.

Jakmile je znalostní báze nadefinována, můžeme z ní inicializovat pracovní
paměť. To provedeme voláním \verb|(reset)|, které (po případném vyčištění
původních faktů) přidá do pracovní paměti fakty ve všech definovaných skupinách.

Poslední nutnou fází exilového programu je spuštění inference. To můžeme udělat
nejjednodušeji voláním \verb|(run)|. Inferenční mechanismus poté postupně
vyhodnocuje, která odvozovací pravidla mají splněné všechny podmínky, v každém
kroku z nich jedno vybere a aktivuje jej. Detaily viz sekce \ref{inference}

\clearpage
Výstup programu je následující:
\begin{minted}{cl}
==> (IN ROBOT B)
==> (IN BOX A)
==> (GOAL MOVE BOX A B)
Firing MOVE-ROBOT
<== (IN ROBOT B)
==> (IN ROBOT A)
Firing MOVE-OBJECT
<== (IN ROBOT A)
<== (IN BOX A)
==> (IN ROBOT B)
==> (IN BOX B)
Firing STOP
Halting
\end{minted}
Řádky začínající symbolem \verb|==>| označují fakty přidané do pracovní paměti,
řádky začínající \verb|<==| fakty z paměti odstraněné. Tento výstup obdržíme
pouze pokud zapneme sledování faktů voláním \verb|(watch facts)| (viz sekce
\ref{inference tracing}). První tři fakty přibydou do pracovní paměti při
vyhodnocení volání \verb|(reset)|, další pak spolu s~postupnou aplikací
odvozovacích pravidel. Dotážeme-li se po skončení inference na seznam faktů v
pracovní paměti voláním \verb|(facts)|, obdržíme výstup
\cl|((GOAL MOVE BOX A B) (IN ROBOT B) (IN BOX B)).|
Robot i krabice jsou tedy na cílové pozici.

Kód exilového programu má deklarativní charakter. Nikde jsme nemuseli
specifikovat, jakou posloupností akcí má systém k výsledku dospět. To nás ovšem
nezbavuje nutnosti chápat fungování inferenčního mechanismu ExiLu. Nebudeme-li
při konstrukci programu opatrní, může výpočet snadno dospět k neočekávaným
výsledkům, dostat se do slepé větve, či se zacyklit. Tyto problémy jsou často
způsobeny nezamýšlenou interferencí podmínek pravidel s důsledky jiných.

Skupina pravidel může například snadno cyklit, pokud první pravidlo závisí na
neexistenci nějakého faktu, tento v důsledcích přidá, načež je tento fakt
posledním pravidlem odstraněn, aniž by byla v průběhu zneplatněna nějaká další
podmínka prvního pravidla. Fakt, který odvodíme v jedné část programu může také
neočekávaně splnit podmínku pravidla definovaného v úplně jiné části, která s
první neměla vůbec přímo komunikovat.

% \FloatBarrier
