\section{Teoretická část}
\subsection{Expertní systémy}

\begin{framed}
  \begin{itemize}
    \item co je expertní systém - vlastnosti, distinkce
    \item popsáno v bakalářce:
      \begin{itemize}
        \item základní definice
        \item historie
        \item zařazení (knowledge-based, rule-based, AI)
        \item postup návrhu (volba reprezentace znalostí, proces učení -
          vytvoření znalostní báze)
        \item typické vlastnosti (dedukce, ne algebraické výpočty, oddělení
          znalosti od odvozovacího aparátu, heuristiky)
        \item základní rozdíl mezi dopředným a zpětným řetězením
      \end{itemize}

    \item zopakovat definici - počítačový program, který usuzuje nad nějakou
      problémovou doménou za účelem vyřešení zadaného problému (např. nalézt
      posloupnost akcí), nebo poskytnutí odpovědi na otázku (zda odpověď
      existuje, nalézt možné vazby proměnných)
    \item problémová doména, vyřešení problému, odpověď na otázku

    \item vyjmenovat, co už řeší bakalářka
    \item typickou vlastností je také schopnost vysvětlit, jak se došlo k
      řešení (to zde řeší watchers, zpětná inference řeší přímo) - motivace
      (str. 9)
    \item typické problémy řešené pomocí ES, aplikabilita
      \begin{itemize}
        \item plánování posloupnosti akcí
        \item interpretace dat (např. z nějakých senzorů), strukturální analýza
          (např. aplikace chemických analytických metod - když vypadá
          chromatograf takhle, pak látka asi obsahuje tohle)
        \item diagnóza poruch systému
        \item hledání konfugurace komplexních objektů - např. provisioning
          serverů pro pokrytí zadaného úkolu
      \end{itemize}
    \item aplikabilita, limitace
  \end{itemize}
\end{framed}

Expertní systém je počítačový program, který usuzuje nad nějakou problémovou
doménou za účelem vyřešení zadaného problému (např. nalézt posloupnost akcí),
nebo poskytnutí odpovědi na otázku (zda odpověď existuje, nalézt možné vazby
proměnných)


\subsection{Reprezentace znalosti}
\begin{framed}
  \begin{itemize}
    \item pracuje se znalostí - vlastnosti znalosti zpracovatelné ES
    \item symbolická reprezentace znalosti $\rightarrow$ symbolické výpočty
    \item reprezentace stavu pomocí symbolických struktur, inference jako jejich manipulace
    \item fakty a odvozovací pravidla
  \end{itemize}
\end{framed}

\subsection{Základní principy}
\begin{framed}
  \begin{itemize}
    \item rule-based systémy - základní pojmy - revisited s citacemi
    \item prohledávání prostoru stavů (stromy)
    \item combinatorial explosion $\rightarrow$ heuristiky (strategie)
  \end{itemize}
\end{framed}

\subsection{Dopředná a zpětná inference}
\begin{framed}
  \begin{itemize}
    \item dopředné vs. zpětné řetězení
    \item zpěnté - means-ends analysis
  \end{itemize}
\end{framed}

Expertní systémy rozdělujeme do dvou skupin podle toho, jakým způsobem nad
zadanými znalostmi usuzují. \textbf{Expertní systém se dopředným řetězením}
(inferencí) postupně vyvozuje závěry ze zadaných znalostí hledáním odvozovacích
pravidel se splněnými podmínkami. \textbf{Expertní systém se zpětným řetězením}
(inferencí) postupuje od zadaného cíle a hledá odvozovací pravidla, která vedou
k jeho splnění. Jejich podmínky pak zpracovává jako dílčí cíle.

Požadovaný cíl či dotaz může být součástí znalostí předaných systému (typické
pro systémy s dopřednou inferencí), nebo může být zadán samostatně (systémy se
zpětnou inferencí). Může nás také zajímat množina všech závěrů, odvoditelných ze
zadaných znalostí.

Zadáním cíle se systému dotazujeme, zda existuje posloupnost aplikace
odvozovacích pravidel, vedoucí k jeho splnění. V případě nale

Buď konkrétní cíl.
Cíl problému můžeme zadat systému jako dotaz (typické pro, ale ne omezené na,
zpětné řetězení).
Nebo nás může zajímat množina všech výsledků odvoditelných z báze.
poskytnutí odpovědi na otázku - zda lze ze znalostní báze odvodit fakt daného
tvaru (vzor), nalézt možné vazby proměnných

\subsection{Interpretace programu}
\begin{framed}
  \begin{itemize}
    \item interpretace výstupu programu
    \item plánování akcí
    \item hledání odpovědí
    \item dokazování vyplývání - sémantika podmínek
  \end{itemize}
\end{framed}

\subsection{Aplikace}
\begin{framed}
  \begin{itemize}
    \item strips, gps - zpětné řetězení
    \item mycin - složitější pravidla
    \item clips - dopředné řetězení
  \end{itemize}
\end{framed}

% \begin{framed}
%   \begin{itemize}
%     \item rozebrat, jak řeší řeší GPS v PoAIP problémy se zpětným řetězením
%     \item MYCIN, jakožo ES se zpětným řetězením, neřeší negativní znalost vůbec
%     \item co se týče kompozitních podmínek, používá MYCIN and-or stromy, šlo by
%       aplikovat v rete?
%     \item CLIPS basic programming guide - defrule construct - conflict resolution
%       strategies, LHS conditional elements
%   \end{itemize}
% \end{framed}
