%%%%%%%%%%%%%%%%%%%%%%%%%%%%%%%%%%%%%%%%%%%%%%%%%%%%%%%%%%%%%%%%%%%%%%%%%%%%%%%%
\subsubsection{Reset prostředí}
\label{env cleanup}

Entitu, která uchovává aktuální stav systému, nazývám v ExiLu prostředím.
Hodnoty tvořící aktuální stav prostředí rozděluji do dvou skupin. První skupinu
tvoří hodnoty trvalé. Sem spadají definice šablon, znalostní báze (skupiny
faktů, pravidla), definice strategií pro volbu shod a aktuálně zvolená
strategie.

Druhou skupinou jsou hodnoty dočasné. Sem patří aktuální stav pracovní paměti,
aktuální cíle, stav sítě RETE spolu s agendou udržující aktuální shody (splněná
pravidla spolu s vazbami proměnných), hodnoty zásobníků sloužících k vrácení
provedených akcí a jejich opětovné provedení (undo/redo) a~hodnota zásobníku
udržujícího aktuální stav zpětné inference (aby bylo možné ji krokovat nebo se
dotázat na alternativní odpovědi).

Dočasné hodnoty lze uvést do výchozího stavu voláním \verb|(clear)|. Tím systém
uvedeme do stavu, kdy zná pouze definované šablony, skupiny faktů a odvozovací
pravidla. Volání \verb|(reset)| provede \verb|(clear)| následovaný zavedením
všech skupin faktů do pracovní paměti. Všechny hodnoty prostředí, tedy včetně
trvalých, lze pak uvést do výchozího stavu voláním \verb|(complete-reset)|.

Chování \verb|reset| a \verb|clear| vzhledem k zásobníkům pro vracení akcí
může být poněkud překvapivé. Makro \verb|reset| totiž vymaže obsah tohoto
zásobníku, poté na něj však uloží akci k vrácení jeho volání. Z výpisu
\verb|(undo-stack)| tak po volání \verb|reset| zmizí předchozí akce, uvidíme zde
jen akci \verb|(reset)|. Po volání \verb|(undo)| se na zásobníku předchozí akce
opět objeví. Totéž platí pro volání \verb|(clear)|.
