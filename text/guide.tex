%%%%%%%%%%%%%%%%%%%%%%%%%%%%%%%%%%%%%%%%%%%%%%%%%%%%%%%%%%%%%%%%%%%%%%%%%%%%%%%%
\subsection{Uživatelská příručka}
\subsubsection{Základní pojmy}

Nyní stručně zadefinuji základní pojmy, nutné pro pochopení fungování knihovny
ExiL a práci s ní. Význam pojmů bude jasnější, jakmile si je ukážeme na
příkladech. K těmto pojmům se posléze vrátím i~v~teoretické části textu
a~jejich popis rozšířím o další souvislosti.

První dva pojmy staví na pojmu znalost, který chápeme intuitivně a nebudu se jej
ani snažit definovat, nikoli na následujícím pojmu znalosti, jak ji chápeme
v~ExiLu (v takovém případě by byla definice cyklická).

Pojem expertního systému zatím chápejme tak, jak jsem jej představil v úvodu
práce. V teoretické části rozeberu pojem v potřebné šíři.
\begin{description}[leftmargin=6cm,style=sameline,align=right,labelsep=0.5cm]
  % \item[problémová doména] množina pojmů relevantních pro řešení určité skupiny
  %   problémů
  \item[fakt] elementární statická znalost - tvrzení
  \item[(odvozovací) pravidlo] elementární odvozovací znalost - pokud víme, že
    (ne)platí nějaká tvrzení, můžeme odvodit, že platí i~nějaká další
  \item[znalost (v ExiLu)] množina faktů a pravidel
  \item[znalostní báze] výchozí znalost expertního systému
  \item[pracovní paměť] aktuální množina faktů
  % \item[production memory] aktuální množina pravidel
  \item[inference] odvozování - postupná aplikace odvozovacích pravidel
\end{description}
Pojem \emph{pracovní paměť} není příliš intuitivní. Jde o doslovný překlad
v~literatuře užívaného pojmu \emph{working memory}, kterým je označována množina
faktů (tvrzení), které expertní systém v danou chvíli považuje za platné. Nejde
tedy ve skutečnosti o paměť, nýbrž o obsah pomyslné paměti. Pojem pracovní
množina faktů by byl jistě výstižnější, bohužel ale také značně těžkopádný.

%%%%%%%%%%%%%%%%%%%%%%%%%%%%%%%%%%%%%%%%%%%%%%%%%%%%%%%%%%%%%%%%%%%%%%%%%%%%%%%%
\subsubsection{Struktura programu}

\begin{listing}[h]
\caption{Základní struktura exilového programu}
\label{typical structure}
\begin{clcode}
;;; definition of knowledge base
;; facts
(deffacts world
  (in box A)
  (in robot B)
  (goal move box A B))

;; inference rules
(defrule move-robot
  (goal move ?object ?from ?to)
  (in ?object ?from)
  (- in robot ?from)
  (in robot ?z)
  =>
  (retract (in robot ?z))
  (assert (in robot ?from)))

(defrule move-object
  (goal move ?object ?from ?to)
  ?rob-pos <- (in robot ?from)
  ?obj-pos <- (in ?object ?from)
  =>
  (retract ?rob-pos)
  (retract ?obj-pos)
  (assert (in robot ?to))
  (assert (in ?object ?to)))

(defrule stop
  (goal move ?object ?from ?to)
  (in ?object ?to)
  =>
  (halt))

;;; initialization of working memory
(reset)

;;; inference execution
(run)
\end{clcode}
\end{listing}

Příklad \ref{typical structure} na straně \pageref{typical structure} ukazuje
minimální strukturu programu nad knihovnou ExiL (dále exilový program). První
část programu tvoří definice znalostní báze. Ta sestává z definic faktů, ze
kterých expertní systém vychází a definic odvozovacích pravidel, jež jsou
následně aplikována při inferenci.

Definice faktů jsou uspořádány do skupin označených názvem (v tomto případě
\verb|world|). V ukázkovém programu si snadno vystačíme s jednou skupinou faktů,
v reálných programech bude ale těchto skupin většinou více. Tato organizace
umožňuje snadnou redefinici, případně odebrání, jen některých skupin faktů v
případě potřeby.  Definice skupiny faktů \verb|world| v příkladu přidává do
znalostní báze informaci o~počáteční pozici robota, krabice a~o~našem záměru
přesunout krabici z~pozice \verb|A| na pozici \verb|B|.

Následuje definice odvozovacích pravidel. Definice každého pravidla sestává
z~množiny podmínek, tedy předpokladů pro jeho splnění (a následnou aktivaci),
a~množiny důsledků, tedy libovolných lispových lispových výrazů, které jsou při
aktivaci pravidla vyhodnoceny.  Tyto dvě množiny jsou od sebe odděleny
\emph{symbolem}~\verb|=>|.

Podmínky odvozovacích pravidel jsou ve formě vzorů (\emph{pattern}). Struktura
vzorů je stejná jako struktura faktů (viz sekce \ref{knowledge base
definition}), ale narozdíl od nich mohou obsahovat proměnné (symboly začínající
otazníkem).  Při vyhodnocování podmínek pravidla je zajišťěna
konzistence vazeb těchto proměnných a výskyty všech proměnných v~důsledcích
pravidla jsou při jeho aktivaci nahrazeny jejich vazbami. Detaily viz sekce
\ref{inference}

Důsledky pravidel typicky obsahují příkazy pro modifikaci pracovní paměti (viz
sekce \ref{modifikace}), tedy přidání (\verb|assert|), odebrání (\verb|retract|),
či úpravů (\verb|modify|) faktů v ní. Nemůsí
tomu tak ale být vždycky - důsledkem aktivace pravidla může být např. vypsání
výstupu, logování, zápis souboru, ale také např. ovládání externího systému.

Ukázkový příklad definuje tři odvozovací pravidla. Pravidlo
\verb|move-robot| je aktivováno, pokud chceme přesunout nějaký objekt z pozice
\verb|?from| na pozici \verb|?to|, objekt se nachází v pozici \verb|?from|
a~robot nikoli (třetí podmínka je negovaná, viz sekce \ref{inference}).
Poslední podmínka slouží pouze k navázání původní pozice robota.  Při aktivaci
pravidla je v pracovní paměti nahrazena informace o~původní pozici robota
pozicí \verb|?from|. Robot se tedy nyní nachází na stejné pozici, jako kýžený
objekt.

Podmínky pravidla \verb|move-object| vyžadují, aby byl jak robot, tak objekt
určený k přesunu, na pozici \verb|?from|. Při jeho aktivaci je robot i s objektem
přesunut na pozici \verb|?to| nahrazením faktů o původních pozicích novými,
podobně jako v prvním pravidle. Definice pravidla obsahuje speciální notaci (s
použitím operátoru \verb|<-|), jejímž účelem je navázání celého faktu na
proměnnou. Ten pak můžeme v důsledcích snadno ostranit z pracovní paměti.
Detaily opět viz sekce \ref{inference}

Poslední pravidlo slouží k zastavení inference, pokud se již objekt nachází na
cílové pozici. Inference je zde zastavena explicitním voláním \verb|(halt)|.
Druhou možností by bylo odstranit z pracovní paměti fakt definující cíl, neboť v
takovou chvíli nemůže být žádné další pravidlo splňeno.

Jakmile je znalostní báze nadefinována, můžeme z ní inicializovat pracovní
paměť. To provedeme voláním \verb|(reset)|, které (po případném vyčištění
původních faktů) přidá do pracovní paměti fakty ve všech definovaných skupinách.

Poslední nutnou fází exilového programu je spuštění inference. To můžeme udělat
nejjednodušeji voláním \verb|(run)|. Inferenční mechanismus poté postupně
vyhodnocuje, která odvozovací pravidla mají splněné všechny podmínky, v každém
kroku z nich jedno vybere a aktivuje jej. Detaily viz sekce \ref{inference}

Výstup programu je následující:
\begin{minted}{cl}
==> (IN ROBOT B)
==> (IN BOX A)
==> (GOAL MOVE BOX A B)
Firing MOVE-ROBOT
<== (IN ROBOT B)
==> (IN ROBOT A)
Firing MOVE-OBJECT
<== (IN ROBOT A)
<== (IN BOX A)
==> (IN ROBOT B)
==> (IN BOX B)
Firing STOP
Halting
\end{minted}
Řádky začínající symbolem \verb|==>| označují fakty přibyvší do pracovní paměti,
řádky začínající \verb|<==| fakty z paměti odstraněné. Tento výstup obdržíme
pouze pokud zapneme sledování faktů voláním \verb|(watch facts)| (viz sekce
\ref{inference tracing}). První tři fakty přibydou do pracovní paměti při
vyhodnocení volání \verb|(reset)|, další pak spolu s postupnou aplikací
odvozovacích pravidel. Dotážeme-li se po skončení inference na seznam faktů v
pracovní paměti voláním \verb|(facts)|, obdržíme výstup
\cl|((GOAL MOVE BOX A B) (IN ROBOT B) (IN BOX B)).|
Robot i krabice jsou tedy na cílové pozici.

Kód exilového programu má deklarativní charakter. Nikde jsme nemuseli
specifikovat, jakou posloupností akcí má systém k výsledku dospět. To nás ovšem
nezbavuje nutnosti chápat fungování inferenčního mechanismu ExiLu. Nebudeme-li
při konstrukci programu opatrní, může výpočet snadno dospět k neočekávaným
výsledkům, dostat se do slepé větve, či se zacyklit. Tyto problémy jsou často
způsobeny nezamýšlenou interferencí podmínek pravidel s důsledky jiných.

\FloatBarrier

\subsubsection{Definice znalostní báze}
\label{knowledge base definition}

ExiL, stejně jako CLIPS, rozlišuje dva typy faktů - jednoduché (\emph{simple,
ordered}) a strukturované (\emph{templated}). Stuktura jednoduchého faktu je udána
pouze pořadím \emph{atomů}, typickou volbou je např. \verb|objekt-attribut-hodnota|:
\cl|(box color red),| či \verb|relace-objekty|: \cl|(in box hall).|

Strukturované fakty mají naproti tomu explicitně pojmenované složky (sloty).
Typicky popisují objekt s množinou pojmenovaných atributů: \cl|(box :color red :size small),|
či relaci s pojmenovanými aktory: \cl|(in :object box :location hall),| kde
\verb|box| a \verb|in| jsou šablony (\emph{template}), které je třeba definovat
předem. Na pořadí specifikace slotů u strukturovaných faktů
nezáleží.

Vyjadřovací síla obou typů faktů je stejná, použitím explicitnějších
strukturovaných faktů ale docílíme lepší čitelnosti a jednoznačnější sémantiky
exilového programu, zláště třeba v případě relací na jedné množině objektů:
\cl|(father john george).|

Šablonu definujeme voláním \emph{makra} \verb|deftemplate|, např:
\cl|(deftemplate in object (location :default here)).| Prvním parametrem je
název šablony, za ním následuje libovolný počet specifikací slotů. Specifikací
slotu je buď symbol - jméno slotu, nebo \emph{seznam}, jehož hlavou
(\emph{car}) je jméno slotu a~tělem (\emph{cdr}) je \emph{property list (plist)}
s dalšími parametry. Aktuálně systém umožňuje pouze specifikaci výchozí hodnoty
slotu \emph{klíčem} \verb|:default|. Ta je použita, není-li při specifikaci
faktu, používajícího tuto šablonu, uvedena hodnota pro daný slot.

Je-li už šablona požadovaného názvu definována, ale neexistují v pracovní
paměti fakty, které ji používají, je její stávající definice nahrazena. Pokud
ale v pracovní paměti existují takové fakty, skončí volání \verb|deftemplate|
výjimkou.

Seznam názvů všech definovaných šablon můžeme získat voláním \emph{funkce}
\verb|templates| (bez parametrů). Specifikaci šablony pak získáme voláním makra
\verb|find-template|, např. \verb|(find-template in).| Definici šablony zrušíme
voláním makra \verb|undeftemplate|, např. \verb|(undeftemplate goal)|. To opět
skončí výjimkou, existují-li v pracovní paměti fakty, které šablonu využívají.

Fakty, ze kterých expertní systém vychází, zavádíme pomocí skupin faktů. Ty
definujeme makrem \verb|deffacts|, např.:
\begin{minted}{cl}
(deffacts initial
  (goal move box A B)
  (in :object box :location A))
\end{minted}
Prvním parametrem je název skupiny, pak následuje libovolný počet specifikací
faktů. Opakovaným voláním makra \verb|deffacts| je skupina faktů
redefinována.

Specifikace faktu je vždy tvořena seznamem. Pokud jde o jednoduchý fakt,
specifikací je prostě seznam atomů. Jde-li o fakt strukturovaný, je prvním
prvkem specifikace název šablony, za ním následuje plist určující hodnoty slotů
faktu. Pokud není hodnota některého slotu uvedena, je buď použita
výchozí hodnota, pokud byla v šabloně specifikována, nebo hodnota \verb|nil| v
opačném případě.

Seznam názvů všech definovaných skupin faktů získáme voláním funkce
\verb|fact-groups| (bez parametrů). Specifikaci skupiny pak voláním makra
\verb|find-fact-group|, např.: \verb|(find-fact-group initial)|. Ke zrušení
definice skupiny slouží makro \verb|undeffacts| (voláme s názvem skupiny).

Pravidla, pomocí nichž expertní systém během inference odvozuje nové fakty,
definujeme makrem
\verb|defrule|, např.:
\begin{minted}{cl}
(defrule move-robot
  (goal :action move :object ?obj :from ?from)
  (in :object ?obj :location ?from)
  (- in :object robot :location ?from)
  ?robot <- (in :object robot :location ?)
  =>
  (modify ?robot :location ?from)).
\end{minted}
Podmínková část pravidla (před symbolem \verb|=>|) je tvořena vzory. Ty mohou
být, stejně jako fakty, jednoduché, nebo strukturované. Kromě toho umožňuje
definice pravidla několik speciálních konstruktů (negace podmínky, navázání
proměnné na celou podmínku). Ty popíšu podrobně v sekci \ref{inference} spolu~s
tím, jak jsou podmínky pravidla při inferenci vyhodnocovány.

Důsledkovou část pravidla (za symbolem \verb|=>|) tvoří libovolný počet
lispových výrazů. Jak se tyto vyhodnocují popíšu opět v sekci \ref{inference}.

Opakovaným voláním makra \verb|defrule| odvozovací pravidlo redefinujeme.
K~získání seznamu názvů definovaných pravidel a jejich specifikací slouží funkce
\verb|rules| a makro \verb|find-rule|, podobně jako u šablon a skupin faktů.
Ke zrušení definice pravidla slouží makro \verb|undefrule|.

\subsubsection{Modifikace pracovní paměti}
\label{modifikace}

Pracovní paměť je množina faktů, které systém v danou chvíli považuje za platné.
Její obsah můžeme vypsat voláním funkce \verb|facts|. Funkcí \verb|reset|
inicializujeme pracovní paměť ze znalostní báze. Jejím voláním jsou do pracovní
paměti zavedeny fakty všech definovaných skupin faktů (viz sekce
\ref{knowledge base definition}).

Obsah pracovní paměťi může být dále modifikován třemi makry:
\begin{itemize}
  \item \verb|assert| přidává fakt(y) do pracovní paměti,
  \item \verb|retract| fakt(y) z pracovní paměti odebírá a
  \item \verb|modify| přímo modifikuje existující fakty.
\end{itemize}
Ta lze volat buď před započetím inference (ale po volání \verb|reset|, neboť to
dodatečné úpravy vymaže), nebo v jejím průběhu, pokud inferenci krokujeme (viz
sekce \ref{inference}). Makra také typicky voláme v důsledcích pravidel.

Makra \verb|assert| a \verb|retract| berou jako parametry libovolný počet
specifikací faktů ve stejném formátu, jako u makra \verb|deffacts| (ale bez
názvu skupiny). Makro \verb|modify| lze použít jen u strukturovaných faktů. Toto
makro bere jako první parametr specifikaci faktu, zbytek parametrů tvoří plist
určující hodnoty slotů ke změně. Např.
\begin{minted}{cl}
(modify (in :object box :location A) :location B)
\end{minted}
nahradí v pracovní paměti fakt \verb|(in :object box :location A)| faktem
\verb|(in :object box :location B)|.
Toto makro je obzvláště užitečné, navážeme-li v podmínkách pravidla celý fakt na
proměnnou (viz sekce \ref{inference}).

Všechny fakty můžeme z pracovní paměti odebrat voláním funkce \verb|retract-all|
(bez parametrů). To je ale zřídka užitečné, typicky použijeme spíše funkci
\verb|reset| pro navrácení pracovní paměti do výchozího stavu.

\subsubsection{Inference}
\label{inference}

Inference (odvozování nových faktů z aktuálních) probíhá v krocích. V každém
kroku jsou vyhodnoceny podmínky všech pravidel, načež je ze splněných pravidel
vybráno jedno, které je posléze aktivováno. Inferenční kroky
můžeme buď spouštět jednotlivě voláním funkce \verb|step| (bez parametrů), nebo
voláním funkce \verb|run| spustit cyklus, který provádí inferenční kroky, dokud
je to možné. Cyklus je buď přerušen ve chvíli, kdy není splněno žádné
další pravidlo, nebo voláním funkce \verb|halt| (bez parametrů) v důsledcích
právě aktivovaného pravidla.

Podmínky pravidel jsou ve tvaru vzorů a jsou spojeny logickou konjunkcí,
pravidlo je tedy splněno, jsou-li splněny všechny jeho podmínky. Kromě
toho mohou být některé podmínky negovány. Taková podmínka je splněna tehdy,
neexistuje-li v pracovní paměti žádný fakt, který by se shodoval s jejím vzorem.
Negovanou podmínku značí znak \verb|-| (minus) na prvním místě specifikace vzoru.

Vyhodnocování podmínek pravidel probíhá ve dvou fázích. V první fází
srovnáváme vzory jednotlivých podmínek se všemi fakty v pracovní paměti a~to
pouze strukturálně, tedy bez ohledu na vazby proměnných. Prvním požadavkem shody
je u jednoduchých faktů stejná délka faktu (počet atomů), u strukturovaných
faktů stejná šablona. Jednoduchý fakt se nikdy nemůže shodovat se strukturovaným
vzorem a naopak.

Dále jsou pak porovnávány jednotlivé atomy (u jednoduchých) či
sloty (u složených) faktu vůči odpovídajícímu atomu (slotu) vzoru. Není-li atom
(slot) vzoru proměnná, je jednoduše porovnán s atomem faktu. Je-li atomem
proměnná, považujeme jej v této fázi automaticky za shodu. Např. vzor
\cl|(in :object robot :location ?loc)|
se shoduje s faktem
\cl|(in :object robot :location A)|
nikoli však s fakty
\begin{minted}{cl}
(in :object box :location A)
(is-in :object robot :location A)
(in robot A).
\end{minted}
Tímto předvýběrem
tedy získáme ke každé podmínce pravidla množinu faktů, které mají stejnou
strukturu a stejné hodnoty neproměnných atomů.

Ve druhé fázi vyhodnocování hledáme z předvybraných faktů takovou posloupnost
(délka odpovídá počtu podmínek pravidla), kde po spárování s odpovídajícími
vzory podmínek obdržíme konzistentní vazby proměnných. To znamená, že
vyskytuje-li se v podmínkách pravila některá proměnná vícekrát, musí mít
odpovídající fakty na daných pozicích stejný atom.  Mějme např. pravidlo s
podmínkami
\begin{minted}{cl}
(goal move ?obj ?from ?to)
(in :object ?obj  :location ?from)
(in :object robot :location ?to)
\end{minted}
Vzor první podmínky je jednoduchý, zatímco další dva jsou strukturované. To ale
ničemu nevadí, je třeba pouze najít fakty odpovídající struktury. Posloupnost
faktů
\begin{minted}{cl}
(goal move box A B)
(in :object box   :location B)
(in :object robot :location A)
\end{minted}
neprojde druhou fází výběru, neboť vazby proměnných nejsou konzistentní.
Proměnná \verb|?from| je např. v první podmínce navázána na symbol \verb|A|, v
druhé ale na \verb|B|. Kdyby si ovšem krabice s robotem vyměnily pozice, budou
vazby proměnných konzistentní a podmínky pravidla budou splněny. Proměnná
\verb|?from| by pak nabyla hodnoty \verb|A|, proměnná \verb|?to| hodnoty
\verb|B| a proměnná \verb|?obj| hodnoty \verb|box|.

Vyhodnocení negovaných podmínek si můžeme představit tak, že nejprve vyhodnotíme
a navážeme proměnné všech ostatních podmínek. Pokud poté neexistuje fakt, který
by se s vzorem negované podmínky shodoval a měl konzistentní vazby se zbytkem navázaných
proměnných, je tato podmínka splněna. Mějme například pravidlo s podmínkami
\begin{minted}{cl}
(goal move box ?from ?to)
(in box ?from)
(- in robot ?from).
\end{minted}
Máme-li v pracovní paměti pouze fakty
\begin{minted}{cl}
(goal move box A B)
(in box A)
(in robot B),
\end{minted}
budou podmínky pravidla splněny, neboť po spárování vzorů prvních dvou podmínek
s prvními dvěma fakty bude proměnná \verb|?from| navázána na hodnotu \verb|A| a
neexistuje fakt, který by se shodoval s vzorem \verb|(in robot A)|. Přesuneme-li
ale robota na pozici \verb|A|, podmínka již splněna nebude a pravidlo nelze
aktivovat.

Ve vzorech podmínek pravidla můžeme využít speciální proměnné \verb|?|.
Konzistence vazby této proměnné není při vyhodnocování testována, takže
vyskytuje-li se tato proměnná na více místech, chová se tak, jako kdyby byl
každý výskyt označen unikátním názvem (podobně jako proměnná \verb|_| v
Prologu). Použitím této proměnné dáváme najevo, že nás konkrétní hodnota daného
atomu nazajímá. Ve strukturovaných podmínkách není třeba tyto sloty uvádět,
neboť \verb|?| je výchozí hodnotou slotu vzoru.

Posledním speciálním konstruktem je navázání celého faktu na proměnnou. Např.
pravidlo
\begin{minted}{cl}
(defrule move
  ?fact <- (in :object ? :location A)
  =>
  (modify ?fact :location B))
\end{minted}
přesune každý objekt z pozice \verb|A| na pozici \verb|B|. Na proměnnou
můžeme navázat i jednoduchý fakt, pak ale nemůžeme použít makra \verb|modify|.
Můžeme ovšem volat \verb|(retract ?fact)|, neboť proměnná \verb|?fact| je při
aktivaci pravidla nahrazena specifikací faktu, který byl s vzorem podmínky
spárován.

\begin{framed}
  Pravidlo může být splněno několika posloupnostmi faktů - pojem matche
\end{framed}

Je-li při vyhodnocování podmínek nalezeno více splněných pravidel, je třeba z
nich jedno vybrat k aktivaci. Výběr pravidla záleží na zvolené strategii.
ExiL poskytuje následující strategie výběru pravidla:
\begin{description}[leftmargin=6cm,style=sameline,align=right,labelsep=0.5cm]
  \item[depth-strategy]
\end{description}

\begin{framed}
  \begin{itemize}
    \item fáze podrobně
    \begin{itemize}
      \item výběr pravidla - strategie
      \item aktivace - vyhodnocení důsledků (typicky modifikace pracovní paměti)
        navázání proměnných, eval
    \end{itemize}
    \item spuštění inference, krokování (může se prolínat s ručnímodifikací w.m.)
    \item queries - agenda, strategies
  \end{itemize}
\end{framed}

%%%%%%%%%%%%%%%%%%%%%%%%%%%%%%%%%%%%%%%%%%%%%%%%%%%%%%%%%%%%%%%%%%%%%%%%%%%%%%%%
\subsubsection{Sledování průběhu inference}
ExiL umožnuje sledovat několik typů událostí, ke kterým dochází během inference.
K nastavení sledovaných událostí slouží makra \verb|watch| a \verb|unwatch|. K
zjištění stavu sledování pak makro \verb|watchedp|.

Základním výstupem programu \ref{typical structure} na straně \pageref{typical
structure} je
\begin{minted}{cl}
Firing MOVE
Firing PUSH
Firing STOP
Halting.
\end{minted}
Zapneme-li sledování faktů voláním \verb|(watch facts)|, obdržíme výstup
\begin{minted}{cl}
==> (IN BOX A)
==> (IN ROBOT B)
==> (GOAL MOVE BOX A B)
Firing MOVE-ROBOT
<== (IN ROBOT B)
==> (IN ROBOT A)
Firing MOVE-OBJECT
<== (IN ROBOT A)
<== (IN BOX A)
==> (IN ROBOT B)
==> (IN BOX B)
Firing STOP
Halting.
\end{minted}

Sledování pravidel (\verb|(watch rules)|), přidává informace o pravidel
přidaných do (odebraných ze) znalostní báze, např.
\begin{minted}{cl}
==> (RULE STOP
  (GOAL MOVE ?OBJECT ?FROM ?TO)
  (IN ?OBJECT ?TO)
  =>
  (HALT)).
\end{minted}

Po zapnutí sledování agendy (voláním \verb|(watch activations)|, název je kvůli
kompatibilitě se systémem CLIPS) budeme navíc informováni o shodách, které do
agendy přibyly, nebo z ní byly odstraněny. Výstup programu pak bude následující:
\begin{minted}[samepage]{cl}
==> (MATCH MOVE-ROBOT ((GOAL MOVE BOX A B) (IN BOX A) (IN ROBOT B)))
Firing MOVE-ROBOT
==> (MATCH MOVE-OBJECT ((GOAL MOVE BOX A B) (IN BOX A) (IN ROBOT A)))
==> (MATCH MOVE-ROBOT ((GOAL MOVE BOX A B) (IN BOX A) (IN ROBOT A)))
<== (MATCH MOVE-ROBOT ((GOAL MOVE BOX A B) (IN BOX A) (IN ROBOT A)))
Firing MOVE-OBJECT
==> (MATCH STOP ((GOAL MOVE BOX A B) (IN BOX B)))
Firing STOP
Halting.
\end{minted}
Každá shoda je zde reprezentována názvem pravidla a posloupností faktů, které
byly spárovány s jeho podmínkami. Odtud můžeme snadno odvodit substituci, jež
byla při vyhodnocení použita. Negované podmínky nejsou spárovány s žádným
konkréním faktem, proto jsou zde jen tři fakty, přestože pravidlo
\verb|move-robot| má podmínky čtyři.

Je zde také vidět, že po aktivaci pravidla
\verb|move-robot| se v agendě na chvíli objeví opětovná shoda tohoto pravidla.
To je způsobeno tím, že obsah agendy se přepočítává po každé změně pracovní
paměti, takže se zde mohou objevit dočasné výsledky.

%%%%%%%%%%%%%%%%%%%%%%%%%%%%%%%%%%%%%%%%%%%%%%%%%%%%%%%%%%%%%%%%%%%%%%%%%%%%%%%%
\subsubsection{Undo/redo}

Jedním z implementovaných rozšíření původního programu je schopnost vrácení
provedených změn. K tomu slouží makra \verb|undo| a \verb|redo|. Ta lze použít k
vrácení jakékoli akce s vedlejším efektem, včetně kroků inference. K vypsání
zásobníků s~akcemi, které je možné vrátit, jsou k dispozici makra
\verb|undo-stack| a \verb|redo-stack|.

Pokud například vyhodnotíme prvních 32 řádků programu \ref{typical structure} na
straně \pageref{typical structure} a~zavoláme dvakrát \verb|undo|, bude výpis
zásobníků následující (přeformátováno):
\begin{minted}[samepage]{cl}
EXIL-USER> (undo-stack)
  1: (defrule MOVE-ROBOT
       ((GOAL MOVE ?OBJECT ?FROM ?TO)
        (IN ?OBJECT ?FROM)
        (- IN ROBOT ?FROM) (IN ROBOT ?Z)
        =>
        (RETRACT (IN ROBOT ?Z))
        (ASSERT (IN ROBOT ?FROM))))
  2: (deffacts WORLD
       ((IN BOX A) (IN ROBOT B) (GOAL MOVE BOX A B)))
\end{minted}
\begin{minted}[samepage]{cl}
EXIL-USER> (redo-stack)
  1: (defrule MOVE-OBJECT
       ((GOAL MOVE ?OBJECT ?FROM ?TO)
        ?OBJ-POS <- (IN ?OBJECT ?FROM)
        ?ROB-POS <- (IN ROBOT ?FROM)
        =>
        (RETRACT ?ROB-POS)
        (RETRACT ?OBJ-POS)
        (ASSERT (IN ROBOT ?TO))
        (ASSERT (IN ?OBJECT ?TO))))
  2: (defrule STOP
       ((GOAL MOVE ?OBJECT ?FROM ?TO)
        (IN ?OBJECT ?TO)
        =>
        (HALT)))
\end{minted}
Vidíme tedy, že jsme vrátili zpět definice pravidel \verb|move-object| a
\verb|stop| (ty bychom mohli opět provést voláním \verb|(redo)|). Dalším voláním
\verb|(undo)| by pak byla vrácena definice pravidla \verb|move-robot| a poté
definice skupiny faktů \verb|world|.

Nemá-li akce žádný vedlejší efekt - např. volání \verb|assert| s faktem, který
už v~pracovní paměti je, či volání \verb|run| ve chvíli, kdy už není co
odvozovat - prázdná akce se na zásobník neuloží.

\begin{framed}
zvážit, zda popis chování undo + reset patří sem, nebo do resetu prostředí
\end{framed}

%%%%%%%%%%%%%%%%%%%%%%%%%%%%%%%%%%%%%%%%%%%%%%%%%%%%%%%%%%%%%%%%%%%%%%%%%%%%%%%%
\subsubsection{Zpětné řetězení}
\begin{framed}
  \begin{itemize}
    \item cíle jako patterny
    \item základní inference - nejdřív fakty, pak pravidla, v jakém pořadí
      vybírá
    \item alternativní odpovědi - backtracking
  \end{itemize}
\end{framed}

Dalším z implementovaných rozšíření je možnost zpětné inference. Inference
popsaná v sekci \ref{inference} je dopředná. V každém kroku jsou nalezeny
všechny možnosti dalšího postupu odvozování, načež je zvolena jedna, kterou se
program dále ubírá. To činí průběh inference značně nedeterministickým. Možnosti
postupu, které nebyly vybrány, mohou být navíc dalším postupem ztraceny, pokud
aplikace některého pravidla zneplatní podmínky jiného.
Míru nedeterminismu můžeme snížit tím, že budeme navrhovat odvozovací pravidla
tak, aby se výpočet neubíral nechtěnými cestami. To ale není vždy jednoduché,
nebo dokonce možné.

Zpětná inference naproti tomu umožňuje definovat cíle, kterých chceme dosáhnout.
K tomu slouží makro \verb|defgoal|, kterému předáme vzor ve stejném formátu,
jako u podmínek pravidel. Definice cíle ovšem nepodporuje negaci ani navázání
faktu na proměnnou (k tomu ani není důvod).

Ke spuštění zpětné inference pak slouží funkce \verb|back-step| a
\verb|back-run|, podobně jako u inference dopředné.

Mějme následující znalostní bázi:
\begin{minted}{cl}
(deffacts world
  (have-money))

(defrule buy-car
  (have-money)
  =>
  (retract (have-money))
  (assert (have-car)))

(defrule pay-rent
  (have-money)
  =>
  (retract (have-money))
  (assert (rent-payed))).
\end{minted}
Spustíme-li dopřednou inferenci, systém nám vesele doporučí nákup auta.
Mít nové auto je sice pěkné, hrozí-li nám ale vyhození z pronajatého bytu,
není nákup auta pravděpodobně cestou, kterou bychom se chtěli ubírat.
Systém by nám v tuto chvíli mohl stejně dobře doporučit správnou cestu. Že bylo
vybráno zrovna první pravidlo je výsledkem toho, jak funguje síť RETE, která
pravidla vyhodnocuje. Za daných okolností ale nechceme špatnou variantu ani
připouštět.

V tomto případě bychom mohli upravit definici programu tak, že do znalostní báze
přidáme informaci o cíli, kterou budou pravidla zohledňovat, podobně jako
v~příkladu \ref{typical structure} na straně \pageref{typical structure}. Muset
ale programovat zohlednění cíle v každém pravidle je přinejmenším otravné. U
větších programů to navíc může být velmi náročné, neboť cíl bude třeba
programově modifikovat v průběhu výpočtu.

S použitím zpětné inference je problém podstatně jednodušší. Zavoláme-li
\begin{minted}{cl}
(reset)
(defgoal (rent-payed))
(back-run),
\end{minted}
bude výsledkem výstup
\begin{minted}[samepage]{cl}
All goals have been satisfied
(RENT-PAYED) satisfied by (RULE PAY-RENT
  (HAVE-MONEY)
  =>
  (RETRACT (HAVE-MONEY))
  (ASSERT (RENT-PAYED)))
(HAVE-MONEY) satisfied by (HAVE-MONEY).
\end{minted}
Zde vidíme, že po spuštění zpětné inference nezačal systém bezhlavě provádět
akce, ke kterým měl dostatečné prostředky. Místo toho systém uvážil zadaný cíl a
jal se hledat akce, které k jeho splnění směřují.

Uvažme nyní složitější příklad:
\begin{minted}[samepage]{cl}
(deffacts world
  (female jane)
  (male john)
  (parent :parent jane :child george)
  (parent :parent john :child george))

(defrule father-is-male-parent
  (male ?father)
  (parent :parent ?father :child ?child)
  =>
  (assert (father :father ?father :child ?child)))

(defrule mother-is-female-parent
  (female ?mother)
  (parent :parent ?mother :child ?child)
  =>
  (assert (mother :mother ?mother :child ?child)))
\end{minted}
Zajímá-li nás, kdo je matkou George, můžeme zkusit spustit dopřednou inferenci.
Po jejím skončení bude v pracovní paměti jak informace o Georgově matce, tak o
jeho otci. Systém se tedy v tomto případě dobral správného výsledku, vypočítal
ale i další fakty, které nás nezajímaly. Dokážeme si snadno představit, že ve
větším programu může být výpočet všech odvoditelných závěrů velmi výpočetně
náročný.

Spustíme-li naopak zpětnou inferenci, je výsledkem
\begin{minted}[samepage]{cl}
All goals have been satisfied
(MOTHER (MOTHER . ?MOTHER-OF-GEORGE) (CHILD . GEORGE)) satisfied by (RULE
MOTHER-IS-FEMALE-PARENT
  (FEMALE ?MOTHER)
  (PARENT (PARENT . ?MOTHER) (CHILD . ?CHILD))
  =>
  (ASSERT (MOTHER MOTHER ?MOTHER CHILD ?CHILD)))
(FEMALE ?MOTHER) satisfied by (FEMALE JANE)
(PARENT (PARENT . JANE) (CHILD . GEORGE)) satisfied by (PARENT
                                                        (PARENT . JANE)
                                                        (CHILD . GEORGE))
These variable bindings have been used:
((?MOTHER-OF-GEORGE . JANE))
\end{minted}


%%%%%%%%%%%%%%%%%%%%%%%%%%%%%%%%%%%%%%%%%%%%%%%%%%%%%%%%%%%%%%%%%%%%%%%%%%%%%%%%
\subsubsection{Funkční alternativy k makrům}
\begin{framed}
  \begin{itemize}
    \item generování specifikací faktů kódem - např. palindromy
    \item použití lispových volání v důsledcích - např. v boxes
  \end{itemize}
\end{framed}

%%%%%%%%%%%%%%%%%%%%%%%%%%%%%%%%%%%%%%%%%%%%%%%%%%%%%%%%%%%%%%%%%%%%%%%%%%%%%%%%
\subsubsection{CLIPSová syntax}
\begin{framed}
  \begin{itemize}
    \item deftemplate, fact specifiery
    \item volání facts s číslem
    \item projít příručku clipsu a vzpomenout si, co v ExiLu ještě nebylo
  \end{itemize}
\end{framed}

%%%%%%%%%%%%%%%%%%%%%%%%%%%%%%%%%%%%%%%%%%%%%%%%%%%%%%%%%%%%%%%%%%%%%%%%%%%%%%%%
\subsubsection{Reset prostředí}
\begin{framed}
  \begin{itemize}
    \item durable/volatile slots
    \item clean, reset, complete reset (neměl by se jmenovat complete clean?)
  \end{itemize}
\end{framed}

%%%%%%%%%%%%%%%%%%%%%%%%%%%%%%%%%%%%%%%%%%%%%%%%%%%%%%%%%%%%%%%%%%%%%%%%%%%%%%%%
\subsubsection{Práce s více prostředími}
\label{multiple environments}

%%%%%%%%%%%%%%%%%%%%%%%%%%%%%%%%%%%%%%%%%%%%%%%%%%%%%%%%%%%%%%%%%%%%%%%%%%%%%%%%
\subsubsection{Grafické uživatelské rozhraní}
