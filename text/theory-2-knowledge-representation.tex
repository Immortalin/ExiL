\subsection{Reprezentace znalostí}
\label{knowledge representation}

% representation language - oriented towards organizing descriptions
% of objects and ideas, rather than stating sequences of instructions or
% storing simple data elements

Aby bylo možné pracovat se znalostmi v expertním systému, musí tyto splňovat
několik vlastností:
\begin{itemize}
  \item přesnost - pro vyvození rozumných závěrů nemůže být znalost vágní (to
    neznamená, že nemůžeme vyjádřit nejistotu, i ta ale musí být definována
    přesně),
  \item relevance - znalosti by se měly týkat pouze pojmů z problémové domény
  \item organizovanost - znalosti by měly postihovat všechny potřebné spojitosti
    mezi pojmy (to zahrnuje například používání vždy stejného pojmu pro tutéž
    věc),
  \item uzavřenost - systém by pro vyvození závěrů neměl potřebovat žádné
    předchozí znalosti.
\end{itemize}

Systém nechápe význam znalostí, pracuje pouze s jejich reprezentací.
Reprezentaci obecně chápeme jako množinu syntaktických a sémantických pravidel,
které umožňují popsat nějaké entity. Reprezentaci entity pak jako výraz, který
podle těchto pravidel popisuje danou entitu. Jazyk reprezentace je množina
všech možných reprezentací entit. V expertních systémech jsou těmito entitami
právě znalosti.

Syntax reprezentace je množina pravidel, která definují základní stavební bloky
(atomy) jazyka reprezentace a jak jejich kombinací vytvářet výrazy tohoto
jazyka, tedy reprezentace jednotlivých entit. Sémantika říká, jak jsou výrazy
jazyka reprezentace interpretovány - jaký je jejich význam. Sémantika nám tedy
dává zobrazení z reprezentace entity na reprezentovanou entitu. Sémantická
pravidla jsou často rekurentní - přiřkneme význam atomům a poté jednotlivým
strukturám, které vznikají jejich kombinací.

Aby byla reprezentace praktická, musí být
\begin{itemize}
  \item jednoznačná - není možné, aby měl jeden výraz jazyka reprezentace
    několik významů,
  \item expresivní - reprezentace musí umožňovat vyjádření všech potřebných
    detailů (\emph{distinction}) popisované entity
  \item srozumitelná - význam výrazu by měl být snadno pochopitelný bez
    nutnosti chápat, jak je systémem zpracováván,
  \item stručná - výrazy by mělo být snadné psát, neměly by být zbytečně
    mnohomluvné;
  \item reprezentaci musí být možno zpracovávat a ukládat v počítači.
\end{itemize}
Míra expresivity je často v protikladu snadného strojového zpracování, hledáme
tedy kompromis, mezi těmito dvěma vlastnostmi.

Často používanými reprezentačními schematy jsou strukturované objekty,
odvozovací pravidla a logické programy. Knihovna ExiL, stejně jako systém CLIPS
používá prvních dvou. Typickou ukázkou třetího je jazyk
Prolog\footnote{\url{http://en.wikipedia.org/wiki/Prolog}}.

\begin{framed}
  \begin{itemize}
    \item ES jakožto rule-based system rozlišuje 2 základní typy znalostí -
      fakty a odvozovací pravidla
    \item různé ES se liší v reprezentaci těchto znalostí
    \item reprezentace faktů a pravidel v STRIPS, odkaz na GPS (nebo až v
      inferenci?)
    \item atomy v CLIPS
    \item fakta v CLIPS - seznamy nebo strukturované objekty z atomů
    \item CLIPS i EXIL - speciální výraz vzoru - v CLIPSU stavební blok - není
      v jazyce reprezentace, v EXILu je i samostatný vzor reprezentací -
      reprezentuje cíl
    \item reprezentace pravidel v CLIPSU a EXILu - syntax a sémantika
      (stavební bloky, význam - podmínky mají existenční charakter)
    \item EXIL - cíle jako vzory (zpětná inference)
    \item dále pracovat s pojmem znalost (reprezentace je implicitní)
    \item stav expertního systému - mna faktů - pracovní paměť
    \item pokud se mohou pravidla v průběhu měnit, patří do stavu i množina
      pravidel - production memory
    \item aplikace pravidla jako přechod mezi stavy
    \item reprezentace problému - počáteční stav, přechody, koncový stav,
      příp. cíle (zpětná inference)
    \item reprezentace problému
      \begin{itemize}
        \item problem solving involves reasoning about actions and states of the
          world (state = current knowledge = describes objects, their properties and
          relationonships)
        \item problems can be formulated in terms of initial state, goal state
          and a set of operations, that can be employed in an attempt to
          transform one state into another
        \item symbolické výpočty - non-numeric computations, manipulation of
          symbolic representations
      \end{itemize}
  \end{itemize}
\end{framed}
