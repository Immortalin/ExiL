\subsection{Reprezentace znalostí}

\begin{framed}
  \begin{itemize}
    \item vlastnosti znalostí zpracovatelných ES
      \begin{itemize}
        \item relevantní
        \item organizovaná (postihuje souvislosti mezi pojmy), extensively
          indexed and content-addressable (should enable cross-referencing)
        \item jednoznačná - i v reprezentaci, která umožnuje rozlišení můžeme
          znalost vyjádřit nejednoznačně
        \item self-contained - can't assume prior knowledge
      \end{itemize}
    \item reprezentace
      \begin{itemize}
        \item "set of syntactic and semantic conventions that make it possible
          to describe things", things ~ knowledge ~ state of the system
          (objects, their properties and relationships)
        \item syntax = "set of rules for combining symbols to form expressions
          in the representation language"
        \item semantics = how expressions so constructed should be interpreted
          - what is the meaning of the forms - typically by assigning meanings
          to individual symbols, than recurently inducing meaning of more
          complex expressions
        \item formally described, well-defined syntax and semantics
        \item unambiguous
        \item processable and storable by computer
        \item logical adequacy - representation should be capable of making all
          distinctions that you want to make
        \item heuristic power - there must be a way of using this representation
          to solve problems - the more expressive language (more possible
          distinctions), the the more difficult to manipulate during inference
          item natational convenience - expressions easy to write, read and
          understand (without knowing, how the computer will interpret them)
        \item declarative - descriptive $\rightarrow$ can be understood without
          knowing what states the program will go though when interpreting the
          representation
        \item example schemes - production rules, structured objects (e.g.
          templates), logic programs
        \item representation language - oriented towards organizing descriptions
          of objects and ideas, rather than stating sequences of instructions or
          storing simple data elements
      \end{itemize}
    \item problem solving involves reasoning about actions and states of the
      world
    \item problems can be formulated in terms of initial state, goal state
      and a set of operations, that can be employed in an attempt to
      transform one state into another
    \item symbolické výpočty - non-numeric computations, manipulation of
      symbolic representations
    \item reprezentace stavu pomocí symbolických struktur, inference jako jejich
      manipulace
    \item fakty a odvozovací pravidla
  \end{itemize}
\end{framed}
