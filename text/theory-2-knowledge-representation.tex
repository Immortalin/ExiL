\subsection{Reprezentace znalostí}
\label{knowledge representation}

% representation language - oriented towards organizing descriptions
% of objects and ideas, rather than stating sequences of instructions or
% storing simple data elements

Aby bylo možné pracovat se znalostmi v expertním systému, musí tyto splňovat
několik vlastností:
\begin{itemize}
  \item přesnost - pro vyvození rozumných závěrů nemůže být znalost vágní (to
    neznamená, že nemůžeme vyjádřit nejistotu, i ta ale musí být definována
    přesně),
  \item relevance - znalosti by se měly týkat pouze pojmů z problémové domény
  \item organizovanost - znalosti by měly postihovat všechny potřebné spojitosti
    mezi pojmy (to zahrnuje například používání vždy stejného pojmu pro tutéž
    věc),
  \item uzavřenost - systém by pro vyvození závěrů neměl potřebovat žádné
    předchozí znalosti.
\end{itemize}

Systém nechápe význam znalostí, pracuje pouze s jejich reprezentací.
Reprezentaci obecně chápeme jako množinu syntaktických a sémantických pravidel,
které umožňují popsat nějaké entity. Reprezentaci entity pak jako výraz, který
podle těchto pravidel popisuje danou entitu. Jazyk reprezentace je množina
všech možných reprezentací entit. V expertních systémech jsou těmito entitami
právě znalosti.

Syntax reprezentace je množina pravidel, která definují základní stavební bloky
(atomy) jazyka reprezentace a jak jejich kombinací vytvářet výrazy tohoto
jazyka, tedy reprezentace jednotlivých entit. Sémantika říká, jak jsou výrazy
jazyka reprezentace interpretovány - jaký je jejich význam. Sémantika nám tedy
dává zobrazení z reprezentace entity na reprezentovanou entitu. Sémantická
pravidla jsou často rekurentní - přiřkneme význam atomům a poté jednotlivým
strukturám, které vznikají jejich kombinací.

Aby byla reprezentace praktická, musí být
\begin{itemize}
  \item jednoznačná - není možné, aby měl jeden výraz jazyka reprezentace
    několik významů,
  \item expresivní - reprezentace musí umožňovat vyjádření všech potřebných
    detailů (\emph{distinction}) popisované entity
  \item srozumitelná - význam výrazu by měl být snadno pochopitelný bez
    nutnosti chápat, jak je systémem zpracováván,
  \item stručná - výrazy by mělo být snadné psát, neměly by být zbytečně
    mnohomluvné;
  \item reprezentaci musí být možno zpracovávat a ukládat v počítači.
\end{itemize}
Míra expresivity je často v protikladu snadného strojového zpracování, hledáme
tedy kompromis, mezi těmito dvěma vlastnostmi.

Často používanými reprezentačními schematy jsou strukturované objekty,
odvozovací pravidla a logické programy. Knihovna ExiL, stejně jako systém CLIPS
používá prvních dvou. Typickou ukázkou třetího je jazyk
Prolog\footnote{\url{http://en.wikipedia.org/wiki/Prolog}}.

Expertní systémy jsou \emph{systémy založené na
pravidlech}\footnote{\url{http://en.wikipedia.org/wiki/Rule-based\_system}}.
Jako takové rozlišují dva základní typy znalostí - \emph{fakty} a
\emph{pravidla}. Fakt je elementární \uv{statická} znalost - vyjadřuje nějaké
tvrzení z problémové domény, např. \uv{obloha je jasná}. Odvozovací pravidlo je
pak elementární odvozovací znalostí a má formu implikace. Pravidlo vyjadřuje,
jaké (jaká) tvrzení můžeme odvodit z platnosti jiných tvrzení, např. \uv{pokud
je obloha jasná, neprší}. Pravidlo tedy sestává z \emph{podmínek} (\uv{obloha je
jasná}) a \emph{důsledků} (\uv{neprší}).

\emph{Stav systému} je tvořen množinou aktuálně platných tvrzení a reprezentován
množinou reprezentací odpovídajících faktů. Tvrzení tedy platí, pokud je
reprezentace příslušného faktu součástí stavu systému. \emph{Aplikace
odvozovacího pravidla} pak definuje přechod mezi dvěma stavy systému. Pravidlo
je možné aplikovat, jsou-li splněny jeho podmínky. Při jeho aplikaci jsou
vyhodnoceny jeho důsledky, na základě čehož je stav systému modifikován -
reprezentace nějakých faktů jsou přidány do a/nebo odebrány z množiny, která
stav reprezentuje.

Počáteční stav expertního systému spolu s množinou definovaných odvozovacích
pravidel tvoří tzv. \emph{znalostní bázi} systému. Pokud se může množina
definovaných pravidel v průběhu práce se systémem měnit, zařazujeme ji také do
stavu systému. Ten je pak reprezentován dvěma množinami - množinou reprezentací
faktů (tzv. \emph{pracovní paměť}) a množinou definovaných pravidel (tzv.
\emph{produkční paměť}).

Pojmy pracovní a produkční paměť nejsou příliš intuitivní. Jde o doslovný
překlad v~literatuře užívaných pojmů \emph{working memory} a \emph{production
memory}. Nejde ve skutečnosti o paměťi, nýbrž o obsahy pomyslných pamětí. Pojmy
\uv{pracovní množina faktů} a \uv{pracovní množina pravidel} by byly jistě
výstižnější, bohužel ale také značně těžkopádné.

Různé expertní systémy se liší v reprezentaci faktů a pravidel. V různých
expertních systémech tedy můžeme vyjádřit znalosti s různou mírou flexibility.
Jednoduchým příkladem reprezentace faktů a pravidel je reprezentace expertního
systému STRIPS\footnote{\url{http://en.wikipedia.org/wiki/STRIPS}}. Atomy sytaxe
této reprezentace jsou symboly. Ty reprezentují názvy jednotlivých objektů a
relací z problémové domény. Fakty jsou pak ve tvaru
\verb|<relace>(<argumenty oddělené čárkami>)|, např. \verb|At(robot,roomA)|.

Pro vyšší flexibilitu pravidel definuje často syntaxe expertního systému
speciální výraz, tzv. \emph{vzor}. Ten má typicky stejnou strukturu jako fakt,
ale může obsahovat speciální atomy - \emph{proměnné}. Podmínky a důsledky
pravidel jsou pak tvořeny nikoli fakty, nýbrž právě vzory. Díky tomu mohou být
podmínky pravidel splněny mnoha různými skupinami faktů.

Při vyhodnocování splnění podmínek pravidla srovnáváme vzory podmínek s
konkrétními fakty. Každý vzor podmínky je spárován s jedním faktem, který má
stejnou strukturu a stejné hodnoty neproměnných atomů. Konkrétní atom, který se
ve faktu vyskytuje na pozici proměnné vzoru, se kterým byl spárován, označujeme
jako \emph{vazbu} této proměnné. Při vyhodnocování pravidla je zajištěna
konzistence těchto vazeb. Vyskytuje-li se táž proměnná na více místech ve
vzorech podmínek pravidla, musí mít vždy stejnou vazbu. Při vyhodnocení pravidla
jsou pak v jeho důsledcích použity místo proměnných jejich vazby.

Odvozovací pravidlo systému STRIPS je reprezentováno názvem, seznamem použitých
proměnných, podmínkami a důsledky. Podmínky i důsledky pravidel jsou
reprezentovány posloupnostmi vzorů. Před vzorem v důsledcích pravidla může
být navíc použit speciální symbol \verb|not|. Ten vyjadřuje, že fakt, získaný
nahrazením proměnných vzoru jejich vazbami, má být z pracovní paměti systému
odebrán, nikoli do ní přidán.  Definice odvozovacího pravidla STRIPSu může tedy
vypadat následovně:
\begin{verbatim}
// move the box up a level
_MoveUp(Box,Lift,Location)_
Preconditions:  At(Box,Location), At(Lift,Location), Level(Box,low)
Postconditions: Level(Box,high), not Level(Box,low).
\end{verbatim}

Pro potřeby definice sémantiky odvozovacího pravidla zavedu pojem
\emph{kongruence} faktu se vzorem (případně faktu a vzoru, nebo i naopak). Fakt
a vzor jsou kongruentní, pokud mají stejnou strukturu, tedy stejný název
predikátu a stejný počet jeho argumentů a pokud jsou neproměnné atomy vzoru
stejné, jako atomy na odpovídajících pozicích faktu.

Podmínky pravidel STRIPSu jsou spojeny logickou konjunkcí a mají existenční
charakter. Význam podmínek pravidla je tedy \uv{existuje posloupnost
\emph{n}~faktů, kde \emph{n}~je počet podmínek pravidla, a jednotlivé fakty
jsou kongruentní s odpovídajícími vzory jeho podmínek při zachování konzistence
vazeb proměnných.} Pokud je toto tvrzení splňeno, lze pravidlo aplikovat výše
uvedeným způsobem.

\begin{framed}
  \begin{itemize}
    \item sémantika podmínek - existenční charakter
    \item reprezentace pravidel v CLIPSU a EXILu - syntax a sémantika
      (stavební bloky, význam - podmínky mají existenční charakter)
    \item EXIL - cíle jako vzory (zpětná inference)
    \item dále pracovat s pojmem znalost (reprezentace je implicitní)
    \item výpočet jako posloupnost stavů s přechody danými aplikací pravidel
    \item reprezentace problému - počáteční stav, přechody, koncový stav,
      příp. cíle (zpětná inference)
    \item reprezentace problému
      \begin{itemize}
        \item problem solving involves reasoning about actions and states of the
          world (state = current knowledge = describes objects, their properties and
          relationonships)
        \item problems can be formulated in terms of initial state, goal state
          and a set of operations, that can be employed in an attempt to
          transform one state into another
        \item symbolické výpočty - non-numeric computations, manipulation of
          symbolic representations
      \end{itemize}
  \end{itemize}
\end{framed}
