\subsection{Implementace}
\begin{framed}
Implementace or v podmínkách pravidla by vyžadovala výraznou změnu v algoritmu,
který vytváří síť RETE, neboť join node je problematické znovu využít.
Implementace forall by vyžadovala změnu vyhodnocování vazeb proměnných a celkově
join algoritmu. Zvážit problémy implementace všech podobných rozšíření - or,
and, negace celé konjunkce či disjunkce, exists, forall, viz
\url{http://www.csie.ntu.edu.tw/~sylee/courses/clips/bpg/node5.4.html}
\end{framed}

\begin{framed}
  \begin{itemize}
    \item architektura programu (objektový návrh, nedostatky Lispu, je
      environment jako god class důsledkem těchto problémů?)
    \item síť RETE, její vytváření $\rightarrow$ výhody - sdílení uzlů
    \item implementovaná rozšíření
    \begin{itemize}
      \item undo/redo - kopie sítě rete, testování ekvivalence
      \item zpětné řetězení - zásobníky, backtracking, obchází rete
      \item kompozitní podmínky (not, and, or, forall) - neimplementováno -
        popsat, jak výrazné změny rete by vyžadovalo a jak by se projevilo na
        efektivitě (hlavní výhodou rete je sdílení uzlů, které by zde bylo dost
        problematické)
      \item syntaktický mód - parser, implementace složených podmínek pro atom
        (\~{}asdf\&asf) by vyžadovalá podobné změny jako implementace kompozitních
        podmínek
      \item gui - capi, propojení s environmentem - notify
    \end{itemize}
  \end{itemize}
\end{framed}
