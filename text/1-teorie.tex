\section{Co je to experní systém}
Expertní systém je počítačový program, který ze zadané znalostní báze pomocí
zadaných pravidel odvozuje nová fakta. Díky tomu je jej možno využít v~praxi
jako asistenta odborníka v~daném oboru, nebo jej, v~ideálním případě, zcela
nahradit.

Pro lepší představu uvedu příklad - expertní systém v~ordinaci
praktického lékaře. Na začátku zadá lékař, případně za pomoci znalostního
inženýra, expertnímu systému znalostní bazi, tj. informace o~příznacích
známých chorob, možnostech jejich léčby, medikaci, konfliktních léků, atd.
Expertní systém potom při zadání příznaků pacienta s~jistou pravděpodobností
(odvislé od pravděpodobností zadaných v~jednotlivých pravidlech příznak
$\rightarrow$ choroba) určí možné choroby, alternativy léčby, apod.

\subsection{Co odlišuje expertní systém od jiných výpočetních programů}
Toto samozrejmě může dělat i~program, který bychom za expertní systém
neoznačili, rozdíl je v~tom, že expertní systém je obecný - znalostní
baze je zcela oddělena řídícího mechanismu. Tento je tudíž zcela
nezávislý na konkrétní doméně. Takto tomu u~jiných programů většinou
nebývá, data bývají smýchána s~rozhodovacím a~řídícím kódem programu
a~program je tudíž jednoúčelový.

\subsection{Charakteristické vlastnosti expertních systémů}
