\subsubsection{Inference}
\label{inference}

Inference (odvozování nových faktů z aktuálních) probíhá v krocích. V každém
kroku jsou vyhodnoceny podmínky všech pravidel, načež je ze splněných pravidel
vybráno jedno, které je posléze aktivováno. Inferenční kroky
můžeme buď spouštět jednotlivě voláním funkce \verb|step| (bez parametrů), nebo
voláním funkce \verb|run| spustit cyklus, který provádí inferenční kroky, dokud
je to možné. Cyklus je buď přerušen ve chvíli, kdy není splněno žádné
další pravidlo, nebo voláním funkce \verb|halt| (bez parametrů) v důsledcích
právě aktivovaného pravidla.

Podmínky pravidel jsou ve tvaru vzorů a jsou spojeny logickou konjunkcí,
pravidlo je tedy splněno, jsou-li splněny všechny jeho podmínky. Kromě
toho mohou být některé podmínky negovány. Taková podmínka je splněna tehdy,
neexistuje-li v pracovní paměti žádný fakt, který by se shodoval s jejím vzorem.
Negovanou podmínku značí znak \verb|-| (minus) na prvním místě specifikace vzoru.

Vyhodnocování podmínek pravidel probíhá ve dvou fázích. V první fází
srovnáváme vzory jednotlivých podmínek se všemi fakty v pracovní paměti a~to
pouze strukturálně, tedy bez ohledu na vazby proměnných. Prvním požadavkem shody
je u jednoduchých faktů stejná délka faktu (počet atomů), u strukturovaných
faktů stejná šablona. Jednoduchý fakt se nikdy nemůže shodovat se strukturovaným
vzorem a naopak.

Dále jsou pak porovnávány jednotlivé atomy (u jednoduchých) či
sloty (u složených) faktu vůči odpovídajícímu atomu (slotu) vzoru. Není-li atom
(slot) vzoru proměnná, je jednoduše porovnán s atomem faktu. Je-li atomem
proměnná, považujeme jej v této fázi automaticky za shodu. Např. vzor
\cl|(in :object robot :location ?loc)|
se shoduje s faktem
\cl|(in :object robot :location A)|
nikoli však s fakty
\begin{minted}{cl}
(in :object box :location A)
(is-in :object robot :location A)
(in robot A).
\end{minted}
Tímto předvýběrem
tedy získáme ke každé podmínce pravidla množinu faktů, které mají stejnou
strukturu a stejné hodnoty neproměnných atomů.

Ve druhé fázi vyhodnocování hledáme z předvybraných faktů takovou posloupnost
(délka odpovídá počtu podmínek pravidla), kde po spárování s odpovídajícími
vzory podmínek obdržíme konzistentní vazby proměnných. To znamená, že
vyskytuje-li se v podmínkách pravila některá proměnná vícekrát, musí mít
odpovídající fakty na daných pozicích stejný atom.  Mějme např. pravidlo s
podmínkami
\begin{minted}{cl}
(goal move ?obj ?from ?to)
(in :object ?obj  :location ?from)
(in :object robot :location ?to)
\end{minted}
Vzor první podmínky je jednoduchý, zatímco další dva jsou strukturované. To ale
ničemu nevadí, je třeba pouze najít fakty odpovídající struktury. Posloupnost
faktů
\begin{minted}{cl}
(goal move box A B)
(in :object box   :location B)
(in :object robot :location A)
\end{minted}
neprojde druhou fází výběru, neboť vazby proměnných nejsou konzistentní.
Proměnná \verb|?from| je např. v první podmínce navázána na symbol \verb|A|, v
druhé ale na \verb|B|. Kdyby si ovšem krabice s robotem vyměnily pozice, budou
vazby proměnných konzistentní a podmínky pravidla budou splněny. Proměnná
\verb|?from| by pak nabyla hodnoty \verb|A|, proměnná \verb|?to| hodnoty
\verb|B| a proměnná \verb|?obj| hodnoty \verb|box|.

Vyhodnocení negovaných podmínek si můžeme představit tak, že nejprve vyhodnotíme
a navážeme proměnné všech ostatních podmínek. Pokud poté neexistuje fakt, který
by se s vzorem negované podmínky shodoval a měl konzistentní vazby se zbytkem navázaných
proměnných, je tato podmínka splněna. Mějme například pravidlo s podmínkami
\begin{minted}{cl}
(goal move box ?from ?to)
(in box ?from)
(- in robot ?from).
\end{minted}
Máme-li v pracovní paměti pouze fakty
\begin{minted}{cl}
(goal move box A B)
(in box A)
(in robot B),
\end{minted}
budou podmínky pravidla splněny, neboť po spárování vzorů prvních dvou podmínek
s prvními dvěma fakty bude proměnná \verb|?from| navázána na hodnotu \verb|A| a
neexistuje fakt, který by se shodoval s vzorem \verb|(in robot A)|. Přesuneme-li
ale robota na pozici \verb|A|, podmínka již splněna nebude a pravidlo nelze
aktivovat.

Ve vzorech podmínek pravidla můžeme využít speciální proměnné \verb|?|.
Konzistence vazby této proměnné není při vyhodnocování testována, takže
vyskytuje-li se tato proměnná na více místech, chová se tak, jako kdyby byl
každý výskyt označen unikátním názvem (podobně jako proměnná \verb|_| v
Prologu). Použitím této proměnné dáváme najevo, že nás konkrétní hodnota daného
atomu nazajímá. Ve strukturovaných podmínkách není třeba tyto sloty uvádět,
neboť \verb|?| je výchozí hodnotou slotu vzoru.

Posledním speciálním konstruktem je navázání celého faktu na proměnnou. Např.
pravidlo
\begin{minted}{cl}
(defrule move
  ?fact <- (in :object ? :location A)
  =>
  (modify ?fact :location B))
\end{minted}
přesune každý objekt z pozice \verb|A| na pozici \verb|B|. Na proměnnou
můžeme navázat i jednoduchý fakt, pak ale nemůžeme použít makra \verb|modify|.
Můžeme ovšem volat \verb|(retract ?fact)|, neboť proměnná \verb|?fact| je při
aktivaci pravidla nahrazena specifikací faktu, který byl s vzorem podmínky
spárován.

\begin{framed}
  Pravidlo může být splněno několika posloupnostmi faktů - pojem matche
\end{framed}

Je-li při vyhodnocování podmínek nalezeno více splněných pravidel, je třeba z
nich jedno vybrat k aktivaci. Výběr pravidla záleží na zvolené strategii.
ExiL poskytuje následující strategie výběru pravidla:
\begin{description}[leftmargin=6cm,style=sameline,align=right,labelsep=0.5cm]
  \item[depth-strategy]
\end{description}

\begin{framed}
  \begin{itemize}
    \item fáze podrobně
    \begin{itemize}
      \item výběr pravidla - strategie
      \item aktivace - vyhodnocení důsledků (typicky modifikace pracovní paměti)
        navázání proměnných, eval
    \end{itemize}
    \item spuštění inference, krokování (může se prolínat s ručnímodifikací w.m.)
    \item queries - agenda, strategies
  \end{itemize}
\end{framed}
