%%%%%%%%%%%%%%%%%%%%%%%%%%%%%%%%%%%%%%%%%%%%%%%%%%%%%%%%%%%%%%%%%%%%%%%%%%%%%%%%
\subsubsection{CLIPSová syntax}
\begin{framed}
  \begin{itemize}
    \item deftemplate, fact specifiery
    \item volání facts s číslem
    \item projít příručku clipsu a vzpomenout si, co v ExiLu ještě nebylo
    \item typy
    \item multisloty
  \end{itemize}
\end{framed}

Dalším z požadavků zadání práce bylo přiblížit syntax exilových volání systému
CLIPS, aby bylo možné programy v něm napsané snáze převést na programy exilové.
Toho se mi podařilo dosáhnout jen částečně.

Systém CLIPS použivá jiný formát specifikací slotů šablony, strukturovaných
faktů a požadovaných změn při volání \verb|modify|. Tuto syntax nyní ExiL podporuje
také. Příklad \ref{clips syntax} na straně \pageref{clips syntax} ukazuje
definici znalostní báze ekvivalentní příkladu \ref{structured facts} s použitím
CLIPSové syntaxe. Syntax je dostatečně odlišná na to, aby ji ExiL rozpoznal,
není tedy třeba syntaktický mód nijak přepínat. Díky tomu dokonce můžeme obě
syntaxe kombinovat.

ExiL také po vzoru CLIPSu umožňuje omezit seznam vrácený voláním \verb|(facts)|
volitelnými číselnými parametry. První volitelný parametr udává index prvního
faktu v seznamu (číslováno od 1). Druhý parametr udává index posledního faktu.
Třetí parametr pak maximální počet vrácených faktů.

Makra \verb|assert| a \verb|retract| také nyní umožňují přidání či odebrání
více faktů najednou. Makru \verb|retract| lze navíc místo specifikací faktů k
odstranění předat jejich číselné indexy v seznamu faktů. Obě možnosti lze
dokonce kombinovat.


\begin{listing}[t]
\caption{Definice znalostní báze s použitím CLIPSové syntaxe}
\label{clips syntax}
\begin{clcode}
(deftemplate goal
  (slot action (default move))
  (slot object)
  (slot from)
  (slot to))

(deftemplate in
  (slot object)
  (slot location))

(deffacts world
  (in (object robot) (location A))
  (in (object box) (location B))
  (goal (object box) (from B) (to A))).

(defrule move-robot
  (goal (action move) (object ?obj) (from ?from))
  (in (object ?obj) (location ?from))
  (- in (object robot) (location ?from))
  ?robot <- (in (object robot) (location ?z))
  =>
  (modify ?robot (location ?from)))

(defrule move-object
  (goal (action move) (object ?obj) (from ?from) (to ?to))
  ?object <- (in (object ?obj) (location ?from))
  ?robot <- (in (object robot) (location ?from))
  =>
  (modify ?robot (location ?to))
  (modify ?object (location ?to)))

(defrule stop
  ?goal <- (goal (action move) (object ?obj) (to ?to))
  (in (object ?obj) (location ?to))
  =>
  (halt))
\end{clcode}
\end{listing}

\FloatBarrier
