\subsection{Inference}

V přechozích sekcích jsem definoval stav expertního systému a aplikaci
odvozovacího pravidla jako přechod mezi dvěma stavy. Výpočet expertního systému,
tedy inferenci, můžeme definovat jako posloupnost stavů systému, mezi nimiž
systém přechází aplikacemi odvozovacích pravidel. Výpočet ovšem může probíhat
dvěma různými směry.

\textbf{Expertní systém s dopředným řetězením} (inferencí) vychází při výpočtu z
výchozího stavu systému, tedy ze stavu, kdy jsou v pracovní paměti fakty ze
znalostní báze. Poté v krocích vyhodnocuje, která odvozovací pravidla mají
splněné podmínky a které posloupnosti faktů z pracovní paměti je splňují.
Z~těchto dvojic (pravidlo, posloupnost faktů), zvaných shody (\emph{match}) pak
systém jednu vybere a aktivuje pravidlo s použitím vazeb proměnných získaných
navázáním vzorů podmínek na odpovídající fakty. Takto systém postupuje, dokud není
inference zastavena, nebo už neexistuje žádná další shoda.

Takovým systémem je i CLIPS. Výsledkem výpočtu systému je tedy určitý stav, ve
kterém se inference zastavila, spolu s posloupností použitých shod. K~ovlivnění
průběhu inference, tedy výběru shody, která bude v daném kroku aplikována,
slouží inferenční strategie. Ty definují uspořádání množiny shod, ze kterých je
pak vybrána první. Několik inferenčních strategií popíšu v sekci \ref{inference}

\textbf{Expertní systém se zpětným řetězením} (inferencí) naopak umožňuje
definici cílů, kterých chceme dosáhnout. Systém poté hledá pravidla, která
vedou ke splnění jednotlivých cílů, zkoumáním jejich důsledků. Takovýto způsob
výpočtu nazývame také v oblasti umělé inteligence jako \emph{means-end
analýza}\footnote{\url{http://en.wikipedia.org/wiki/Means-ends\_analysis}}.
Výhodou zpětného řetězení je právě možnost zadání konkretních cílů, jejich
implementace je ale náročnější a proto tyto systémy často poskytují omezenější
možnosti co do definice odvozovacích pravidel. Systém CLIPS například umožňuje
použití libovolných volání jeho jazyka v důsledkové části pravidel. Důsledky
těchto volání ale nelze obecně předvídat, systém se zpětným řetězením tedy
takovou možnost poskytovat nemůže.

Systém STRIPS používá zpětné inference. Pravidla se tedy ve skutečnosti
neaplikují tak, jak jsem popsal v sekci \ref{knowledge representation} Přesto o
nich z uživatelského hlediska můžeme takto uvažovat. Tato představa je pro
konstrukci pravidel přímočařejší, navíc nás posloupnost aplikovaných pravidel
většinou zajímá v pořadí jejich pomyslné aplikace, tedy proti směru inference.

Cíle v systému STRIPS zadáváme ve formě vzorů, stejně jako podmínky pravidel.
Pokud systém zvolí k aktivaci nějaké odvozovací pravidlo, protože jeho aktivace
vede ke splnění aktuálního cíle, přidá podmínky tohoto pravidla do množiny
cílů. Navázání proměnných funguje při aplikaci pravidla opačně, než jsem
u~aplikace pravidla popsal.

V případě splnění všech cílů, zadaných uživatelem i odvozených při aplikaci
pravidel, je výsledkem výpočtu posloupnost pravidel, které je třeba aplikovat,
abychom přešli z počátečního stavu systému do cílového. Pokud obsahovaly zadané
cíle proměnné, je součástí odpovědi také množina použitých vazeb těchto
proměnných, tedy jejich substituce.

U expertních systémů můžeme také definovat cíle výpočtu. Tyto ale musíme přidat
jako fakta do znalostní báze a v pravidlech je musíme explicitně testovat,
případně modifikovat, což může být velmi složité.

Budeme-li uvažovat všechny možné cesty, kterými se systém může v každém stavu
ubírat, ať už aplikací splněného pravidla, nebo pravidla, jehož důsledky vedou
ke splnění cíle, může možné dosažitelné stavy uspořádat do stromu. Ten bude mít
kořen buď v počátečním stavu (u systému s dopředným), nebo v cílovém (u systému
se zpětným řetězením).

Výpočet si tedy můžeme představit také jako prohledávání tohoto stromu stavů.
Protože množství dosažitelných stavů je často obrovské (narůstá exponenciálně
vzhledem k počtu kroků výpočtu), systém neprohledává celý strom, nýbrž používá
heuristik k rozhodnutí, kterou větví výpočtu se ubírat.  Těmito heuristikami
jsou právě zmíněné inferenční strategie.

Přiložená knihovna ExiL implementuje jak dopřednou, tak velmi omezenou zpětnou
inferenci. Její použití a implementaci popisuje praktická část práce.
